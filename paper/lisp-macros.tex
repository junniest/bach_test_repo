\subsection{Macros in Lisp}
Due to the Lisp's fully parenthesized Polish prefix syntax notation there is a
powerful macro engine. We are going to cover its main principles and
advantages.
\begin{enumerate}
    \item There is no need to mark out a structure from a token sequence for
    the internal representation (IR). Every sequence of tokens in parenthesis
    shapes an expression called \emph{form}. A program on Lisp represents a tree
    of nested forms. This property is called \emph{homoiconicity}, as the
    source code of a program can be proceeded by means of the same language. In
    this case it is possible to consider a macro-definition as a left tree
    substitution for one in the right part. For example, 
    \begin{verbatim}
    (defmacro sum (x y z) 
    (list '+ (list '* x y) (list '* z z))
    \end{verbatim} 
    macros defines a list substitution for the nested list structure.
    \item Any valid expression in Lisp represents a form. Even the whole
    program is a form.  Therefore, it is possible to replace huge parts of a
    program and even the whole program.  
    \item Lisp's compiler does not match the whole expression (or a form) in
    order to find out either a corresponding macros is defined. Due to the
    prefix notation a `keyword' is the first token in a form.  If a macro
    definition is occured in the macro table under the same keyword, then we
    have to perform a macro substitution.  \\
    However, this simplicity has a disadvantage too. Macros overloading is
    not allowed in Lisp.  
    \item A macro processor in Lisp not only substitutes
    expressions, but also evaluates them. There is an expression-value table in
    Lisp, where each expression is associated with it's value. For example, if
    we write \verb|(setq a 3)|, then an expression \verb|a| has a corresponding
    value \verb|3|. Consequently, if we create a list \verb|(list a (+ a 1))|,
    a list \verb|(3 4)| will be returned, as this expression will be evaluated
    during compilation. This emphasizes an interpetive nature of the language.
    Although expression evaluation is not covered in the Common Lisp standart,
    many compilers, such as GNU CLisp, CMU Common List support this feature.
\end{enumerate}
