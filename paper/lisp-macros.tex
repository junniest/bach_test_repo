\subsection{Macros in Lisp}
In some way the concept of Lisp is very close to the idea of dynamic parser
described here. Instead of operating with strings, Lisp deals with tokens.
However, the syntax of Lisp is very different from the most programming
languages. This makes macro transformation an efficient tool in Lisp. The main
question here is to figure out are we capable to have a similar functionality
in our dynamic parser regardless syntax differences. \\
A Lisp program is represented as a nested list structure. Each list
represents a form which can be transformed or evaluated. The homoiconicity in
Lisp allows to simplify macro processing significantly. Consider the problem
mentioned in section \ref{sec:intro}: \verb|log (x) - log (y)| -- expression
which could be interpreted differently depending on the context. In Lisp each
case would have a separate representation: \verb|((log (- (log x))) y)| or
\verb|(- (log x) (log y))|. You can see that there is no need for compiler to
guess the context. \\
There is another advantage of such approach. The prefix notation allows to make
a decision depending on the first character in the list. Particularly, there 
are two options:
\begin{enumerate}
    \item Mark expression as a function call, providing  the first character is 
          defined in the function table.
    \item Evaluate the expression given a macro definition. 
\end{enumerate}
A macro expansion consists of two steps as well, which are performed
sequentionally:
\begin{enumerate}
    \item Build an expression specified by macro definition.
    \item Evaluate generated expression.
\end{enumerate}
These steps are distinguished as the first one operates with expressions, and
the second one deals with its values. To perform evaluation an expression-value
table is needed, because when expression transformation can't be performed, the
evaluation process begins. Macro expansion performance capabilities generally
depend on evaluation. To improve evaluation there are several incremental Lisp
compilers. These compilers can evaluate expressions in macro when all necessary
information become available. \\
It should be pointed out that it is possible to build a dynamic parser similar
to the one in Lisp, however some issues are to be handled:
\begin{itemize}
    \item Given a program represented as a sequence of tokens, it is necessary
    to resolve the context unambiguously. In some particular cases it could be
    a tricky task, as a number of look-ahead tokens is unbounded.
    \item There is no `decisive' token in the sequence. We are to use regular
    expressions to parse the whole expression. 
\end{itemize}
