\section{Introduction}
\label{sec:intro}
Most of the modern programming language syntax cannot be formulated
using a context free grammar only.  The problem is that rich
syntax very often comes with a number of ambiguities.  Consider the 
following examples:
\begin{enumerate}
    \item The classical example from C language is a type-cast 
          syntax.  As a user can define an arbitrary type using
          \verb|typedef| construct, the type casting expression
          \verb|(x) + 5| is undecidable, unless we know if
          \verb|x| is a type or not.
    \item Assume that we extend C syntax to allow an array 
          concatenation using infix binary \verb|++| operator and
          constant-arrays to be written as \verb|[1, 2, 3]|.
          We immediately run into the problem to disambiguate the 
          following expression: \verb|a ++ [1]|, as it could mean
          an application of postfix \verb|++| indexed by \verb|1|
          or it could be an array concatenation of \verb|a| and 
          \verb|[1]|.
    \item Assuming the language allows any unary function to be
          applied as infix, postfix and standard notation, we cannot 
          disambiguate an expression \verb|log (x) - log (y)|, if
          we allow unary application of postfix minus.  Potential
          interpretations are: \verb|log (- log (x)) (y)| which is
          obviously an error, or \verb|minus (log (x), log (y))|.
\end{enumerate}

Sometimes it may be the case that context influences not only
parsing decisions but also the lexing decisions.  Consider
the following examples:
\begin{enumerate}
    \item C++ allows nested templates, which means that one could
          write an expression \verb|template <type foo, list <int>>|, 
          assuming that the last \verb|>>| is two closing groups.  In
          order to do that, the lexer must be aware of this context,
          as in a standard context character sequence \verb|<<| means
          shift left.
    \item Assuming that a programmer is allowed to define her own 
          operators, the lexer rules must be changed, in case 
          the name of the operator extends the existing one.  For
          example, assume one defined an operation \verb|+-|.
          It means that from now on an expression \verb|+-5| should
          be lexed as (\verb|+-|, \verb|5|), rather than (
          \verb|+|, \verb|-|, \verb|5|).
\end{enumerate}

In order to resolve the above ambiguities using LALR parser generator
engine, we have to make sure that one can annotate the grammar with a correct
choices for each shift/reduce or reduce/reduce conflict, which puts
a number of restrictions on the execution engine.  Secondly, we have
to implement the context support, which means that we need to have a
mechanism which would not interfere with conflict-resolution.
Finally, one has to have an interface to a lexer in case lexing becomes
context-dependent, and it may be integrated with an error-recovery
mechanism.

Having said that, we may see that using parser generators could be of
the same challenge as writing a parser by hands, where all the
ambiguities could be carefully resolved according to the language
specification.  As it turns out most of the complicated languages
front-ends use hand-written recursive descendent parsers, specially
treating ambiguous cases.  For example the following languages do:
C/C++/ObjectiveC in GNU GCC~\cite{gcc}, clang in LLVM~\cite{}, 
javascript in google V8~\cite{}.

