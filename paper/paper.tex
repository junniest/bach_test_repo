\documentclass[a4paper]{llncs}

\usepackage[T2A]{fontenc}
\usepackage[utf8x]{inputenc}
\usepackage[russian, english]{babel}
\usepackage{color}
\usepackage[colorlinks,bookmarksopen,bookmarksnumbered,citecolor=red,urlcolor=red]{hyperref}
\usepackage{graphicx}
\usepackage{listings}

\definecolor{lbcolor}{rgb}{0.95,0.95,0.95}
\newcommand{\codesize}{\fontsize{8}{8}\selectfont}
\lstset{% general command to set parameter(s)
    language=C,
    basicstyle=\codesize,          % print whole listing small
    %keywordstyle=\color{black}\bfseries,
                                % underlined bold black keywords
    identifierstyle=,           % nothing happens
    stringstyle=\ttfamily,      % typewriter type for strings
    showstringspaces=false%
}


\newcommand{\fixme}[1]{\vskip 5mm\noindent{\bf FIXME}: {\it #1}}

\author{Jūlija Pečerska\inst{1}, Artjoms Šinkarovs\inst{2}, Pavels Zaičenkovs\inst{3}}
\date{\today}
\title{On dynamic extensions of context-dependent parser\\
       --- Extended Abstract ---}
\institute{
  University of Latvia,
  Raiņa bulvāris 19, Rīga,
  Latvija, LV-1586
\and
  Heriot-Watt University,
  Riccarton, Edinburgh,
  EH14 4AS, United Kingdom
\and
  Moscow Institute of Physics and Technology,
  141700, 9, Institutskii per., Dolgoprudny, 
  Moscow Region, Russia
}


\begin{document}
\maketitle

\begin{abstract}
The idea of using a preprocessor before the actual compilation 
is proven to be useful, however, the automatic verification of
such a programs is extremely hard.  The main difficulty comes
from conceptual separation of grammatical rules of the language
and substitution mechanisms underneath a preprocessor.  C/C++
preprocessor is included into the standard of the languages,
however, expressiveness of this tool is very limited in both
computational power, and syntax extension capabilities.  The
modern programming languages nowadays define a syntax which 
cannot be described precisely using context free grammars.
This problem is well known, and as a solution there are parser
generators like ANTLR which generates LL(*) parsers rather than
LALR/LR.  Nevertheless, a lot of real-world parsers are 
implemented by hand.

In this paper we present a way of building a preprocessor which 
allows dynamic changes of the grammar and which works on top of
any recursive descent parser which meets a certain requirements.
The preprocessing step consists of matching a list of tokens 
against a regular expression built on tokens and encoded 
production names of the given grammar and transforming the
result using a functional language.  We demonstrate that using
the proposed approach it becomes possible to i) give a static
guarantees regarding the preprocessing rules; ii) safely express
non-trivial syntactical constructions; and iii) perform a 
restricted partial evaluation.
\end{abstract}

\section{\label{sec:intro}Introduction}

Very often expressiveness of a programming language introduces a number of
ambiguities in its syntax.  The language specification clearly states how to
resolve the conflict, however it may not be possible to formulate the
resolution in terms of context free grammars.  In order to illustrate that we
present the following examples.

\begin{enumerate}
    \item The classical example from C language is a type-cast 
          syntax.  As a user can define an arbitrary type using
          \verb|typedef| construct, evaluation of the expression
          \verb|(x) + 5| is impossible, unless we know if
          \verb|x| is a type or not.
    \item Assume that we extend C syntax to allow array 
          concatenation using infix binary \verb|++| operator and
          constant-arrays to be written as \verb|[1, 2, 3]|.
          We immediately run into the problem to disambiguate the 
          following expression: \verb|a ++ [1]|, as it could mean
          an application of postfix \verb|++| indexed by \verb|1|
          or it could be an array concatenation of \verb|a| and 
          \verb|[1]|.
%    \item Assuming the language allows any unary function or operator
%          to be applied as infix and postfix, we cannot 
%          disambiguate the following expression:
%\begin{verbatim}
%log (x) - log (y)
%\end{verbatim}
%          Potential interpretations are: 
%          \verb|log (uminus (log (x))) (y)|, which is obviously an 
%          error, or \verb|minus (log (x), log (y))|.
\end{enumerate}

\noindent
Sometimes the context may influence not only the parsing decisions but 
also lexing decisions.  Consider the following examples:
\begin{enumerate}
    \item C++ allows nested templates, which means that one could
          write an expression \verb|template <typename foo, list <int>>|, 
          assuming that the \verb|>>| closes the two groups.  In
          order to do that, the lexer must be aware of this context,
          as in a standard context character sequence \verb|>>| means
          shift right operation.
    \item Assuming that a programmer is allowed to define her own 
          operators, the lexer rules must be changed, in case 
          the name of a new operator extends the existing one.  For
          example, assume one defines an operation \verb|+-|.
          It means that from now on an expression \verb|+-5| should
          be lexed as (\verb|+-|, \verb|5|), rather than (%
          \verb|+|, \verb|-|, \verb|5|).
\end{enumerate}

\noindent
In order to resolve the above ambiguities using table-based parser generator,
we have to make sure that one can annotate the grammar with correct choices for
each shift/reduce or reduce/reduce conflict. This puts a number of requirements
on the syntax of a parser generator and on the finite-state machine execution
engine.  Firstly, one has to introduce contexts without interfering with the
above conflict-resolution.  Secondly, one has to have an interface to the lexer,
in case lexing becomes context-dependent, and all the mechanisms should be aware 
of error-recovery facilities.

Having said that, we can see that using parser generators could be of
the same challenge as writing a parser manually, where all the
ambiguities could be carefully resolved according to the language
specification.  As it turns out, most of the real-world language
front-ends use hand-written recursive descent parsers that specially
treat cases that cause ambiguity.  For example the following languages do:
C/C++/ObjectiveC in GNU GCC~\cite{gcc}, clang in LLVM~\cite{clang}, 
JavaScript in Google V8~\cite{v8}.

The main goal of any preprocessor is to perform a substitution of
one element sequence with another.  The unit of the sequence may 
be different depending on the agreement, however the common
case is to say that the unit is a sequence of characters of the
same class.  The number of classes is normally fixed, however 
character belonging to the class may be static as in C preprocessor,
where, for example, notion of space cannot be changed, or dynamic as
in \TeX, where one could specify that a certain character is a 
delimiter.  Then the substitution itself is a replacement of 
units in a sequence, which are treated as arguments, with the 
assigned arguments.  The key problem here, as we are concerned
in this paper, is a lack of separation between the rewriting itself
and the transformation of a token-sequence.  Consider an example of
a C macro:
\begin{verbatim}
#define foo(x, y) x y
\end{verbatim}
First of all, it is really hard to say anything about the result
of this macro, as \verb|foo (5,6)| expands to \verb|5 6|, but
both \verb|foo (,5)| and \verb|foo (5,)| expands to \verb|5|.
Secondly, as comma is a part of syntax definition of the macro
then one cannot just pass a token sequence \verb|5, 6| as a
first argument of \verb|foo|.  In order to resolve this one
may escape the comma by wrapping it in parentheses and calling
\verb|foo ((5,6), 7)| which will expand to \verb|(5, 6) 7|.
The only way to flattern the list is to perform an application
of another macro.  For example:
\begin{verbatim}
#define first(x, y) x
#define bar(x, y) first x y
\end{verbatim}
So we have higher-order macro here, but it would work only if
arguments have a right type and the application of 
\verb|bar ((5,6), x)| would expand to \verb|5, x|, however
application of \verb|bar (5, 6)| would not provide an error
but would expand to \verb|first 5 6|.  And as a last example
we can make original macro \verb|foo| return 3 arguments
by expanding \verb|foo (5, foo (6, 7))| which will expand to
\verb|5 6 7|.  We may clearly see that making some static 
conclusions by checking a system of macros is impossible, as
it may all depend on the application; and making any dynamic
decisions with respect to the correctness of substitution is
also not possible, as there is no way to declare the criteria
of correctness.

Despite all the correctness complications macro systems are not powerful
enough to introduce new language constructs.  For example, it would be natural
to represent a number's absolute value as \verb/|a|/, or to allow a number of
user-defined literals to introduce units in a programming language like 
\verb/5kg/ or \verb|8 mm|.  Even if a macro-system can do it, resolving nested
expressions treating one and the same symbol differently still might be
confusing.  For example, would it be possible for some macro system to
transform an expression \verb/| a|b |/ into \verb/abs (a|b)/?

The proper way of doing macro-substitutions is to allow an
extension to the grammar.  However, providing a handle for
arbitrary changes of the grammar may lead to uncontrolled 
changes in the semantics of the language, which again would
make the proof of program corectness hard to create.

As a solution to the given problem we introduce a preprocessor
which works on the sequence of pseudo-tokens which is a combination
of tokens recognized by parser and productions of the grammar
used by parser.  To make it even more powerful we allow a 
regular expression on pseudo-tokens, still being able to 
guarantee the correctness of the transformation.

The rest of the paper is organized as follows: we are going to 
introduce a basic model of the parser which we use to build a
preprocessor on in section 2.  Then we describe the syntax of
the preprocessing rules and the way we intend to prove correctness
in section 3.  Section 4 describes the way we deal with a number
of preprocessor definitions and finally we evaluate the work
and conclude in sections 5 and 6.




\section{\label{sec:parser}Parser model}

The parser which is a basis for building a preprocessor on is 
based on recursive descent LL(k) or LL(*) algorithm.  Recursive
descent is a natural human approach to write parsers, and in
case if $k$ is small, the efficiency of parsing is linear with
respect to the number of tokens on the input stream. 

As a running example in this paper we are going to use a
grammar of a language with C-like syntax described in
Fig.~\ref{fig:grammar}.

\begin{figure}[h!]
\centering
\begin{verbatim}
program       ::=  ( function ) * ;
function      ::=  type-id  '(' arg-list ')' stmt-block ;
arg-list      ::=  ( type-id  id ) * ;
stmt-block    ::=  '{' ( expr | return ';' ) * '}' ;
expr          ::=  fun-call | assign | cond-expr ;
fun-call      ::=  id '('expr (',' expr ) * ')' ;
assign        ::=  id '=' expr ;
cond-expr     ::=  bin-expr '?' cond-expr ':' expr ;
bin-expr      ::=  bin-expr binop primary-expr
primary-expr  ::=  number | prefix-op expr | '(' expr ')' ;
binop         ::=  '&&' | '||' | '==' | '!=' ... ;
prefix-op     ::=  '-' | '+' | '!' | '~' ;
\end{verbatim}
\caption{\label{fig:grammar}A grammar of a C-like language.}
\end{figure}

The preprocessor is build as an extension to the parser, so
it expect a certain behaviour from the parser.  

\begin{description}
    \item[Token stream] The parser should conceptually represent
    a stream of tokens as a doubly linked list, which is possible
    to traverse in either direction and perform a substitution of
    a token group with another token group, and restarting a 
    stream from an arbitrary position.  The implementation details
    are left to the creators of the parser.

    \item[Pseudo tokens] The parser normally reduces the grammar rule
    by reading tokens from the input stream.  We introduce a notion
    of pseudo-token, which conceptually is an atomic element of the 
    input stream, but which represents a reduced grammar rule.  The 
    implementation details are left to the parser creator.  The most
    straight forward and inefficient way would be to convert the 
    pseudo-token into the token stream and parse again.
    
    \item[Handle functions] First of all we ask, that every production
    is represented as a function\footnote{Note, that these functions
    have side-effects, so the order of calling is important.} with a
    signature \verb/Parser -> (AST|Error)/, i.e. function gets a
    parser-object as an input and returns either an Abstract Syntax Tree
    node or an error.  We call those functions handle-functions. We
    require that handle-functions structure mimic a formulation of the
    grammar, i.e.  if a production A depends on a production B, we
    require function handle-A to call function handle-B.

    Each handle-function implements error recovery (if needed) and takes
    care about disambiguating productions according to the language
    specification, resolving operation priorities, syntax ambiguities
    and so on.  Each handle function has an access to the parser, which
    keeps has an internal state, which changes when a handle-function is
    applied.  
    %In a some sense an application of a handle-function
    %is a reduce step of a shift-reducer.
    
    \item[Is functions] Each handle-function is paired with a predicate
    function which checks whether a sequence of tokens pointed by a
    parser-state matches a given rule.  This type of functions we will
    call is-functions.  Application of an is-function does not modify
    the state of the parser.  Is-functions may require unbounded
    look-ahead in general case, however we leave the implementation
    decision to the parser creator.  One can always reuse matched AST
    nodes to perform subsequent matches.

    \item[Match function] In the beginning of each handle-function, each
    production calls a function called \verb|match| with a signature
    \verb/(Parser, Production) -> Parser/.  A match-function is an
    interface to the preprocessor which checks if a stream of tokens
    pointed by the parser has a valid substitution in a given
    production; if it does it performs a substitution and makes sure
    that the parser token stream starts with that substitution.  In 
    case \verb|match| simply returns a parser, then the preprocessor\
    would not perform any substitutions.
\end{description}

Assuming that all the requirements are met, the grammar $G = (N, T, P,
S)$ provides a full information required to build a support for
user-defined matches.



\section{\label{sec:dynext}Dynamic extensions}

In this section we are going to describe a syntax of the preprocessing
rules and demonstrate the way to prove a correctness of the
transformation.

\subsection{Match Syntax}

The preprocessing rules are defined using the following syntax:
\begin{verbatim}
match [\prod1] v = regexp   ->   [\prod2] f (v)
\end{verbatim}
This reads as following: if at the beginning of production \verb|prod1|
a stream of pseudo-tokens pointed by parser matches a regular expression
\verb|regexp|, which can be aliased with variable named \verb|v| in the
right hand side of the match, then the matched tokens will be replaced 
with a reduction of \verb|prod2| production applied on a list of 
pseudo-tokens that is being returned by \verb|f (v)|.  Function \verb|f|
is a function which is defined in functional language $T$ and which
is used to perform a preprocessing transformation on the list of 
tokens.  

The \verb|regexp| regular expression is a box standard regular
expression which is defined by the grammar at Fig.~\ref{fig:reggram}.
\begin{figure}[h!]
\begin{verbatim}
regexp          ::= concat-regexp '|' regexp
concat-regexp   ::= asterisk-regexp  concat-regexp
asterisk-regexp ::= unary-regexp '*' | unary-regexp
unary-regexp    ::= pseudo-token | '(' regexp ')'
\end{verbatim}
\caption{\label{fig:reggram}Grammar of the regular expressions on
pseudo-tokens}
\end{figure}
In this paper we are using a minimalistic syntax for regular expression
to demonstrate some basic properties.  Later on, this syntax may be 
easily extended.

Further down in this paper we are going to use an escaped syntax 
for pseudo-tokens which represent grammar production names, like
\verb|\expr|, \verb|function|, etc.  The operators of regular
expression, namely (\verb/|/, \verb|*|, \verb|(|, \verb|)|) will
be escaped as well like: (\verb/\|/, \verb|\*|, \verb|\(|, \verb|\)|).
For the simplicity reasons we would assume that we do not introduce
any conflicts by defining escape symbols.  In case the conflict 
would appear, we may change an escape symbol, or even whole escaping
mechanism, but it does not influence the matter.  In case the conflict 
would appear, we may change an escape symbol, or even whole escaping
mechanism, but it does not influence the matter.

Now we can demonstrate a simple substitution example on the language
defined in Fig.~\ref{fig:grammar}.  Assume that function \verb|reverse|
is defined in $T$ and it reverses any list of pseudo-tokens, and what
we need is to flip the arguments in a function called \verb|foo|.  In
that case the following match would perform such a substitution.
\begin{verbatim}
match [\fun_call] v = \id ( \expr \( , \expr \) \*
   -> [\fun_call] reverse (v)
\end{verbatim}

\subsection{$T$ language and correctness}
In order to perform a transformation of the matched list we define a
minimalistic functional language called $T$ in order to demonstrate
the approach.  The core definition of $T$ is given by Fig.~\ref{fig:t}.

\begin{figure}
\begin{verbatim}
program         ::= ( function ) *
function        ::= id '::' fun_type id var-list '=' expr
fun_type        ::= (type | '(' fun_type ')') '->' fun_type
var_list        ::= id ? | id (',' id ) *
expr            ::= id | expr expr | let_expr | if_expr | builtin
let_rec         ::= 'let' id '=' expr (',' id '=' expr) * 'in' expr
if_expr         ::= 'if' cond_expr 'then' expr 'else' expr
cond_expr       ::= 'type' expr '==' type | expr
builtin         ::= 'cons' '(' expr ',' (expr | 'nil')')' 
                    | 'head' expr | 'tail' expr | 'value' expr
                    | pseudo_token '[' expr ']' | number
                    | + | - | ...
type            ::= pseudotoken_regexp | int | regexp_t
\end{verbatim}
\caption{\label{fig:t}Grammar to define language $T$}
\end{figure}

The main use-case of the language is to traverse over the matched
list of pseudo-tokens applying recursion, head, tail and cons
constructs.  In order to stop the recursion we also introduce
arithmetic operations on integers.  In order to perform partial
evaluation, we need to have an interface to the value of the
pseudo-token.  For that reason we introduce function \verb|value|
which is applicable to the pseudo-tokens which have a constant
integer value (in our example it is a \verb|\number| pseudo-token).
In order to construct an object from integer, we are using
\verb|\number[42]| syntax.  The \verb|value| function operates
on integers only for the simplicity of the model only, the basic
types can be extended in future.

\subsubsection{Type system}
In order to prove a correctness of the match, we are going to use
a specially designed type-system which is based on the regular 
expression being treated as types.  We are going to use a type
inference to check if the function application in the right hand
side of the match is allowed within a given production.  In the
current paper we are not going to give a definition of all the
type deduction rules, however, we are going to do that in the 
full paper.  Further in this section we are going to share the
basic ideas and principles.

The main idea of the type system for $T$ is to treat a regular
expression as a type.  We start with a fact that if the 
right-hand side of the system was called, then the match succeeded
and the result of the match is a flat list of pseudo-tokens
but from the regular expression we have an additional information
about the structure of this list.  In order to perform a type 
inference, we observe that regular languages, hence regular 
expression bring a number of set operations, which is a key
driving force of the inference.  First of all, it is easy to
define a subset relationship on two regular expressions
$r_1 \sqsubseteq r_2$.  As we know, we can always build a DFA for
a regular expression and minimize it, which gives us a minimal
possible automaton for the language recognized by a given regular
expression.  It means that $r_1 \sqsubseteq r_2 \Rightarrow 
min (det (r_1)) \sqsubseteq min (det (r_2))$.  For two minimized 
automatons $A_1$ and $A_2$, $A_1 \sqsubseteq A_2$ means that there
is a mapping $\Psi$ of $A_1$ states to $A_2$ states such that:
\[
    Start (A_1) \to Start (A_2) \in \Psi
\]
\[
    \forall s \in States (A_1) \forall e \in Edges (s),
    \Psi (Transition (s, e)) = Transition (\Psi (s), e)
\]

$States (x)$ denotes a set of all the states of automaton $x$,
$Edges (s)$ is a set of pseudo-tokens which mark the outcoming
edges of state $s$.  Finally, $Transition (s, t)$ denotes a
state which is reachable from $s$, using edge marked with $t$.

The subset relationship is going to be used to create a sub-typing
hierarchy, and we also have a notion of a super-type of
the hierarchy, which is given by a regular expression \verb|.*|,
we are going to denote this type $\top$.  It is obvious to see that
$\forall t_i \in R, t_i \sqsubseteq \top$.

It is possible to construct a type for head and tail application
by following the edges from the starting state of DFA in case of
head, and creating a set of sub-automatons in case of tail.
Furthermore we can infer a type for recursive head/tial traversal
over the list in either direction.  In order to do that, one has
to construct a set of all the sub-automatons of a given one in 
case of forward traversal and the set of all sub-automatons of
the reversed automaton in case of backward.

In order to propagate type information in the branches obtained from the
application of \verb|type| construct, we have to know how to subtract
two regular expressions, which is again simple.  If $T$ and $S$  are
respective DFAs for subtracted types, all we have to do is to construct
$C = T \times S$ and make the final states of $C$ be the pairs where $T$
states are final and $S$ states are not. 

The type inference procedure itself borrows an idea from SaC type
system~\cite{} when we start with an abstract type \verb|regexp_t|
which is $\top$ and recursively precise the type until we get to
a fixed point.









\section{\label{sec:regexps}Regexps}
\fixme{This is a weird draft by Petch}

\subsection{Match as a regular expression}
The left part of the match (without the resulting type) is actually a
regular expression, but with tokens to be matched instead of single
characters. That does not change the concept, because just the same as
we can have a getter for the next symbol in the input stream, we have
the get next token function in the parser. We operate not with an input
stream of symbols, but with an input stream of tokens, generated by the
parser.

Currently our regular expression syntax allows using or \verb/|/ and
asterisk \verb|*| notations. Asterisk is more binding than or, so if you
want to have (a or b) zero to n times you will have to write
\verb/(a|b)*/. The supported syntax can easily be extended, but it is
unnecessary for demonstration purposes. Matching classes of tokens is
unneeded, because the classes can be emulated by using or constructions.
The token count is fairly limited (how many token types do we have?),
unlike the character set that can be used in regular expressions, so
symbol classes like (not a) are not essential. 

The given regular expression is parsed to create an executable
automaton. 

\subsection{Regular expression to NFA}
During parsing each element of the regular expression is depicted with
an object of a specific class. There are three classes with a similar
interface, every class is a matcher object for a construction from the
parsed expression. Token matching is a simple class, whereas asterisk
and or classes are containers for other token matching sequences. The
created objects are then linked in a list. These classes form an
undetermined automata, where a token matching object depicts a transfer
between automata states by a specific token, and all other connections
are actually epsilon transfers (e.g. next element for an object or the
path from asterisk object to it's contents).

\subsection{NFA to DFA}
Next step is to create a DFA out of the created NFA to minimize the
effort needed to execute the given automata. Currently the subset
construction algorithm is used for the determinisation process. (Should
I describe the algorithm? It's pretty standard, although I couldn't find
the name of it). Then the determinate automata created is minimized.
(Should I describe the algorithm, again?)

The created DFA is minimal and therefore the most effective for matching
the given expression. The DFA is represented by a list of objects where
each object contains a map of symbols and objects to which the current
symbol transfers the automaton.

Currently we decided to give up on matching result grouping support in
the regular expressions in favour of maintaining a fully determinate
automaton for each regular expression. Consequently, in the matches'
output, all of the tokens are returned in a single unnested list. (An
option to allow grouping - determinate only parts of the automaton that
are enclosed in braces, then combine the created automata by
consequently glueing the accepting and starting states together. This
approach disallows the combining of several matcher automata into a
single one, as the procedure would ruin the grouping.)

\subsection{Several DFA to single DFA}
Although the prototype doesn't support this feature yet, we intend on
creating a single automaton from the several automata we created. 

For the time being, however, the prototype executes the matches as
follows. We have, for example, $m$ matches, and the simplest of the
decisions on how to match (or not) all of them at one pass through the
source text is to depict them as an NFA with m epsilon branches from the
start state, each of which leads to a DFA we already created. This is a
simple solution and the complexity of it is $O(m)$.

\fixme{Add text and stuff}


In due course we will combine the given m automata into 1 bigger
automata. One by one as the matches are being added to the match list,
the previous and the new automata are combined into a single DFA, so
that it's execution complexity is $O(1)$, a single traverse option for a
single input token.

Some tests should be created to understand whether the combination and
determinisation is time-effective in a general case. It might take more
time than actual execution of the improvised NFA.

\subsection{Optimisation}
\fixme{Write}
\begin{enumerate}
    \item Context matches can be stored for later use, although it is
    unlikely that an exact same match will appear later in the code.
    \item Maybe it is reasonable to create a huge automata for the
    global matches and independent ones for each of the contexts. They
    can be thrown away once function is parsed.
    \item For the big automata - store only a single regexp id in the
    accepting states, no id lists and stuff.
    \item Cache automata while matches come in an uninterrupted
    sequence, once the sequence is disrupted, combine the cached
    automata into a single one and match.
\end{enumerate}


\section{Application}
\subsection{Preprocessing}
\subsection{Templates}
\subsection{Optimisation potential}

\section{Evaluation}
Here is a bunch of links for the existing macro-preprocessors:

\begin{tabular}{l | p{.8\textwidth}}
ML/I & \url{http://www.ml1.org.uk/htmldoc/ml1sig.html} \\
GEMA & \url{http://gema.sourceforge.net/new/docs.shtml} \\
GPP  & \url{http://files.nothingisreal.com/software/gpp/gpp.html}
В этой штуке советую заглянуть в ADVANCED EXAMPLES с лямбдой\\
TRAC &
\url{http://web.archive.org/web/20050205172849/http://tracfoundation.org/t2001tech.htm}
Это очень разумная идея правда совсем дохлая -- там тоже
функциональный язык внутри живет, но работает на строках кажись
\end{tabular}

Еще бывают: m4, cpp, lisp/scheme macros, tex?...
\subsection{Macros in Lisp}
In some way the concept of Lisp is very close to the idea of dynamic parser
described here. Instead of operating with strings, Lisp deals with tokens.
However, the syntax of Lisp is very different from the most programming
languages. This makes macro transformation an efficient tool in Lisp. The main
question here is to figure out are we capable to have a similar functionality
in our dynamic parser regardless syntax differences. \\
A Lisp program is represented as a nested list structure. Each list
represents a form which can be transformed or evaluated. The homoiconicity in
Lisp allows to simplify macro processing significantly. Consider the problem
mentioned in section \ref{sec:intro}: \verb|log (x) - log (y)| -- expression
which could be interpreted differently depending on the context. In Lisp each
case would have a separate representation: \verb|((log (- (log x))) y)| or
\verb|(- (log x) (log y))|. You can see that there is no need for compiler to
guess the context. \\
There is another advantage of such approach. The prefix notation allows to make
a decision depending on the first character in the list. Particularly, there 
are two options:
\begin{enumerate}
    \item Mark expression as a function call, providing  the first character is 
          defined in the function table.
    \item Evaluate the expression given a macro definition. 
\end{enumerate}
A macro expansion consists of two steps as well, which are performed
sequentionally:
\begin{enumerate}
    \item Build an expression specified by macro definition.
    \item Evaluate generated expression.
\end{enumerate}
These steps are distinguished as the first one operates with expressions, and
the second one deals with its values. To perform evaluation an expression-value
table is needed, because when expression transformation can't be performed, the
evaluation process begins. Macro expansion performance capabilities generally
depend on evaluation. To improve evaluation there are several incremental Lisp
compilers. These compilers can evaluate expressions in macro when all necessary
information become available. \\
It should be pointed out that it is possible to build a dynamic parser similar
to the one in Lisp, however some issues are to be handled:
\begin{itemize}
    \item Given a program represented as a sequence of tokens, it is necessary
    to resolve the context unambiguously. In some particular cases it could be
    a tricky task, as a number of look-ahead tokens is unbounded.
    \item There is no `decisive' token in the sequence. We are to use regular
    expressions to parse the whole expression. 
\end{itemize}

\subsection{ML/1 macros}
ML/1 is a stream-based macros processor\cite{mli}. It operates on a sequence of
tokens.  The processor reads tokens one by one and performs input stream
transformation taking into account the rules defined. \\ 
In macro definition we define a correspondence between token sequence from the
input and replace tokens. All atomic tokens are separated by delimiter tokens,
such as `space' or `new line'. Suprisingly, there are no arguments placed in
the macro rule to match. The arguments can appear between any atomic tokens.
For example, 
\begin{verbatim}
MCDEF foo bar baz AS ...
\end{verbatim}
will match \verb|foo xx bar yy baz|, because arguments are inserted between
atomic tokens. In order to restrict such insertion, tokens have to be combined
into another one, atomic token. Thus in most cases the information about exact
number of arguments and their names is not accessible. Therefore, arguments are
accessed by number in the order they are met in the input string. Basically,
this allows to support variable number of arguments. Here arises a problem of
handling these arguments, because we do not know in advance how many arguments
we have and we don't even know their types.  \\
The following features of macro language allow to handle arguments properly:
\begin{enumerate}
    \item Tokens placed between argument tokens are called delimiters. It is
    possible not only to access arguments, but also delimiter tokens,
    enumerating them.
    \item It is possible to define `if' condition statements. `Jumps' or `goto'
    statemens are also supported. Consequently, we can verify delimiter tokens
    number and its type, and perform substitutions accordingly.
    \item Local variables can be used inside macros. This allows to describe
    loop statements for iterating over the arguments.
    \item ML/1 supports nested macro calls. While searching for delimiters and
    arguments we can meet another macro call. In this case, we descend to a
    lower level and return it's delimiters and arguments. Finally, they are
    inserted in the `top' list.
\end{enumerate}
To sum up, ML/1 provides advanced features for macro processing. It is
implemented as an imperative language operating on the stream of tokens.  It
supports conditions, loops, branchs and assignment, so the language is
Turing-complete.


\section{Future work}

\bibliographystyle{plain}
\bibliography{paper}


\end{document}
