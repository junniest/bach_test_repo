%\documentclass[a4paper]{article}
\documentclass[a4paper]{llncs}

\usepackage[T2A]{fontenc}
\usepackage[utf8x]{inputenc}
\usepackage[russian, english]{babel}
\usepackage{color}
\usepackage[colorlinks,bookmarksopen,bookmarksnumbered,citecolor=red,urlcolor=red]{hyperref}
\usepackage{graphicx}

\newcommand{\fixme}[1]{\vskip 5mm\noindent{\bf FIXME}: {\it #1}}

\author{Jūlija Pečerska\inst{1}, Artjoms Šinkarovs\inst{2}, Pavels Zaičenkovs\inst{3}}
\date{\today}
\title{On dynamic extensions of context-dependent parser\\
       --- Extended Abstract ---}
\institute{
  University of Latvia,
  Raiņa bulvāris 19, Rīga,
  Latvija, LV-1586
\and
  Heriot-Watt University,
  Riccarton, Edinburgh,
  EH14 4AS, United Kingdom
\and
  Moscow Institute of Physics and Technology,
  141700, 9, Institutskii per., Dolgoprudny, 
  Moscow Region, Russia
}


\begin{document}
\maketitle

\begin{abstract}
Most of the modern programming languages define a syntax which 
cannot be described precisely using context free grammars.  For
that reason a lot of parsers are implemented by hands, avoiding
parser-generators like bison or antlr.  The idea of using a
preprocessor before the actual compilation is proven to be
useful, however the automatic verification of such a programs
is extremely hard in case preprocessor and the parser act
independently, like for example in C/C++.

In this paper we present a way of building a preprocessor on top
of an existing recursive descent parser.  The preprocessing step
consists of matching a list of tokens against a regular expression
built on tokens and encoded production names of the given grammar
and transforming the result using a functional language.  We 
demonstrate that using the proposed approach it becomes possible
to i) give a static guarantees regarding the preprocessing rules;
ii) safely express non-trivial syntactical constructions; and
iii) perform a restricted partial evaluation.
\end{abstract}

\section{\label{sec:intro}Introduction}

Very often expressiveness of a programming language introduces a number of
ambiguities in its syntax.  The language specification clearly states how to
resolve the conflict, however it may not be possible to formulate the
resolution in terms of context free grammars.  In order to illustrate that we
present the following examples.

\begin{enumerate}
    \item The classical example from C language is a type-cast 
          syntax.  As a user can define an arbitrary type using
          \verb|typedef| construct, evaluation of the expression
          \verb|(x) + 5| is impossible, unless we know if
          \verb|x| is a type or not.
    \item Assume that we extend C syntax to allow array 
          concatenation using infix binary \verb|++| operator and
          constant-arrays to be written as \verb|[1, 2, 3]|.
          We immediately run into the problem to disambiguate the 
          following expression: \verb|a ++ [1]|, as it could mean
          an application of postfix \verb|++| indexed by \verb|1|
          or it could be an array concatenation of \verb|a| and 
          \verb|[1]|.
%    \item Assuming the language allows any unary function or operator
%          to be applied as infix and postfix, we cannot 
%          disambiguate the following expression:
%\begin{verbatim}
%log (x) - log (y)
%\end{verbatim}
%          Potential interpretations are: 
%          \verb|log (uminus (log (x))) (y)|, which is obviously an 
%          error, or \verb|minus (log (x), log (y))|.
\end{enumerate}

\noindent
Sometimes the context may influence not only the parsing decisions but 
also lexing decisions.  Consider the following examples:
\begin{enumerate}
    \item C++ allows nested templates, which means that one could
          write an expression \verb|template <typename foo, list <int>>|, 
          assuming that the \verb|>>| closes the two groups.  In
          order to do that, the lexer must be aware of this context,
          as in a standard context character sequence \verb|>>| means
          shift right operation.
    \item Assuming that a programmer is allowed to define her own 
          operators, the lexer rules must be changed, in case 
          the name of a new operator extends the existing one.  For
          example, assume one defines an operation \verb|+-|.
          It means that from now on an expression \verb|+-5| should
          be lexed as (\verb|+-|, \verb|5|), rather than (%
          \verb|+|, \verb|-|, \verb|5|).
\end{enumerate}

\noindent
In order to resolve the above ambiguities using table-based parser generator,
we have to make sure that one can annotate the grammar with correct choices for
each shift/reduce or reduce/reduce conflict. This puts a number of requirements
on the syntax of a parser generator and on the finite-state machine execution
engine.  Firstly, one has to introduce contexts without interfering with the
above conflict-resolution.  Secondly, one has to have an interface to the lexer,
in case lexing becomes context-dependent, and all the mechanisms should be aware 
of error-recovery facilities.

Having said that, we can see that using parser generators could be of
the same challenge as writing a parser manually, where all the
ambiguities could be carefully resolved according to the language
specification.  As it turns out, most of the real-world language
front-ends use hand-written recursive descent parsers that specially
treat cases that cause ambiguity.  For example the following languages do:
C/C++/ObjectiveC in GNU GCC~\cite{gcc}, clang in LLVM~\cite{clang}, 
JavaScript in Google V8~\cite{v8}.

The main goal of any preprocessor is to perform a substitution of
one element sequence with another.  The unit of the sequence may 
be different depending on the agreement, however the common
case is to say that the unit is a sequence of characters of the
same class.  The number of classes is normally fixed, however 
character belonging to the class may be static as in C preprocessor,
where, for example, notion of space cannot be changed, or dynamic as
in \TeX, where one could specify that a certain character is a 
delimiter.  Then the substitution itself is a replacement of 
units in a sequence, which are treated as arguments, with the 
assigned arguments.  The key problem here, as we are concerned
in this paper, is a lack of separation between the rewriting itself
and the transformation of a token-sequence.  Consider an example of
a C macro:
\begin{verbatim}
#define foo(x, y) x y
\end{verbatim}
First of all, it is really hard to say anything about the result
of this macro, as \verb|foo (5,6)| expands to \verb|5 6|, but
both \verb|foo (,5)| and \verb|foo (5,)| expands to \verb|5|.
Secondly, as comma is a part of syntax definition of the macro
then one cannot just pass a token sequence \verb|5, 6| as a
first argument of \verb|foo|.  In order to resolve this one
may escape the comma by wrapping it in parentheses and calling
\verb|foo ((5,6), 7)| which will expand to \verb|(5, 6) 7|.
The only way to flattern the list is to perform an application
of another macro.  For example:
\begin{verbatim}
#define first(x, y) x
#define bar(x, y) first x y
\end{verbatim}
So we have higher-order macro here, but it would work only if
arguments have a right type and the application of 
\verb|bar ((5,6), x)| would expand to \verb|5, x|, however
application of \verb|bar (5, 6)| would not provide an error
but would expand to \verb|first 5 6|.  And as a last example
we can make original macro \verb|foo| return 3 arguments
by expanding \verb|foo (5, foo (6, 7))| which will expand to
\verb|5 6 7|.  We may clearly see that making some static 
conclusions by checking a system of macros is impossible, as
it may all depend on the application; and making any dynamic
decisions with respect to the correctness of substitution is
also not possible, as there is no way to declare the criteria
of correctness.

Despite all the correctness complications macro systems are not powerful
enough to introduce new language constructs.  For example, it would be natural
to represent a number's absolute value as \verb/|a|/, or to allow a number of
user-defined literals to introduce units in a programming language like 
\verb/5kg/ or \verb|8 mm|.  Even if a macro-system can do it, resolving nested
expressions treating one and the same symbol differently still might be
confusing.  For example, would it be possible for some macro system to
transform an expression \verb/| a|b |/ into \verb/abs (a|b)/?

The proper way of doing macro-substitutions is to allow an
extension to the grammar.  However, providing a handle for
arbitrary changes of the grammar may lead to uncontrolled 
changes in the semantics of the language, which again would
make the proof of program corectness hard to create.

As a solution to the given problem we introduce a preprocessor
which works on the sequence of pseudo-tokens which is a combination
of tokens recognized by parser and productions of the grammar
used by parser.  To make it even more powerful we allow a 
regular expression on pseudo-tokens, still being able to 
guarantee the correctness of the transformation.

The rest of the paper is organized as follows:
\fixme{BLA-bla-bla}





\section{\label{sec:parser}Parser model}

Our work is concerned with a dynamic grammar modification on the
fly, and as a base of our approach we are going to consider an
LL(k) recursive descent parser with a certain properties.

As a running example in this paper we are going to use an imaginary
language with a C-like syntax.  Consider a grammar of the language.

\begin{verbatim}
program       ::=  ( function ) * ;
function      ::=  type-id  '(' arg-list ')' stmt-block ;
arg-list      ::=  ( type-id  id ) * ;
stmt-block    ::=  '{' ( expr | return ';' ) * '}' ;
expr          ::=  fun-call | assign | cond-expr ;
fun-call      ::=  id '(' ( expr ) * ')' ;
assign        ::=  id '=' expr ;
cond-expr     ::=  bin-expr '?' cond-expr ':' expr ;
bin-expr      ::=  bin-expr binop primary-expr
primary-expr  ::=  number | prefix-op expr | '(' expr ')' ;
binop         ::=  '&&' | '||' | '==' | '!=' ... ;
prefix-op     ::=  '-' | '+' | '!' | '~' ;
\end{verbatim}

% These are the requirements.
First of all we ask, that every production is represented as a function
with a signature \verb/Parser -> (AST|Error)/, i.e. function gets a
parser-object on input and returns either an AST node or an error.
We would call those functions handle-functions. We require that 
handle-functions structure mimic a formulation of the grammar,
i.e. if a production A depends on a production B, we require 
function handle-A to call function handle-B.

Each handle-function implements error recovery (if needed) and takes
care about disambiguating productions according to the language
specification, resolving operation priorities, syntax ambiguities
and so on.  Each handle function has an access to the parser, which
keeps has an internal state, which changes when a handle-function
is applied.  In a some sense an application of a handle-function
is a reduce step of a shift-reducer.

Each handle-function is paired with a predicate function which checks
whether a sequence of tokens pointed by a parser-state matches a 
given rule.  This type of functions we will call is-functions.
Application of an is-function does not modify the state of
the parser.  Is-functions may require unbounded look-ahead from 
the parser, which also happens to be a requirement.  We assume that in
order to resolve complicated ambiguities unbounded look-ahead is needed
anyways, as language expressions normally allow unbounded nesting.


Assuming that all the requirements are met, the grammar $G = (N, T, P,
S)$ provides a full information required to build a support for
user-defined matches.



\section{\label{sec:dynext}Transformation system}

In this section we are going to describe a syntax of the rules of 
the transformation system and demonstrate the way to prove a correctness
of the transformation.

\subsection{Match Syntax}

The transformation system consists of the \textit{match} rules and
token-transformation functions.  First of all we would consider the
match rule, which modifies a behaviour of a given production.  The
syntax of the rule can be learned from the following example.
\begin{verbatim}
match [\prod1] v = regexp   ->   [\prod2] f (v)
\end{verbatim}
This reads as follows: if at the beginning of production \verb|prod1|
a stream of pseudo-tokens pointed at by the parser matches a regular expression
\verb|regexp|, which can be aliased with variable named \verb|v| in the
right hand side of the match, then the matched tokens will be replaced 
with a reduction of \verb|prod2| production applied on a list of 
pseudo-tokens that is being returned by \verb|f (v)|.  Function \verb|f|
is a function which is defined in functional language $T$ and which
is used to perform a transformation on the list of tokens matched by
the left hand side of the match.  

The \verb|regexp| regular expression is a box standard regular
expression~\cite{regexp} which is defined by the grammar at 
Fig.~\ref{fig:reggram}.
\begin{figure}[h!]
\begin{verbatim}
regexp          ::= concat-regexp '|' regexp
concat-regexp   ::= asterisk-regexp  concat-regexp
asterisk-regexp ::= unary-regexp '*' | unary-regexp
unary-regexp    ::= pseudo-token | '(' regexp ')'
\end{verbatim}
\caption{\label{fig:reggram}Grammar of the regular expressions on
pseudo-tokens}
\end{figure}
In this paper we are using a minimalistic syntax for regular expression
to demonstrate some basic properties.  Later on, this syntax may be 
easily extended.

Further down in this paper we are going to use an escaped syntax 
for pseudo-tokens which represent grammar production names, like
\verb|\expr|, \verb|\function|, etc.  The operators of regular
expression, namely (\verb/|/, \verb|*|, \verb|(|, \verb|)|) will
be escaped as well like: (\verb/\|/, \verb|\*|, \verb|\(|, \verb|\)|).
For simplicity reasons we assume that we do not introduce
any conflicts by defining escape symbols.  In case a conflict 
appears, we may change the escape symbol, or even the whole escaping
mechanism, but it does not influence the matter.

Now we can demonstrate a simple substitution example on the language
defined in Fig.~\ref{fig:grammar}.  Assume that function \verb|replace|
is defined in $T$ with three arguments and it replaces in any list 
of pseudo-tokens occurrence each of the second argument with the third, 
and what we need is to call a function called \verb|bar| with an 
argument being a summed-up arguments of function called \verb|foo|.
In that case the following match would perform such a substitution.
\begin{verbatim}
match [\fun_call] v = foo ( \expr \( , \expr \) \* )
   -> [\fun_call] cons bar (replace (tail v) \, \+)
\end{verbatim}

\subsection{Definition of matches}
Match rules can be defined in the arbitrary places of a program
and the rule activates immediately after the definition was parsed.
We do however differentiate between the global matches and context
matches.  In our case the context is created by a \verb|stmt_block|
production.  In that case, all the matches declared within the 
\verb|stmt_block| production are valid only within this particular
production.  When the production is finished, the matches declared
within the production would be removed.  Declaration of the context
is up to the parser, it may define it in any which way by calling 
two interface functions.  The context definition can be omitted in
which case all the matches would be global.

Both global and context matches depend on the order of definition
and in both cases the first match has a stronger priority than the
subsequent in case the token stream can be matched with more than
one regular expression.  The priority helps to cover the following
case:
\begin{verbatim}
match [\expr] v = f ( \num )            ...
mathc [\expr] v = f ( \primary_expr )   ...
match [\expr] v = f ( \expr )           ...
\end{verbatim}
Here we can see that the last regular expression includes the 
previous as \verb|\primary_expr| is also an \verb|\expr| and
so on.  But introducing priorities, one may still have a different
behaviour in each case.

Nested context matches overload the outer matches, in a same
way as local variables have a first match, in case a variable
of the same name exists in the outer context.

%The match definition is rejected with an error in case the system
%can prove an infinite loop there.  It may happen in case of left
%recursion in the left-hand side of the match and in case when 
%the left-hand side production and left-hand side regular expression
%of the match 

\subsection{$T$ language and correctness}
In order to perform a transformation of the matched list we define a
minimalistic functional language called $T$ in order to demonstrate
the approach.  The core definition of $T$ is given by Fig.~\ref{fig:t}.

\begin{figure}
\begin{verbatim}
program         ::= ( function ) *
function        ::= id '::' fun_type id id * '=' expr
fun_type        ::= (type | '(' fun_type ')') '->' fun_type
expr            ::= id | expr expr | let_expr | if_expr | builtin
let_rec         ::= 'let' id '=' expr (',' id '=' expr) * 'in' expr
if_expr         ::= 'if' cond_expr 'then' expr 'else' expr
cond_expr       ::= 'type' expr '==' type | expr
builtin         ::= 'cons' expr  (expr | 'nil') 
                    | 'head' expr | 'tail' expr | 'value' expr
                    | pseudo_token '[' expr ']' | number
                    | + | - | ...
type            ::= pseudotoken_regexp | int | regexp_t
\end{verbatim}
\caption{\label{fig:t}Grammar to define language $T$}
\end{figure}

The main use-case of the language is to traverse over the matched
list of pseudo-tokens applying recursion, head, tail and cons
constructs.  In order to stop the recursion we also introduce
arithmetic operations on integers.  In order to perform partial
evaluation, we need to have an interface to the value of the
pseudo-token.  For that reason we introduce function \verb|value|
which is applicable to the pseudo-tokens which have a constant
integer value (in our example it is a \verb|\number| pseudo-token).
In order to construct an object from integer, we are using
\verb|\number[42]| syntax.  The \verb|value| function operates
on integers only for the simplicity of the model only, the basic
types can be extended in future.

\subsubsection{Type system}
In order to prove the correctness of the match, we are going to use
a specially designed type-system which is based on the regular 
expressions being treated as types.  We are going to use a type
inference to check if the function application in the right hand
side of the match is allowed within a given production.  In the
current paper we are not going to give a definition of all the
type deduction rules, however, we are going to do that in the 
full paper.  Further in this section we are going to share the
basic ideas and principles.

The main idea of the type system for $T$ is to treat a regular
expression as a type.  We start with a fact that if the 
right-hand side of the system was called and the match succeeded
then the result of the match is a flat list of pseudo-tokens.
However from the regular expression we have additional information
about the structure of this list.  In order to perform a type 
inference, we observe that regular languages, hence regular 
expression bring a number of set operations, which is the key
driving force of the inference.  First of all, it is easy to
define a subset relationship on two regular expressions
$r_1 \sqsubseteq r_2$.  As we know, we can always build a DFA for
a regular expression and minimize it~\cite{dragon-book}, which gives us a minimal
possible automaton for the language recognized by a given regular
expression.  It means that $r_1 \sqsubseteq r_2 \Rightarrow 
min (det (r_1)) \sqsubseteq min (det (r_2))$.  For two minimized 
automata $A_1$ and $A_2$, $A_1 \sqsubseteq A_2$ means that there
is a mapping $\Psi$ of $A_1$ states to $A_2$ states such that:
\[
    Start (A_1) \to Start (A_2) \in \Psi
\]
\[
    \forall s \in States (A_1) \forall e \in Edges (s),
    \Psi (Transition (s, e)) = Transition (\Psi (s), e)
\]

$States (x)$ denotes a set of all the states of automaton $x$,
$Edges (s)$ is a set of pseudo-tokens which mark the outcoming
edges of state $s$.  Finally, $Transition (s, t)$ denotes a
state which is reachable from $s$, using edge marked with $t$.

The subset relationship is going to be used to create a sub-typing
hierarchy, and we also have a notion of a super-type of
the hierarchy, which is represented by a regular expression \verb|.*|,
we are going to denote this type $\top$.  It is obvious to see that
$\forall t_i \in R, t_i \sqsubseteq \top$.

It is possible to construct a type for head and tail application
by following the edges from the starting state of DFA in case of
head, and creating a set of sub-automata in case of tail.
Furthermore we can infer a type for recursive head/tail traversal
over the list in either direction.  In order to do that, one has
to construct a set of all the sub-automata of a given one in 
case of forward traversal and the set of all sub-automata of
the reversed automaton in case of backward.

In order to propagate type information in the branches obtained from the
application of \verb|type| construct, we have to know how to subtract
two regular expressions~\cite{reglang}, which is, again, simple.  
If $T$ and $S$  are
respective DFA for subtracted types, all we have to do is to construct
$C = T \times S$ and make the final states of $C$ be the pairs where $T$
states are final and $S$ states are not. 

The type inference procedure itself borrows the idea from inferring
the shape of the array in SaC type system~\cite{sac2c} i.e. we will
start with an abstract type \verb|regexp_t|
which is $\top$ and recursively precise the type until we get to
a fixed point.









\section{\label{sec:regexps}Regular expressions}

We have devised a method of implementing match syntax that will be time
effective during pre-processing. We have also created a prototype that
ilustrates some of the features described in this article. This section will,
in short, describe the approach we selected, for the full description, however,
turn to the full paper.

As it was said, match syntax is a simple regular expression. This expression is
parsed and transformed into a non-deterministic finite automaton (NFA). Next
step is to create a deterministic finite automaton (DFA) out of the created NFA
to minimize the effort needed to execute the match. This task is performed
using the subset construction algorithm.

In order to minimise execution time even further, we minimise the created DFA.
The minimisation algorithm is described in detail in the "Dragon
Book"\cite{dragon-book}. The algorithm creates sets of states that cannot be
distinguished by any input token sequence. Once the algorithm fails to break
the sets into smaller ones it stops. These sets of states then become new
states of the minimal automaton.

It is essential to note that the existence of such minimal automaton is
provable, despite the complexity of the regular expression it describes. This
statement implies that we have a possibility to prove equality of two automata.
As the naming of the states is unimportant, we will say that two automata are
the same up to state names if one can be transformed to another be simply
renaming the states. Therefore two regular expressions match the same input if
and only if their automata are the same up to state names.

The construction of the DFA does not support back-referencing, so the bracket
groups are intended only to change the priority of the operations in the
regular expressions. We abandon the back-referencing feature in favor of
maintaining a fully determinate automaton for each regular expression.

\subsection{DFA grouping}
As we operate with source code, we want to check all of the matches that we
have in one pass through the code. We consider two possibilities to merge the
created automata into a match system in order to improve effieciency. For both
of them we evaluate DFA adding, matching and context inheritance algorithmic
difficulty in the main article.

Context inheritance difficulty is important, as we can have several included
contexts, where the matches introduced inside the included one have a higher
priority. Imagining that we have a system of \verb/n/ automata we have two
options of doing so. The first one is that on entering a context we add the
\verb/m/ matches we come across to the main match set and remove the \verb/m/
matches upon exiting the context. The second one is that on entering a context
we create a copy of the parent context, to which we add the \verb/m/ newly
found matches and upon exiting the context throw out the entire new system of
matches.

First of the options we consider is joining the automata in a list. Let us
assume that there are \verb/n/ matches to be checked. Then the automata list
represents an NFA with \verb/n/ epsilon branches from the start state, each of
which leads to one of the DFA we already created. This case is fairly simple
and will not be described here extensively.

The second option is more complex. In this case we combine the automata created
for each of the matches together into a single DFA. This is done in order to
reduce matching time, the automata megring algorithm is described in the full
article.  The second option is more interesting, as it allows matching of
\verb/n/ independent patterns in $O(l)$ steps, where \verb/l/ is the amount of
tokens to be matched.

We do not explicitly select a method of operation with joining DFA in the
article, we only give the complexities of the proposed variants. This is
because the optimal solution selection should be based on the practical uses of
the system. It is impossible to state, without any actual examples, what will
be more time-effective during execution. Even though the second option is
time-consuming when adding and removing matches, the dramatic improvement in
execution time might come from the fact that match count is relatively small,
but the automaton execution will, at worst cases, be performed for every token
in the source text.



\section{Application}
\subsection{Preprocessing}
\subsection{Templates}
\subsection{Optimisation potential}

\section{Evaluation}
Here is a bunch of links for the existing macro-preprocessors:

\begin{tabular}{l | p{.8\textwidth}}
ML/I & \url{http://www.ml1.org.uk/htmldoc/ml1sig.html} \\
GEMA & \url{http://gema.sourceforge.net/new/docs.shtml} \\
GPP  & \url{http://files.nothingisreal.com/software/gpp/gpp.html}
В этой штуке советую заглянуть в ADVANCED EXAMPLES с лямбдой\\
TRAC &
\url{http://web.archive.org/web/20050205172849/http://tracfoundation.org/t2001tech.htm}
Это очень разумная идея правда совсем дохлая -- там тоже
функциональный язык внутри живет, но работает на строках кажись
\end{tabular}

Еще бывают: m4, cpp, lisp/scheme macros, tex?...
\subsection{Macros in Lisp}
Due to the Lisp's fully parenthesized Polish prefix syntax notation there is a
powerful macro engine. We are going to cover its main principles and
advantages.
\begin{enumerate}
    \item There is no need to mark out a structure from a token sequence for
    the internal representation (IR). Every sequence of tokens in parenthesis
    shapes an expression called \emph{form}. A program on Lisp represents a tree
    of nested forms. This property is called \emph{homoiconicity}, as the
    source code of a program can be proceeded by means of the same language. In
    this case it is possible to consider a macro-definition as a left tree
    substitution for one in the right part. For example, 
    \begin{verbatim}
    (defmacro sum (x y z) 
    (list '+ (list '* x y) (list '* z z))
    \end{verbatim} 
    macros defines a list substitution for the nested list structure.
    \item Any valid expression in Lisp represents a form. Even the whole
    program is a form.  Therefore, it is possible to replace huge parts of a
    program and even the whole program.  
    \item Lisp's compiler does not match the whole expression (or a form) in
    order to find out either a corresponding macros is defined. Due to the
    prefix notation a `keyword' is the first token in a form.  If a macro
    definition is occured in the macro table under the same keyword, then we
    have to perform a macro substitution.  \\
    However, this simplicity has a disadvantage too. Macros overloading is
    not allowed in Lisp.  
    \item A macro processor in Lisp not only substitutes
    expressions, but also evaluates them. There is an expression-value table in
    Lisp, where each expression is associated with it's value. For example, if
    we write \verb|(setq a 3)|, then an expression \verb|a| has a corresponding
    value \verb|3|. Consequently, if we create a list \verb|(list a (+ a 1))|,
    a list \verb|(3 4)| will be returned, as this expression will be evaluated
    during compilation. This emphasizes an interpetive nature of the language.
    Although expression evaluation is not covered in the Common Lisp standart,
    many compilers, such as GNU CLisp, CMU Common List support this feature.
\end{enumerate}

\section{ML/I macros}
ML/I macros processor\cite{mli} is going to be covered in this section. 
Basically our 
dynamic parser described here and ML/I share the idea of supporting the 
regular expression functionality. This could be a very powerful
instrument to support almost any expression substitution. However, there is a
bunch of problems. Let's have a close look on ML/I macros processor and see how
the problems are handled there. \\
ML/I can be regarded as a processor which operates with strings. In many cases
it behaves like a macro-processor in C. If we define macro 
\verb|#define foo bar| in C, it will substitute \verb|xfoo foo yfooz foox| 
string for \verb|xfoo bar yfooz foox|. It follows that delimeter characters as
whitespaces and `new line' are
taken into account. In ML/I the same macro will look like
\begin{verbatim}
MCDEF foo AS bar
\end{verbatim}
This leads to the fact that even in the simplest case characters can't be
interpreted in the same way. \\
ML/I would be a poor macro-processor if it didn't support any argument passing.
The argument passing is tightly connected with delimiter characters in ML/I.
Arguments can be occured in any place between non-delimiter characters. An
equivalent expression for
\begin{verbatim}
#define unstack x \
{		  \
  x = stack[ptr]; \
  ptr = ptr - 1;  \
}
\end{verbatim}
will be\cite{mli-guide}
\begin{verbatim}
MCDEF UNSTACK
AS <%A1. = STACK[PTR];
PTR = PTR - 1;>
\end{verbatim}
Access to arguments is done by writing \verb|\%An.| where n is the number of the
argument passed. We access to the arguments in the dynamic parser in the same
way, by enumerating arguments. Problems arise when we would like to support
variable number of arguments and more complex match expressions.
...Some text here...


\section{Future work}

\bibliographystyle{plain}
\bibliography{paper}


\end{document}
