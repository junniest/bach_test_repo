\subsection{Templates}
Another example of macro-processing are templates, which are widely used in
C++. In contrast with macros, which performs a string substitution on
preprocessing stage, templates are a language that is executed during
compilation and integrated into the type system of C++. \\
Templates are used to operate with generic types. It allows to avoid code
repetition for every specific type. We define a function using some abstract
types and then abstract types are replaced with actual ones in function call.
For every type a compiler generates a function, repeating the entire code for
every type. \\ 
Templates can be used in terms of metaprogramming, as it allows to perform
evaluations at compile-time, instead of computing at runtime. It provides
sufficient features to perform all kind of evaluations. It can be proved that
C++ templates are turing complete\cite{veldhuizen}. The matter is that
templates support function specialisation and nested template calls. This
allows to describe recursive calls which will be unrolled in compile-time. Code
optimizations, such as constant folding or loop unrolling, are able to fold
expressions produced by templates in order to reduce amount of runtime
computations.
