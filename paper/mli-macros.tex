\subsection{ML/1 macros}
ML/1 is a stream-based macros processor\cite{mli}. It operates on a sequence of
tokens.  The processor reads tokens one by one and performs input stream
transformation taking into account the rules defined. \\ 
In macro definition we define a correspondence between token sequence from the
input and replace tokens. All atomic tokens are separated by delimiter tokens,
such as `space' or `new line'. Suprisingly, there are no arguments placed in
the macro rule to match. The arguments can appear between any atomic tokens.
For example, 
\begin{verbatim}
MCDEF foo bar baz AS ...
\end{verbatim}
will match \verb|foo xx bar yy baz|, because arguments are inserted between
atomic tokens. In order to restrict such insertion, tokens have to be combined
into another one, atomic token. Thus in most cases the information about exact
number of arguments and their names is not accessible. Therefore, arguments are
accessed by number in the order they are met in the input string. Basically,
this allows to support variable number of arguments. Here arises a problem of
handling these arguments, because we do not know in advance how many arguments
we have and we don't even know their types.  \\
The following features of macro language allow to handle arguments properly:
\begin{enumerate}
    \item Tokens placed between argument tokens are called delimiters. It is
    possible not only to access arguments, but also delimiter tokens,
    enumerating them.
    \item It is possible to define `if' condition statements. `Jumps' or `goto'
    statemens are also supported. Consequently, we can verify delimiter tokens
    number and its type, and perform substitutions accordingly.
    \item Local variables can be used inside macros. This allows to describe
    loop statements for iterating over the arguments.
    \item ML/1 supports nested macro calls. While searching for delimiters and
    arguments we can meet another macro call. In this case, we descend to a
    lower level and return it's delimiters and arguments. Finally, they are
    inserted in the `top' list.
\end{enumerate}
To sum up, ML/1 provides advanced features for macro processing. It is
implemented as an imperative language operating on the stream of tokens.  It
supports conditions, loops, branchs and assignment, so the language is
Turing-complete.
