\section{ML/I macros}
ML/I macros processor\cite{mli} is going to be covered in this section. 
Basically our 
dynamic parser described here and ML/I share the idea of supporting the 
regular expression functionality. This could be a very powerful
instrument to support almost any expression substitution. However, there is a
bunch of problems. Let's have a close look on ML/I macros processor and see how
the problems are handled there. \\
ML/I can be regarded as a processor which operates with strings. In many cases
it behaves like a macro-processor in C. If we define macro 
\verb|#define foo bar| in C, it will substitute \verb|xfoo foo yfooz foox| 
string for \verb|xfoo bar yfooz foox|. It follows that delimeter characters as
whitespaces and `new line' are
taken into account. In ML/I the same macro will look like
\begin{verbatim}
MCDEF foo AS bar
\end{verbatim}
This leads to the fact that even in the simplest case characters can't be
interpreted in the same way. \\
ML/I would be a poor macro-processor if it didn't support any argument passing.
The argument passing is tightly connected with delimiter characters in ML/I.
Arguments can be occured in any place between non-delimiter characters. An
equivalent expression for
\begin{verbatim}
#define unstack x \
{		  \
  x = stack[ptr]; \
  ptr = ptr - 1;  \
}
\end{verbatim}
will be\cite{mli-guide}
\begin{verbatim}
MCDEF UNSTACK
AS <%A1. = STACK[PTR];
PTR = PTR - 1;>
\end{verbatim}
Access to arguments is done by writing \verb|\%An.| where n is the number of the
argument passed. We access to the arguments in the dynamic parser in the same
way, by enumerating arguments. Problems arise when we would like to support
variable number of arguments and more complex match expressions.
...Some text here...
