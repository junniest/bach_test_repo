\section{Evaluation}

In this section we are going to consider the common cases for
application of the designed approach and evaluate the results.

\subsection{Syntax extension}

One of the typical examples of using the transformation system is to 
extend the
syntax of the language.  Assume that we would like to have a mathematical 
notation for absolute value, such as $\left|5\right| \equiv abs(5)$. 
It can be achieved using the following match:
\begin{verbatim}
match [\expr] v = | \expr | 
   -> [\expr] cons (\id[abs]) cons \( cons (head (head v)) cons \) cons nil
\end{verbatim}

Keep in mind that we could also define the same match on numbers
and in that case we may use the partial evaluation capabilities of 
the system.  Consider the following example:
\begin{verbatim}
match [\expr] v = | \num |  -> [\num] mabs (head (head v))
mabs :: \num -> \num
mabs x =  if value (x) > 0 then
             \num[(value x)]
          else
             \num[0 - (value x)]
\end{verbatim}

The one thing that is left outside the scope of this paper and that 
is directly connected with this example -- do we allow expressions such as
\verb/| a|b |/.  Currently it is going to be considered as an error, as there is
no way to express the fact that one should use the original pseudo-token
for the expression inside the absolute value,
without applying matches.  It depends on the agreement, of course,
but currently one can overload a special case for 
\verb/| \expr \| \expr \* |/ to allow the latter.  However, in the
current model the following would work just fine: \verb/||A| + |B||/, 
as well as this: \verb/|(a|b)|/.

\subsection{C++ Templates}

The described transformation system may be used in a similar way as template
meta-programming technique.  In the case of templates, instantiation process
may be delayed until some compiler optimisation finds out that certain template
parameter is a constant.  In our setting the instantiation happens during the
parsing phase and if the parser is unable to infer the value for a match
parameter the instantiation will simply not happen.  We are going to consider a
way to express the factorial function in match syntax that will be evaluated in
compile time.

\begin{verbatim}
\match [\number] v = \number ! -> [\number] fact-match (v)
\match   [\expr] v = \expr   ! ->   [\expr] fact-match-expr (v)

fact-match :: \number -> \number
fact-match x = if (value x) == 0 then
                   \number[1]
               else
                   \number[(value x) 
                            * (value (fact-match 
                                      \number[(value x) - 1]))]

fact-match-expr :: \expr -> regexp_t
fact-match-expr x = cons \id[factorial] cons \( cons x cons \) nil
\end{verbatim}

This example demonstrates an advantage over the C++ templates, which
is the shared syntactical form of compile-time and run-time match instances.
It means that if in C++ the factorial function is defined as a template:
\begin{verbatim}
template <int n> struct fact 
{
  static const int value = n * fact<n - 1>::value;
};
 
template <>
struct fact<0> 
{
  static const int value = 1;
};
\end{verbatim}
, there is no way to call it with a non-static variable.  The only way 
would be to create a wrapper function/macro, but in order to find out
if something is a constant at compile time one has to have mechanisms
similar to \verb|__builtin_constant_p| introduced in GCC v4.7.

\subsection{Preprocessing}
We describe a method of building a transformation system, which operates 
in terms of a
single production only.  C preprocessor, on the other hand, works out of scope.
For example, we can use \verb|#if ... #endif| directives wherever in the
source code, for example, starting in the middle of one production and
finishing in the middle of another, which makes static verification difficult.
Using the transformation system defined in this paper it is possible to 
create a match
that would emulate C preprocessor macro-if.  It might look as follows:
\begin{verbatim}
match [\p1] v = # if expr  \. \* # endif  -> [\p2] ...
\end{verbatim}
The dot syntax is currently not allowed, but it is possible to emulate it using
a disjunction over all of the pseudo-tokens.  Now, depending on what is written
in the right hand side of the match, it may be accepted by the type system or
not.  In case the type-system can prove that the result of the right-hand side
expression could be recognized by \verb|\p2|, it is allowed.  However, it is
easy to understand that there is always a way to make the type-system happy,
i.e. by generating a constant expression.  It means that the macro-if can be
emulated.  We will leave the discussion open on the question whether it is good
or bad.  In order to avoid this situation one may declare stricter
type-checking rules.
