\section{Application}
\subsection{Syntax extension}
With the help of macro extension it is possible to extend language's syntax.
Assume, we would like to have a mathematical notation for absolute value
\verb/|5|/. Our macros will look like:
\begin{verbatim}
\match [\expr] v = '|' \expr '|' -> [\num] abs-match (tail (init (v)))

abs-match :: \expr -> \expr
abs-match x = abs(x)
\end{verbatim}
Here we define an additional production to parse \verb|expr| rule. A matcher
will return a list with matched tokens, such as 
\verb/('|', ..., '|')/ from which we need to remove the
first and the last tokens. Then we substitute matched expression with a
function call.
\subsection{Templates}
The macroprocessor can be used as templates in C++. Templates are used to
evaluate expressions at compile-time. To illustrate this, let's consider
factorial computation example.
\begin{verbatim}
\match [\number] v = \number '!' -> [\number] fact-match (v)
\match [\expr] v = \expr '!' -> [\expr] fact-match-expr (v)

match-fun fact-match :: \number -> \number
fact-match x = 1 if value (x)
      | value (x) * fact-match (\number[value (x) - 1]) otherwise

match-fun fact-match-expr :: \expr -> \expr
fact-match-expr x = factorial (x)
\end{verbatim}
We extend here \verb|number| and \verb|expr| productions with mathematical
notation for factorial using exclamation sign. In case of \verb|expr| we simply
call \verb|factorial| function. In addition, we want to compute expression at
compiler-time if we evaluate a constant expression. Similarly to templates, we
can unroll macro-calls and evaluate constant expressions. Thus if we write
\verb|3!|, we will get \verb|6| during compilation phase.
\subsection{Preprocessing}
The macroprocessor deals with language context. However, C preprocessor is
working out of context. For example, we can write any expression inside
\verb|#if ... #endif| directives. In our macroprocessor we can allow to work
out of parser's scope. However, in this case we have to deal with token
conflicts between lexer and parser. In this way, if we define `out of scope'
macro \verb|< ... >|, then binary operations \verb|<| and \verb|>| will be
interpreted as parts of this directive.  
