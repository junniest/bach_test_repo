\section{\label{sec:parser}Parser model}

The parser which serves as a basis for building a preprocessor is 
based on recursive descent LL(k) or LL(*) algorithm.  Recursive
descent is a natural human approach to writing parsers, and in
case if $k$ is small, the efficiency of parsing is linear with
respect to the number of tokens on the input stream. 

As a running example in this paper we are going to use a
grammar of a simple language with C-like syntax described in
Fig.~\ref{fig:grammar}.

\begin{figure}[h!]
\centering
\begin{verbatim}
program       ::=  ( function ) * ;
function      ::=  type_id  '(' arg_list ')' stmt_block ;
arg_list      ::=  ( type_id  id ) * ;
stmt_block    ::=  '{' ( expr | return ';' ) * '}' ;
expr          ::=  fun_call | assign | cond_expr ;
fun_call      ::=  id '('expr (',' expr ) * ')' ;
assign        ::=  id '=' expr ;
cond_expr     ::=  bin_expr '?' cond_expr ':' expr ;
bin_expr      ::=  bin_expr binop primary_expr
primary_expr  ::=  number | prefix_op expr | '(' expr ')' ;
binop         ::=  '&&' | '||' | '==' | '!=' ... ;
prefix_op     ::=  '-' | '+' | '!' | '~' ;
\end{verbatim}
\caption{\label{fig:grammar}A grammar of a C-like language.}
\end{figure}

\noindent
As the preprocessor is build as an extension to the parser, 
it expects a certain behaviour of the parser.  Further down
we list a set of properties we require to be present in the 
implementation of the parser.

\begin{description}
    \item[Token stream] The parser should conceptually represent
    a stream of tokens as a doubly linked list, which allows
    traversing in either direction, performing a substitution of
    a token group with another token group, and restarting a 
    stream from an arbitrary position.  The implementation details
    are left to the creators of the parser.

    \item[Pseudo-tokens] The parser normally reduces the grammar rule
    by reading tokens from the input stream.  We introduce a notion
    of pseudo-token, which conceptually is an atomic element of the 
    input stream, but that represents a reduced grammar rule.  The 
    implementation details are left to the parser creator.  The most
    straight forward and inefficient way would be to convert the 
    pseudo-token back into the token stream and parse again.
    
    \item[Handle-functions] First of all, we ask that every production
    is represented as a function\footnote{Note that these functions
    have side-effects, so the order of calling is important.} with a
    signature \verb/Parser -> (AST|Error)/, i.e. function gets a
    parser-object as an input and returns either an Abstract Syntax Tree
    node or an error.  We call those functions handle-functions. We
    require that handle-function structure mimics the formulation of the
    grammar, i.e. if a production A depends on a production B, we
    require function handle-A to call function handle-B.

    Each handle-function implements error recovery (if needed) and takes
    care of disambiguating productions according to the language
    specification, resolving operation priorities, syntax ambiguities
    and so on.  Each handle function has access to the parser, which
    keeps record of an internal state, which changes when a 
    handle-function is applied.  
    %In a some sense an application of a handle-function
    %is a reduce step of a shift-reducer.
    
    \item[Is-functions] Each handle-function is paired with a predicate
    function which checks whether a sequence of tokens pointed at by a
    parser state matches a given rule.  We will call this type of
    functions is-functions.  Application of an is-function does not modify
    the state of the parser.  Is-functions may require unbounded
    look-ahead in general case, however we leave the implementation
    decision to the parser creator.  One can always reuse matched AST
    nodes to perform subsequent matches.

    \item[Match-function] In the beginning of each handle-function, each
    production calls a function called \verb|match| with a signature
    \verb/(Parser, Production) -> Parser/.  A match-function is an
    interface to the preprocessor that checks if a stream of tokens
    pointed at by the parser has a valid substitution in the given
    production; if it does, it performs the substitution and makes sure
    that the parser points to the beginning of the substitition in the
    token stream.  In case if no matches were found, the preprocessor
    does not perform any substitutions and returns the parser in its
    original state.
\end{description}

Assuming that all the described requirements are met, the grammar 
$G = (N, T, P, S)$ provides complete information required to create 
support for user-defined matches.

