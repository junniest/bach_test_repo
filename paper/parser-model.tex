\section{\label{sec:parser}Parser model}

Our work is concerned with a dynamic grammar modification on the
fly, and as a base of our approach we are going to consider an
LL(k) recursive descent parser with a certain properties.

As a running example in this paper we are going to use an imaginary
language with a C-like syntax.  Consider a grammar of the language.

\begin{verbatim}
program       ::=  ( function ) * ;
function      ::=  type-id  '(' arg-list ')' stmt-block ;
arg-list      ::=  ( type-id  id ) * ;
stmt-block    ::=  '{' ( expr ';' ) * '}' ;
expr          ::=  fun-call | assign | return | cond-expr ;
fun-call      ::=  id '(' ( expr ) * ')' ;
assign        ::=  id '=' expr ;
cond-expr     ::=  bin-expr '?' cond-expr ':' expr ;
bin-expr      ::=  bin-expr binop primary-expr
primary-expr  ::=  number | prefix-op expr | '(' expr ')' ;
binop         ::=  '&&' | '||' | '==' | '!=' ... ;
prefix-op     ::=  '-' | '+' | '!' | '~' ;
\end{verbatim}

% These are the requirements.
First of all we ask, that every production is represented as a function
with a signature \verb/Parser -> (AST|Error)/, i.e function gets a
parser-object on input and returns either an AST node or an error.
We would call those functions handle-functions. We require that 
handle-functions structure mimic a formulation of the grammar,
i.e. if a production A depends on a production B, we require 
function handle-A to call function handle-B.

Each handle-function implements error recovery (if needed) and takes
care about disambiguating productions according to the language
specification, resolving operation priorities, syntax ambiguities
and so on.  Each handle function has an access to the parser, which
keeps has an internal state, which changes when a handle-function
is applied.  In a some sense an application of a handle-function
is a reduce step of a shift-reducer.

Each handle-function is paired with a predicate function which checks
whether a sequence of tokens pointed by a parser-state matches a 
given rule.  This type of functions we will call is-functions.
Application of an is-function does not modify the state of
the parser.  Is-functions may require unbounded look-ahead from 
the parser, which also happens to be a requirement.  We assume that in
order to resolve complicated ambiguities unbounded look-ahead is needed
anyways, as language expressions normally allow unbounded nesting.


Assuming that all the requirements are met, the grammar $G = (N, T, P,
S)$ provides a full information required to build a support for
user-defined matches.

