\documentclass{article}

\author{June Pecherskaya \and Artem Shinkarov}
\date{\today}
\title{Automaton minimization}

\begin{document}

\maketitle

% Here is a very useful link:
% http://www.cs.uky.edu/~lewis/essays/compilers/min-fa.html

A set of words that are recognized by a certain regular expression are
also recognized by a group of automata. These automata might have
different state names and even different number of states. It is
natural that in a parser we would want to use the automaton with the
smallest possible amount of states to minimize the storage required
for it.

As the naming of the states is unimportant, we will say that two
automata are the same up to state names if one can be transformed to
another be simply renaming the states.It turns out that for each
regular language exists a unique (up to state names) automaton with a
minimal number of states\footnote{Do I have to give the proof that the
minimal automaton exists?}.

Before being able to continue to the actual minimization of automata,
a definitions is required. We will say that string x distinguishes
state $s$ from state $t$ if only one state reached from $t$ and $s$ by
following x is an accepting state. So two states are distinguishable
if there exists a string that distiguishes them. Any accepting state
is distinguishable from any nonaccepting state by an empty string (a
state cannot be accepting and nonaccepting at the same time).

The minimization algorithm breaks the automata states into groups that
cannot be distinguished. That is, it creates groups of states that are
equivalent to each other and therefore can be united to make a single
state. By the course of work, states are partitioned into groups that
cannot yet (or at all) be distinguished, and any two states from two
different group are distinguishable. Upon the next iteration the
current groups are broken into smaller ones in case the group has
distinguishable states. The algorithm stops as soon as no groups can
be partitioned anymore.

Before the algorithm starts working, the states are divided in two
groups - accepting states and nonaccepting states, which are
distinguishable by an empty string. Then we take a group from the
current partition and check whether it’s states can be distinguished
by some input character - whether some input character leads to two or
more different state groups. If it does, new groups are created so
that two states are grouped together if and only if they go to the
same group on the same input character. The process is repeated for
all groups in the current partition, then again for the new partition,
until no group can be split further.   

Then a representative is selected from each group so that:
\begin{enumerate}
  \item The start state of the new automaton is the representative of the
        group containing the start state of the old automaton.
  \item The accepting states of the new automaton are representatives of
        the groups that contain accepting states of the old automaton.
  \item If state $s$ is a representative of some group G and it has a
        transition to state $t$ of the old automaton. Let $r$ be the
        representative of the group containing $t$. Then we replace $t$ in all
        of the transitions from $s$ to the representative state $r$.
\end{enumerate}
Therefore we create a new automaton with the minimum number of states.
\end{document}

