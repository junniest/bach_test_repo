\section{Prototipa realizācija}
Pagaidām sistēma nav ieviesta Eq kompilatorā, bet atrodas prototipēšanas stadijā. \fixme{Šeit tiks aprakstītas izstrādātā prototipa īpašības un izvēlētās pieejas.}
\subsection{Pieejas izvēle}

\subsection{Lietotie algoritmi}
Determinizācija,
Minimizācija,
Apvienošana.

\subsection{Kāpēc tieši šāds risinājums}
???

Kāpēc šīm uzdevumam neder jau eksistējošas regulāro izteiksmju bibliotēkas. Kāpēc  neder vispārpieņemtie automātu apvienošanas algoritmi.
Regulāro izteiksmju dzinēji strādā ar tekstu, nevis ar tokeniem, nav vērts mēģināt pielāgot. Automātu apvienošana - visur aprakstītās pieejas nesaglabā, pie kāda no automātiem pieder katrs stāvoklis, it īpaši akceptējošie stāvokļi. Mums ir svarīgi zināt, kāds no automātiem ir akceptēts, jo no tā ir atkarīgs, kura no produkcijām tiks lietota. 

