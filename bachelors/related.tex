\section{Līdzīgu darbu apskats}
\label{s:related}

Šī nodaļa apskata darbus, kuriem piemīt transformācijas sistēmai līdzīga funkcionalitāte. Šie darbi ir līdzīgi pēc savas būtības un dažreiz piedāvā alternatīvas pieejas aprakstītam uzdevumam. Toties lielāka šādu risinājumu daļa ir ļoti cieši piesaistīta pie konkrētas valodas, pat ne pie valodu klases. Zemāk tiks aprakstīti daži līdzīgi darbi un parādītas atšķirības starp tiem un aprakstīto pieeju.

Šīs darbs tika iedvesmots ar dažiem rakstiem par dinamisko gramatiku iespējām un pielietojumiem programmēšanas valodu izstrādē. \ref{sbs:rel_dynamicgrammars} apakšnodaļa apskata izskatītus darbus par dinamiskām gramatikām.

Vispārīgi dinamiskas gramatikas un valodu dinamiska parsēšana gandrīz netiek lietota valodu implementēšanā. Tomēr valodu paplašināšana ir zināms uzdevums, kuram eksistē dažādi risinājumi. Katrs no risinājumiem ir darbspējīgs un pamatots priekš sava mērķa, un katram ir savas labās un sliktās puses. Šī nodaļas \ref{sbs:rel_lisp}, \ref{sbs:rel_forth}, \ref{sbs:rel_nemerle} un \ref{sbs:rel_openzz} apakšnodaļas piedāvā līdzīgu projektu un darbu apskatu, kā arī uzrāda aprakstītā projekta atšķirības no šiem projektiem.

Šī nodaļa neiedziļinās sistēmu sintakses īpatnībās, jo šāda apskate būtu pārāk apjomīga. Tā tikai pavirši apskata nozīmīgākas sistēmu īpatnības. Tālākai izpētei katra apakšnodaļa piedāvā literatūras avotus, kas piedāvā nepieciešamu informāciju.

\subsection{\label{sbs:rel_dynamicgrammars}Dinamiskas gramatikas}

Ir dažas dinamisku gramatiku pieejas, kas, diemžēl, vairākumā ir tīri teorētiskas. Labu ieskatu adaptīvo gramatiku pieejās dod Heninga Kristiansena raksts par adaptīvām gramatikām --\cite{Christiansen:SurveyAdaptableGrammars} un Džona Šutta maģistra darbs -- \cite{Shutt:AdaptiveGrammars}. Abi šie darbi apkopo visas uz to brīdi eksistējošās pieejas, bet, diemžēl, kopš šo rakstu laika citu ievērojamu variantu un implementāciju skaits ir ļoti mazs.

Dinamiskas gramatikas ir rīks, kas var ģenerēt neierobežotu kontekst-neatkarīgu gramatiku kopu. Katra gramatika ir veidota izejas teksta parsēšanas laikā sastopot speciālu konstrukciju. Dinamiska gramatika var tikt apskatīta ka parasto kontekstneatkarīgu gramatiku sekvence, ko definē programmas kods. Vispārīgi dinamiskas gramatikas var ļaut brīvi pievienot un dzēst savus produkciju likumus.

Toties pārsvarā dinamiskas gramatikas tiek lietotas lai pārbaudīt kontekst-atkarīgas sintakses korektumu, kas, piemēram, ir saistība starp mainīgā definīciju un tā lietošanu. Bez konteksta analīzes parsētājs nevarēs kļūdainu piešķiršanu, piemēram,  šādā izteiksmē \verb|int i; i = 'a'|. Šādas kļūdas parasti atrod kompilators pēc parsēšanas fāzes. Lietojot dinamiskās gramatikas katra mainīgā definīcija \verb|type id;| varētu pievienot gramatikai likumu, ka \verb|type| mainīgā vietā var būt mainīgais ar nosaukumu \verb|id|.

Šī ideja izskatās ļoti pievilcīgi no pirmā skata, jo tā ļaus izlaist veselu pārbaužu kopu no kompilatora implementācijas. Diemžēl šādu gramatiku izveide nav tik eleganta, tā ir viegli izveidojama tikai tādiem vienkāršiem gadījumiem ka mainīgo un funkciju signatūru definīcija. Bloka beigu gadījumā gramatikai ir jāprot izmest ārā visus likumus, kas tika pievienoti bloka iekšienē, kā arī kontrolēt mainīgo redzamību bloku ietvaros. Dinamisko gramatiku pieejas parasti arī neapstrādā rekursīvas deklarācijas. Vairāk informācijas var atrast Kristiansena rakstā \cite{Christiansen:SurveyAdaptableGrammars}.

Pjērs Bulliers savā rakstā par dinamiskām gramatikām \cite{Boullier:DynamicGrammars} apskata iespēju lietot adaptīvas gramatikas valodas sintakses kontrolei. Tas apraksta, kā tās dod iespēju realizēt kontekst-atkarīgas sintakses pārbaudi programmas parsēšanas, nevis kompilēšanas fāzē. Diemžēl, aprakstīta sistēma ir eksperimentāla un tikai prototipēta, nevis izveidota par lietojamu risinājumu. Tajā tiek lietota saskarne pie neeksistējoša orakula, kura mērķis ir risināt iespējamus gramatikas konfliktus.

\subsection{\label{sbs:rel_lisp}Lisp}

Lisp (\emph{LISt Processing}) ir viena no funkcionālam valodām, kuras ievērojama īpašība ir spēcīga meta-programmēšanas iespēja. Lisp ļauj paplašināt valodas konstrukcijas ar makro izteiksmēm un pievienot valodai jaunus atslēgas vārdus.

Lisp gan dati, gan programmas kods ir attēloti sarakstu veidā, tātad funkcijas var tikt apstrādātas tāpat ka dati. Tas dod iespēju rakstīt programmas, kas manipulē ar citām programmām un iedod bezgalīgas iespējas programmētājam, kuram nav nepieciešamības mācīt jaunu valodu, lai modificētu eksistējošo. Sintakses paplašināšana ir izpildāma lietojot pašu Lisp un tā makro sistēma ļauj veidot Lisp domēn-specifiskus dialektus.

Lisp makro apstrādes spējas ir ļoti specifiskas tieši šai valodai. Tas var tikt lietotas tāpēc, ka pati valoda ir speciālā veidā implementēta un uztver visu informāciju vienādi. Lisp makro sistēma bez izmaiņām nav pielietojama imperatīvām valodām, jo to instrukciju kopa ir cieši atdalīta no programmas datu kopas.

Lisp ļauj pievienot valodai jaunus atslēgas vārdus, bet neļauj veidot jaunus operatorus ne infiksā, ne postfiksā formā. Visām jaunām konstrukcijām joprojām jābūt prefiksa notācijā un to argumentiem saraksta formā.

Visas iegūtās konstrukcijas joprojām būs tīri funkcionālas, ar Lisp-specifisku sintaksi, t.i. nebūs iespējas izveidot moduļa pierakstu \verb/|a|/. Lisp sintakse ir grūti saprotama cilvēkam, kas nepazīst valodu programmēšanas līmenī, t.i. ja nestrādāja ar to jau iepriekš. Ar Lisp makro sistēmu nav iespējams izveidot sintaksi, kas būtu lasāma un saprotāma cilvēkam kas neprogrammē.

Plašāka informācija par Lisp un tās makro sistēmu ir atrodama, starp citiem avotiem, grāmatā \cite{Seibel:PracticalCommonLisp}.

\subsection{\label{sbs:rel_forth}Forth}

Forth ir steka valoda, kas neatbalsta nekādas programmēšanas paradigmas un vienlaikus atbalsta tās visas. Pateicoties Forth īpatnībām, tā var tikt lietota vienlaikus gan kā interpretators, gan kā kompilators.

Forth satur tikai divus daļiņu tipus, skaitļus un visas citas valodas vienības - vārdus. Šāda pieeja ļauj rakstīt programmas dabiskā valodā, nelietojot iekavas lai padotu parametrus vārdiem-funkcijām. Forth standarts definē speciālu vārdu kopu, kas ir iebūvēti valodā, bet tie arī var tikt pārdefinēti. Forth nesatur nekādus atslēgvārdus vispārpieņemtā nozīmē.

Visas konstrukcijas ir ieraksti Forth vārdnīcā, ar kuru var manipulēt kā ar datiem. No šī viedokļa Forth ir līdzīgs Lisp, kas arī uztver programmu un datus vienādi. Tas līdzīgi dod iespēju modificēt izpildāmo kodu un paplašināt valodas sintaksi, bez nepieciešamības mācīties jaunu transformācijas valodu.

Forth ļauj ne tikai veidot jaunas sintaktiskas konstrukcijas, bet ļauj arī iejaukties kompilēšanas procesā. Tas tiek atļauts ar speciāli definētiem vārdiem, un ļauj iegulst pat citu valodu kodu Forth programmā. Tomēr interpretēt iegulto kodu vajadzēs pašam programmētājam, kas grib to izpildīt.

Diemžēl Forth īpatnības padara to ļoti specifisku lietošanā. Postfiksā forma ir diezgan izteiksmīga valodiski\footnote{Sakarā ar sintakses īpašībām pēc filmas "Zvaigžņu kari" iziešanas uz ekrāniem, parādās joks par maģistra Jodas runas stilu - "The mistery of Yoda’s speech uncovered is: Just an old Forth programmer Yoda was".}, bet nav izteiksmīga gadījumos, kad ir nepieciešams ieviest matemātikas notācijas. Tam, ka var rakstīt programmas dabiskā valodā, arī ir divas monētas puses - ja katrs rakstīs savā valodā, citiem programmētājiem visticamāk būs grūti saprast (ja vien vispār tas būs iespējams). Forth implementēto sistēmu nebūs iespējams pielietot valodās, kuras satur vairākus tokenu tipus, jo nebūs iespējams apstrādāt kodu un datus vienādi.

Plašāka informācija par Forth valodu un tās iespējām ir atrodama tās mājaslapā \cite{ForthHome}.

\subsection{\label{sbs:rel_nemerle}Nemerle}

Nemerle ir statiski tipizējama universāla programmēšanas valoda .NET platformai. Tai piemīt gan funkcionālas, gan objektorientētas, gan imperativās paradigmas iezīmes. Tai ir C\#-līdzīga sintakse un ļoti spēcīga meta-programmēšanas sistēma.

Viena no svarīgākām Nemerle pazīmēm ir tas, ka tai ir raksturīga ļoti augsta līmeņa pieeja visiem valodas aspektiem. Tā ir statiski tipizējama un mēģina atbrīvot programmētāju no lieka darba lietojot tipu izvadīšanas iespēju un makro sistēmu. Tipu izvadīšana ļauj nerakstīt kodā tos tipus, kas var tikt izsecināti no koda gabala konteksta.

Nemerle makro sistēma dod iespēju ģenerēt bieži atkārtojāmo kodu bez programmētāja piepūles. Tas statiskā tipizācija ļauj kompilatoram izpildīt statiskas ģenerētā koda pārbaudes kompilācijas laikā. Tas kopumā dod iespēju programmatiski ģenerēt pārbaudāmu un tipu korektu kodu.

Nemerle realizētā makro sistēma ir ļoti līdzīga aprakstītai transformāciju sistēmai, bet kaut arī tā ir ļoti spēcīga un ļauj izpildīt daļēju novērtēšanu, tā ir ļoti atkarīga no valodas specifikas. Tā kā Nemerle ir statiski tipizēta, tā dod iespēju kompilatoram pārbaudīt kompilēto kodu, nevis ķert tipu kļūdas programmas izpildes laikā. Diemžēl plaši lietojamas valodas ne vienmēr ir statiski tipizējamas (piem. C/C++, Python), un šāda pieeja nebūs realizējama vairākumam valodu. 

Plašāka informācija par Nemerle valodu un makro sistēmas īpatnībām ir atrodama Nemerle mājaslapā \cite{NemerleWiki}.

\subsection{\label{sbs:rel_openzz}OpenZz}

OpenZz parsētājs ir interpretējams dinamisks parsētājs, kas ļauj ātri izstrādāt parsēšanas risinājumus. OpenZz ļauj modificēt un paplašināt parsētās valodas gramatiku lietojot komandas tajā pašā valodā. To var pielāgot dažādu valodu parsēšanai, bet izstrādāts tas tika lietošanai "Apese" programmēšanas valodai.

Ļoti svarīga šī parsētāja īpašība ir tas, ka tas ļauj modificēt valodas gramatiku ar pašas valodas palīdzību. Tomēr tā kā parsētājs atbalsta parsēšanas tabulu izmaiņas, kas ietekmē parsētāja ātrdarbību.

Tas nav pielietojams jebkādai jau eksistējošai valodai, jo valodā ir jāievieš speciāli modificēšanas mehānismi. Arī integrācija ar jau eksistējošiem kompilatoriem varētu būt problemātiska.

Diemžēl šīs parsētājs netiek attīstīts kopš 2002. gada un informācija par to ir ļoti ierobežota. Sīkāka informācija par šo rīku ir dabūjama rakstā \cite{Cabasino:DynamicParsers}, kas apskata parsētāja koncepciju, un parsētāja mājaslapa \cite{OpenZZParser}.