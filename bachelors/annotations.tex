\begin{abstract}

Parasti universālas valodas tiek lietotas specifisku uzdevumu risināšanai, bet vairākumā tiem piemīt ļoti ierobežota iespēja modificēt un papildināt savu sintaksi. Šīs darbs apskata iespēju ļaut programmēšanas valodas lietotājam paplašināt valodas sintaksi ar netriviālām sintaktiskām konstrukcijām, lai pielāgotu valodu saviem mērķiem.

Šīs darbs apraksta iespējamo koda transformācijas sistēmas koncepciju, kas ar nelielu darba ieguldījumu varētu ļaut papildināt plašas valodu klases sintaksi. Tas tiks izpildīts ar regulāro izteiksmju paplašinātu makro šablonu sistēmas palīdzību.

Šī darba ietvaros tika izstrādāta transformācijas sistēmas šablonu sakrišanu meklēšanas apakšsistēmas koncepcija un izveidota piemērotu algoritmu kopa. Tika izveidots apakšsistēmas prototips un identificēti iespējamie sistēmas ierobežojumi.

\kw{Atslēgvārdi}{Dinamiskās gramatikas, priekšprocesēšana, makro, regulārās izteiksmes, galīgi determinēti automāti, Python}
\end{abstract}

\begin{otherlanguage}{english}
\begin{abstract}
\textbf{Dynamic parsing using regular-expression-extended grammars}

Abstract text in English

\kw{Keywords}{Dynamic grammars, preprocessing, macros, regular expressions, determinate finite automata, Python}
\end{abstract}
\end{otherlanguage}