\section{Random thoughts}
\subsection{Our goals}
Apskatāmās sistēmas 2 galvenie mērķi ir dot iespēju ieviest jaunas konstrukcijas un tajā pašā laikā saglabāt korektu jau iepriekšeksistējošās sintakses apstrādi.

\subsection{Why adaptable grammars are cool}
Adaptīvās gramatikas ir ļoti spēcīgs rīks kompilatoru un parsētāju būvēšanā. 
Static semantics - gandrīz sintakse, bet ne gluži

\subsection{Why adaptable grammars suck}
В общем к ним сложно написать формализм и они не особо применимы реально, потому что очень уж ограничены.

Адаптирующиеся грамматики в общем случае мало того, что требуют пересчитывания всей таблицы парсинга - омг, так ещё и могут накосячить с тем, что потеряется парсабельность самой грамматики - LL(1) или LALR() или ещё какая-нибудь, которая необходима для того, чтобы работал парсер. Следовательно фиг ты их прикрутишь в обычном виде к уже существующему парсеру \cite{Christiansen:SurveyAdaptableGrammars}