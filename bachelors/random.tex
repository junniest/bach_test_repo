\section{Random thoughts}
\subsection{Our goals}
Apskatāmās sistēmas 2 galvenie mērķi ir dot iespēju ieviest jaunas konstrukcijas un tajā pašā laikā saglabāt korektu jau iepriekšeksistējošās sintakses apstrādi.

\subsection{Programmas konteksti}
Programmas konteksts (pēc Wikipedia) ir vismazākā datu kopa, ko vajag saglabāt programmas darbības pārtraukuma gadījumā, lai varētu atjaunot programmas darbu. Bet pašas programmas iekšienē var eksistēt lokālie konteksti, ko ievieš, piemēram, figūriekavas C/C++ gadījumā. Tad mainīgie, kas tiek definēti vispārīgā programmas kontekstā (globālie mainīgie), var tikt pārdefinēti mazākajā kontekstā (piemēram, kaut kādas funkcijas vai klases robežās) un iegūst lielāku prioritāti. Tas nozīmē, ka ja tiek lietots šāds pārdefinēts mainīgais, tas tiek uzskatīts par lokālu un tiek lietots lokāli līdz specifiska konteksta beigām, nemainot globālā mainīgā vērtību.

Konteksta piemērs:
\begin{singlespace}
\begin{verbatim}
int a = 0;
int b = 1;
int main() {
    int a = 2;
    a++;
    b += a;
}
\end{verbatim}
\end{singlespace}
Šajā piemērā \verb|a| ir definēta gan globāli, gan lokāli. Kad tiek izpildīta rindiņa \verb|a++;|, lokāla mainīgā vērtība tiks samazināta uz 3, jo \verb|a| ir pārdefinēts ar vērtību 2. Globālais \verb|a| tā ara paliks ar vērtību 0. Un kad tiks izpildīta rindiņa \verb|b += a;|, \verb|b| pieņems vērtību 4. Konteksta iekša tiks samainīta globālā mainīgā \verb|b| vērtība, jo tas netika pārdefinēts.

Tālāk termins koda konteksts tiks lietots tieši šajā nozīmē. 

\subsection{Why adaptable grammars are cool}
Adaptīvās gramatikas ir ļoti spēcīgs rīks kompilatoru un parsētāju būvēšanā. 
Static semantics - gandrīz sintakse, bet ne gluži

\subsection{Why adaptable grammars suck}
В общем к ним сложно написать формализм и они не особо применимы реально, потому что очень уж ограничены.

Адаптирующиеся грамматики в общем случае мало того, что требуют пересчитывания всей таблицы парсинга - омг, так ещё и могут накосячить с тем, что потеряется парсабельность самой грамматики - LL(1) или LALR() или ещё какая-нибудь, которая необходима для того, чтобы работал парсер. Следовательно фиг ты их прикрутишь в обычном виде к уже существующему парсеру \cite{Christiansen:SurveyAdaptableGrammars}
