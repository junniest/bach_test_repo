\section{Transformācijas sistēma}
\label{s:system}

Šī nodaļa piedāvā uzbūves principus sistēmai, kura ļaus lietotājam dinamiski paplašināt programmēšanas valodas iespējas ar makro valodas palīdzību. Šī makro valoda ļaus izveidot jaunas valodas konstrukcijas no jau eksistējošām vienībām. Sistēma ir projektēta ka virsbūve parsētājam un strādās paralēli ar parsētāju, analizējot kodu ar ierakstītiem makro un apstrādājot daļiņu virknes.

Šīs sistēmas mērķis ir piedāvāt iespēju modificēt valodas sintaksi programmas rakstīšanas gaitā, nebojājot jau eksistējošo konstrukciju darbu. Sistēma ieviesīs pašmodificēšanos lietojot konstrukciju aizvietošanu, kas vienlaikus nodrošinās gramatikas modifikācijas un sākotnējās valodas gramatikas nemainīgumu. Tajā pašā laikā sistēma būs stabila pret kļūdām dēļ tā, ka tā strādās tikai konkrētās gramatikas produkcijas ietvaros un tā, ka tā pārbaudīs tipus jaunizveidotām virknēm.

Sistēmas raksturīga īpašība ir tas, ka tā nav piesaistīta konkrētai programmēšanas valodai. Uz doto brīdi tā var tikt pielāgota un integrēta dažādu valodu parsētājos, kuru arhitektūra atbilst dažiem nosacījumiem. Tai nav ierobežojumu pret atbalstāmo sintaksi, jo tā strādās ārpus valodas gramatikas.

%Tālāk nodaļa ir organizēta sekojoši. Apakšnodaļa~\ref{sbs:sys_architecture} iedod ieskatu plānotā sistēmas arhitektūrā. Apakšnodaļa~\ref{sbs:sys_approach} vispārīgi pamato izvēlēto transformācijas pieeju. Apakšnodaļa~\ref{sbs:sys_parserqualities} apraksta īpašības, kuram jāpiemīt parsētājam, lai tas varētu iekļaut transformācijas sistēmu. Apakšnodaļa~\ref{sbs:sys_macrosyntax} apraksta un pamato sistēmas makro sintaksi. Apakšnodaļa~\ref{sbs:sys_qualities} detalizētāk apskata sistēmas sadalījumu uz trim apakšsistēmām un apraksta to sadarbību.

\subsection{\label{sbs:sys_parserqualities}Parsētāju īpašības}

Šajā darbā piedāvātā sistēma tiek izstrādāta uz LL(k) parsētāja bāzes. Lai parsētājs varētu tikt integrēts ar aprakstāmo sistēmu, tam jābūt implementētam ar rekursīvas nokāpšanas algoritmiem LL(k) vai LL(*). LL parsētājs tika izvēlēts tāpēc, ka LL ir viena no intuitīvi saprotamākām parsētāju rakstīšanas pieejam, kas ar lejupejošo procesu apstrādā programmatūras tekstu. LL parsētājiem nav nepieciešams atsevišķs darbs parsēšanas tabulas izveidošanā, tātad parsēšanas process ir vairāk saprotams cilvēkam un vienkāršāk realizējams.

Tā kā transformāciju sistēma tiek veidota ka paplašinājums parsētājam, parsētājam jāatbilst dažiem nosacījumiem, kas ļaus sistēmai darboties. Zemāk ir aprakstītas īpašības, kuram jāpiemīt parsētājam, lai tas varētu veiksmīgi sadarboties ar aprakstāmo sistēmu.

\begin{description}
\item[Daļiņu virkne]
Parsētājam jāprot aplūkot daļiņu virkni ka abpusēji saistītu sarakstu, lai eksistētu iespēja to apstaigāt abos virzienos. Tam arī jādod iespēju aizvietot kaut kādu daļiņu virkni ar jaunu un ļaut uzsākt apstrādi no patvaļīgas vietas daļiņu virknē.

Tas ir nepieciešams, lai transformācijas sistēma varētu aizvietot transformētās virknes visā programmas daļiņu virknē.

\item[Pseido-daļiņas]
Parsētāji parasti pielieto (reducē) gramatikas likumus ielasot daļiņas no ieejas virknes. Pseido-tokens, savukārt, ir produkcijas redukcijas rezultāts. Tas tiek attēlots ka atomārs ieejas plūsmas elements, bet īstenībā attēlo kaut kādu valodas gramatikas netermināļi. Viena no pseido-daļiņām, piemēram, ir izteiksmes daļiņa - \verb|{expr}|, kas var sastāvēt no daudziem dažādiem daļinām (piem. \verb|(a+b*c)+d|).

Šāda tipa daļiņas ir nepieciešamas, lai šablonu sistēma varētu meklēt sakritības ar gramatikas produkcijām, neimplementējot gramatikas atpazīšanu. Piemēram, ja ir nepieciešams aizvietot visas izteiksmes pēc vienādojuma zīmes ar tām pašām virknēm, bet ar iekavām, tad varēs lietot izteiksmes pseido-daļiņu, nemēģinot šablonā aprakstīt visas iespējamas izteiksmes variācijas.

\item[Vadīšanas funkcijas]
Sistēmas darbam ir prasīts, lai katra gramatikas produkcija tiktu reprezentēta ar vadīšanas funkciju (\emph{handle-function}). Vadīšanas funkcijām jāeksistē katrai gramatikas produkcijai, un katra šāda tipa funkcija prot atpazīt un reducēt produkciju.

Ir svarīgi atzīmēt, ka šim funkcijām būs blakus efekti - daļiņu virknes pārveidošana pēc likuma reducēšanas, tāpēc to izsaukšanas kārtība ir svarīga. Šīs funkcijas atkārto gramatikas struktūru, tas ir ja gramatikas produkcija A ir atkarīga no produkcijas B, A-vadīšanas funkcija izsauks B-vadīšanas funkciju.

\item[Ir-funkcijas]
Katra no vadīšanas funkcijām nāk pārī ar predikāta tipa funkciju (\emph{is-function}). Šīs funkcijas pārbauda, vai tajā vietā daļiņu virknē, uz kura dotajā brīdī atrodas parsētāja radītājs, ir atrodama apakšvirkne, kas atbilst dotās produkcijas aprakstam. Šādu funkciju pielietošana nemaina parsētāja stāvokli.

\item[Sakrišanas funkcijas]
Katras vadīšanas funkcijas darbības sākumā vai beigās (tas ir atkarīgs tikai no implementācijas izvēles) tiek izsaukta tā sauktā sakrišanas funkcija (\emph{match-function}). Sakrišanas funkcija ir transformācijas sistēmas saskarne, kas, zinot, kādā produkcijā uz doto brīdi strādā parsētājs, mēģina izdarīt daļiņu virknes transformāciju.

Tā pārbauda, vai tā vieta tokenu virknē, uz kuru rāda parsētājs, ir derīga kaut kādai transformācijai dotās produkcijas ietvaros. Ja pārbaude ir veiksmīga, funkcija izpilda sakrītošās virknes substitūciju ar jaunu virkni un parsētāja stāvoklī uzliek norādi uz aizvietotās virknes sākumu. Gadījumā,  ja pārbaude nav veiksmīga, funkcija nemaina parsētāja stāvokli, un parsētājs var turpināt darbu nemodificētas gramatikas ietvaros.
\end{description}

Ja izstrādājamās valodas parsētāja modelis atbilst aprakstītām īpašībām, tad tā var tikt veiksmīgi savienota ar aprakstāmo transformācijas sistēmu un ļaut programmētājam ieviest modifikācijas oriģinālās valodas sintaksē.

\subsection{\label{sbs:sys_macrosyntax}Makro sistēmas sintakse}

Makro sintakse ir veidota transformācijas likuma izskatā. Transformāciju likums sastāv no divām daļām. Kreisā puse satur šablonu un produkciju, kurā tā var tikt pielietota. Makro šablons satur regulāro izteiksmi no daļiņām un pseido-daļiņām, kas tālāk tiek izmantota lai atrast virkni, kurai šī transformācija ir pielietojama. Labā makro puse satur transformācijas funkciju un produkciju, kurai ir jāatbilst transformācijas rezultātam.

Makro izteiksmes strādā stingri kaut kādas produkcijas ietvaros, tāpēc makro šablona sintaksē tiek lietoti produkciju nosaukumu apzīmējumi. Tie tiks lietoti lai pārbaudītu pseido-daļiņu virknes korektumu sākotnējās gramatikas ietvaros pēc sintakses izmaiņu ieviešanas. Par tipu pārbaudi tiks runāts zemāk šajā nodaļā.

Apskatīsim \textit{match} funkciju likumus, kas modificē apstrādājamās gramatikas produkcijas uzvedību. \emph{Match} makro sintakses vispārīgo formu var redzēt attēlā~\ref{fig:matchsyntax}.

\begin{figure}[h!]
\begin{verbatim}
match [\prod1] v = regexp → [\prod2] f(v)
\end{verbatim}
\caption{\label{fig:matchsyntax}\emph{Match} makro sintakses vispārīgā forma}
\end{figure}

Šīs apraksts ir uztverams sekojoši. Ja produkcijas \verb|prod1| sākumā ir atrodama pseido-daļiņu virkne, kas atbilst regulārai izteiksmei \verb|regexp|, tad tai tiek nosacīti piekārtots vārds \verb|v|. Mainīgais \verb|v| ir visu atrasto daļiņu virkne, kas var tikt lietota makro labajā pusē pārveidošanas funkcijas izpildē. Tātad ja tāda virkne \verb|v| eksistē, tā tiek aizstāta ar pseido-daļiņu virkni, ko atgriezīs \verb|f(v)| un kurai ir jāatbilst gramatikas produkcijas \verb|prod2| likumiem. 

Regulārā izteiksme \verb|regexp| ir vienkārša standarta regulārā izteiksme, kuras gramatika ir definēta attēlā~\ref{fig:regexpsyntax}. Pagaidām sistēmas prototipa izstrādē tiek lietota šāda minimāla sintakse, bet nepieciešamības gadījumā tā var tikt paplašināta.

\begin{figure}[h!]
\begin{verbatim}
regexp          → concat-regexp | regexp
concat-regexp   → asterisk-regexp  concat-regexp
asterisk-regexp → unary-regexp * | unary-regexp
unary-regexp    → pseudo-token | ( regexp )
\end{verbatim}
\caption{\label{fig:regexpsyntax}Regulāro izteiksmju gramatika uz pseido-tokeniem}
\end{figure}

Piemēros uzskatības nolūkos \verb|v| virkne būs uzskatīta par apstaigājamu un lietota ar indeksiem, kas norāda strastās virknes elementa numuru.

Tagad mēs varam izveidot definētās makro sintakses korektu piemēru. Pieņemsim, ka ērtības dēļ programmētājs grib ieviest sekojošu notāciju absolūtās vērtības izrēķināšanai - \verb/|{expr}|/. Sākotnējā valodas gramatikā eksistē absolūtās vērtības funkcija izskatā \verb|abs({expr})|. Tad makro, kas parādīts attēlā~\ref{fig:matchsample1} izdarītu šo substitūciju, ļaujot programmētājam lietot ērtāku funkcijas pierakstu.

\begin{figure}[h!]
\begin{verbatim}
match [{expr}] v = {|} {expr} {|}
    → [{expr}] {id:abs} {(} v[1] {)}
\end{verbatim}
\caption{\label{fig:matchsample1}Makro piemērs \#1}
\end{figure}

Šajā piemērā šablons \verb/{|} {expr} {|}/, strādājot \verb|{expr}| produkcijas ietvaros, atradīs sakritību ar trim daļiņām, \verb/|/, kaut kāda izteiksme, \verb/|/. Transformējot atrasto elementu sarakstu tiek veidots jauns saraksts no \verb|{id:abs}|, \verb|{(}|, atrastās izteiksmes \verb|v[1]| un \verb|{)}|. Izveidotam sarakstam jāatbilst produkcijai \verb|{expr}|.

Vēl viens korektā makro piemērs: pieņemsim, ka funkcija \verb|replace| ir definēta transformāciju valodā ar trim argumentiem, un darba gaitā tā jebkurā pseido-tokenu virknē aizvieto elementus, kas sakrīt ar otro argumentu, ar trešo funkcijas argumentu. Pieņemsim, ka mums ir nepieciešams aizvietot visas funkcijas \verb|sum| izsaukšanas reizes ar tās argumentu summu. Funkcija \verb|sum| ir iespējams padot patvaļīgu argumentu skaitu, kas ir lielāks par vienu. Šādā gadījumā makro, kas parādīts attēlā~\ref{fig:matchsample2}, izpildīs nepieciešamu darbību.

\begin{figure}[h!]
\begin{verbatim}
match [{expr}] v = {id:sum} {(} {expr} {,} {expr} ( {,} {expr} ) * {)}
    → [{expr}] replace(v, {,}, {+})
\end{verbatim}
\caption{\label{fig:matchsample2}Makro piemērs \#2}
\end{figure}

\subsection{\label{sbs:sys_qualities}Transformācijas sistēmas apakšsistēmas}

Šī nodaļa parāda sistēmas sadalīšanu uz trim neatkarīgam daļām. Pirmā no tām ir sakrišanu meklēšanas daļa. Tā tokenu virknē atrod makro šablonu satikšanas reizes. Otrā ir atrasto virkņu apstrādes daļa. Tā pārveido sakrišanas mehānisma atrasto tokenu virkni atbilstoši tam, kas norādīts makro labajā daļā. Trešā ir tipu pārbaudīšanas sistēma, kas statiski pārbauda, vai uzrakstītais makro var būt derīgs valodas gramatikas ietvaros. Šīs sadalījums ir tikai konceptuāls, kas ir izveidots ērtības dēļ, lai varētu apskatīt sistēmu ka atsevišķu apakšsistēmu kombināciju.

Šablonu sakrišanu meklēšanas sistēma ir īsumā aprakstīta apakšnodaļā~\ref{sbsbs:sys_matches}. Sīkāk šīs apakšsistēmas īpašības un tās prototipa realizācija ir aprakstīta nodaļā~\ref{s:prototype}.

Ir nepieciešams izveidot mehānismu, kas ļaus transformēt makro kreisās puses akceptētu pseido-tokenu virkni, izveidojot virkni, kas to aizvietos. Lai to izdarītu ir nepieciešama kaut kāds papildus rīks, par kuru ies runa apakšnodaļā~\ref{sbsbs:sys_language}.

Ir plānots, ka transformāciju sistēma varēs atpazīt nepareizi sastādītus makro šablonus lietojot tipu kontroli. Šīs pieejas bāzes principi ir aprakstīti apakšnodaļā~\ref{sbsbs:sys_typesystem}. Jāņem vērā tas, ka lai šī sistēma varētu tikt pielietota, izvēlētai transformāciju valodai jāpiemīt tipu secināšanas (\emph{type inference}) īpašībai.

\subsubsection{\label{sbsbs:sys_matches}Sakrišanu meklēšana}

Šablonu apstrādei ir nepieciešama kāds sakrišanu meklēšanas mehānisms. Tā kā šabloniem tika izvēlēts lietot regulāro izteiksmju elementus, parastā virkņu saildzināšana nedos vēlamo rezultātu. Tāpēc tika izvēlēts veidot apakšsistēmu, kas varēs apstrādāt un saglabāt ielasītās regulārās izteiksmes un veikt ieejas virknes apstrādi, meklējot sakrišanas.

Galvenais princips uz kura jābāzējas šīs apakšsistēmas izstrādē minimāls apstrādes laiku. Tātad apakšsistēmas koncepcija  tika izveidota tā, lai jebkurā laika momentā šablonu sakrišanu meklēšanai būtu nepieciešams lineārs laiks un tikai viena daļiņu virknes apstaigāšana. Šāda pieeja ir izvēlēta ar iedomu, ka makro pievienošana tiks izpildīta tikai vienreiz, un to daudzums būs samērā neliels, bet sakrišanu meklēšana tiks pildīta katrā produkcijā, un, sliktākajā gadījumā, katrai daļiņai no ieejas plūsmas.

Darba ietvaros šī apakšsistēma tika prototipēta

Ielasītā regulārā izteiksme tiek pārsēta un pārveidota nedeterminēta galīgā automātā. Tad šablona nedeterminēts galīgs automāts tiek determinēts un minimizēts. Tātad katrai regulārai izteiksmei tiek izveidots minimāls determinēts automāts, kurš ir optimizēts gan pēc apstaigāšanas laika, gan pēc aizņemtās vietas.

Tālāk, lai nodrošinātu visu šablonu pārbaudi vienlaikus un minimizēt meklēšanas laiku, ir nepieciešams apvienot izveidotos automātus. To var izdarīt dažos veidos. Vienkāršākais no tiek būtu glabāt visus galīgos automātus sarakstā. Pieņemsim, ka ir $n$ šabloni, kurus vajag pārbaudīt. Tad automātu saraksts reprezentē nedeterminētu galīgu automātu ar $n$ $\varepsilon$-zariem no sākuma stāvokļa, katrs no kuriem ved pie sākuma stāvokļa vienam no jau izveidotiem determinētiem automātiem.

Cits veids, kā to varētu izpildīt, ir apvienot visus izveidotos šablonu automātus vienā determinētā galīgā automātā. Tieši šīs veids tika izvēlēts šī darba ietvaros lai pēc iespējas samazinātu laiku sakrišanu meklēšanai. Kaut arī automātu apvienošana šādā veidā ir laikietilpīga, tā samazina laika kārtu sakritību meklēšanai.

Sīkāk šīs apakšsistēmas prototipa īpašības un lietotie algoritmi ir izklāsti nodaļā~\ref{s:prototype}.

\subsubsection{\label{sbsbs:sys_language}Daļiņu virkņu apstrāde}

Gadījumā, ja šablonu sistēma nesaturētu regulāro izteiksmju sintaksi (it īpaši \verb|*|), būtu iespējams transformēt tokenu virknes ar pašu makro palīdzību. Atrastās tokenu virknes vienmēr būtu ar vienādu un determinētu garumu un saturu. Bet tā kā šabloni ļauj meklēt sakrišanas ar elementu sarakstiem, ir nepieciešams veids, kā apstrādāt jaunizveidotus un, iespējams, tukšus sarakstus.

Tātad lai varētu izpildīt atrastās tokenu virknes apstrādi un modificēšanu ir nepieciešams kaut kāds papildus rīks. Šīs rīks varētu būt kaut kāda programmēšanas valoda. Izplatītākās programmēšanas valodu paradigmas mūsdienās ir imperatīvā  vai funkcionālā paradigma. Katru no tiem varētu lietot atrasto virkņu apstrādei.

Šīm uzdevumam varētu lietot kādu no imperatīvam programmēšanas valodām, piemēram C, vienkārši izveidojot saskarni ar tās valodas kompilatoru. Bet vairākumam šādu valodu nav statiskas tipu secināšanas iespējas. Statiska tipu secināšana C valodas gadījumā ir neiespējama rādītāju mainīgo dēļ, kuru tipus nevar izrēķināt parsēšanas laikā. Lai varētu ieviest statiskās tipu izsecināšanas iespējas, vajadzēs ierobežot valodas iespējas, tātad modificēt eksistējošo kompilatoru vai kaut kā citādāk ierobežot pieejamo konstrukciju kopu.

Uzdevumam arī varētu lietot kādu no jau eksistējošām funkcionālām valodām ar tipu secināšanas īpašību. Tad nebūs nepieciešamības veidot savu valodu pilnīgi no jauna. Bet tas nozīmēs, ka būs rūpīgi jāizpēta izvēlētās valodas sintaksi, kas var būt pārāk grūti.

Varētu lietot arī vienu no jau eksistējošām funkcionālām valodām ar tipu secināšanas īpašību, kas piemīt vairākumam funkcionālo valodu. Vēl viena ērtā funkcionālo valodu īpašība ir tas, ka tās funkcijām nepiemīt blakusefekti, tātad to izpilde nevarēs samainīt eksistējošos datus. Valoda, kuras funkcijām ir blakusefekti, varētu sabojāta parsētāja darbu.

Šīs sistēmas implementācijā tika izvēlēts izveidot minimalistisku funkcionālu valodu, kura būs statiski tipizējama. Tātad visiem šīs valodas mainīgajiem varēs izsecināt piederību pie tipa un pie kaut kāda virstipa, kas tiks lietots lai nodrošināt transformāciju korektumu. Funkcionālā pieeja nodrošina arī to, ka apstrādes funkcijām nepiemīt blakusefekti, kas varētu samainīt eksistējošos datus. Valodas mērķis ir ļaut izveidot atrasto tokenu permutācijas ar kādiem papildinājumiem nepieciešamības gadījumā.

Galvenais šīs valodas pielietojums ir dot iespēju apstaigāt pseido-tokenu virkni, kura tika atzīta par sakrītošu ar atbilstošu šablonu. Lai to darīt, tā dos iespēju lietot rekursiju un dažas iebūvētās funkcijas - saraksta pirmā elementa funkciju \verb|head|, saraksta astes funkciju \verb|tail| un objektu pāra izveidošanas funkciju \verb|cons|. Funkcija \verb|cons| funkcionālo valodu kontekstā strādā kā saraksta izveidošanas funkcija, jo saraksts \verb|list(1, 2, 3)| tiek reprezentēta ka \verb|cons(1, cons(2, cons(3, nil)))|, kur \verb|nil| ir speciāls tukšs objekts. Valoda saturēs arī \verb|if| konstrukciju, kas ļaus pārbaudīt dažādus nosacījumus.

Lai būtu iespēja apstādināt rekursiju, šī valoda arī ļaus izpildīt aritmētiskās operācijas ar veseliem skaitļiem. Tas dos iespēju izveidot skaitītājus un izveidot rekursijas izejas nosacījumus.

Tiek plānots, ka šī valoda arī ļaus izpildīt daļēju novērtējumu izteiksmēm, tur kur būs nepieciešams. Tas nozīmē, ka valodai jāsatur saskarne, kas ļaus piekļūt pie tokena vērtības. Šim mērķim ir domāta funkcija \verb|value|, kas ir pielietojama pseido-tokeniem ar skaitlisku vērtību, piemēram, lai dabūt skaitli \verb|5| no pseido-tokena \verb|{int:5}|. Valoda arī ļaus izveidot jaunus tokenus ar izrēķinātu vērtību.

Funkcija \verb|type|, savukārt, ļaus pārbaudīt tokenu tipu, kas var būt nepieciešams transformācijas procesā, piemēram, lai atpazīt kādu operatoru.

Lai būtu iespēja apstādināt rekursiju, šī valoda arī ļaus izpildīt aritmētiskās operācijas ar veseliem skaitļiem. Tas dos iespēju izveidot skaitītājus un izveidot rekursijas izejas nosacījumus.

\subsubsection{\label{sbsbs:sys_typesystem}Tipu sistēma}

%\fixme{надо убрать ощущение что я ничего не знаю (Х }
\fixme{идея строится вокруг того, что вот такое вот наблюдение}

Kā bija redzams attēlā~\ref{fig:matchsyntax}, katrā makro pusē ir atrodams produkcijas nosaukums, \verb|[prod1]| un \verb|[prod2]|. Tas tiek darīts tādēļ, lai kontrolētu, kad dotais makro ir pārbaudīts, un kāda tipa izejas virkni tas radīs. Abas šīs atzīmes ir rādītas tipu kontroles sistēmas dēļ.

Katrs atsevišķs makro strādā konkrētas gramatikas produkcijas ietvaros, \verb|[prod1]| dotā makro gadījumā. Tas nodrošinās to, ka katrs no makro tiks izpildīts pareizajā vietā un visas konstrukcijas tiks apstrādātas. 

Otrais tips, \verb|[prod2]|, atzīmē to, ka pēc transformācijas procesa beigām mums jāsaņem tieši šādai produkcijai korektu izteiksmi. Tātad ir jāparbauda tas, ka funkcijas \verb|f(v)| rezultāts attiecībā uz atrasto tokenu virkni, ir atļauta ieejas virkne priekš produkcijas \verb|prod2|.

Lai to paveikt ir nepieciešams izveidot pseido-tokenu regulāro izteiksmi produkcijai \verb|prod2|. Tālāk ir nepieciešams izsecināt funkcijas \verb|f| no virknes \verb|v| rezultāta tipu.

\fixme{v  нужно обязательно из грамматического правила построить регулярное выражение (возможно более общее) и проверить подтип ли. Можно по-другому - вывести из регулярки грамматику, и проверить является ли грамматика подграмматикой}

Šajā darbā netiks apskatīts jautājums, kādā veidā tiks izveidota regulārā izteiksme priekš katras gramatikas produkcijas. To varētu izveidot programmētājs, vai, varbūt tā varētu tikt izveidota automātiski. Ir svarīgi pieminēt, ka pāreja no gramatikas likuma uz regulāro izteiksmi noved pie kādas informācijas zaudēšanas. Piemēram, nav iespējams uzkonstruēt precīzu regulāro izteiksmi valodai:

\begin{verbatim}
A :=  aAb | ab
\end{verbatim}

Tomēr ir iespējams izveidot regulāro izteiksmi kas iekļaus sevī gramatikas aprakstīto valodu, piemēram, \verb|a+b+|. Makro lietotā transformācijas shēma tiks atzīta par pareizo, ja ir iespējams pierādīt, ka produkciju aprakstošā regulārā izteiksme atpazīst arī valodu, ko veido \verb|f(v)|.

Ir viegli pamanīt, ka regulārās izteiksmes izveido dabisku tipu hierarhiju. Valoda, kura var tikt atpazīta ar regulāro izteiksmi $r_1$, var tikt iekļauta citas regulārās izteiksmes $r_2$ atpazītās valodā apakškopas veidā. Piemēram, regulārās izteiksmes \verb|a+| valoda ir atpazīstama arī ar regulāro izteiksmi \verb|a*|, bet \verb|a*| atpazīst vēl papildus tukšu simbolu virkni. Šādai tipu hierarhijai uz regulārām izteiksmēm eksistē arī super-tips, ko uzdod regulārā izteiksme \verb|.*| - $\top$. Ir acīmredzami, ka $\forall t_i \in R, t_i \sqsubseteq \top$, kur $R$ ir visu regulāro izteiksmju kopa.

Ir svarīgi izveidot procedūru, kas ļaus izsecināt, vai $r_1 {\sqsubseteq} r_2$. Ir zināms, ka ir iespējams katrai regulārai izteiksmei uzbūvēt minimālu akceptējošu galīgu determinētu automātu\footnote{Sk. nodaļu~\ref{sbs:prot_algorithms}.}. Šīs automāts atpazīs precīzi to pašu valodu, ko atpazīst regulārā izteiksme. Tas nozīmē, ka no $r_1 \sqsubseteq r_2 \Rightarrow seko min (det (r_1)) \sqsubseteq min (det (r_2))$. Diviem minimāliem automātiem $A_1$ un $A_2$, $A_1 \sqsubseteq A_2$ nozīmē, ka eksistē kaut kāds attēlojums $\Psi$ no $A_1$ stāvokļiem uz $A_2$ stāvokļiem, tāds, ka:

\[
    Start (A_1) \to Start (A_2) \in \Psi
\]
\[
    \forall s \in States (A_1) \forall e \in Edges (s),
    \Psi (Transition (s, e)) = Transition (\Psi (s), e)
\]

Šeit $States (x)$ apzīmē automāta $x$ stāvokļu kopu, $Edges (s)$ apzīmē pseido-tokenu kopu, kas atzīmē no stāvoļa $s$ izejošās šķautnes. $Transition(s, t)$, savukārt, apzīmē stāvokli, kas ir sasniedzams no $s$ pārejot pa šķautni, kas atzīmēta ar pseido-tokenu $t$.

Otra svarīga īpašība, kas tiks lietota šajā tipu pārbaudīšanas sistēmā ir tas, ka transformāciju valoda ir statiski tipizējama un tā satur ļoti ierobežotu iebūvēto funkciju skaitu. Katrai no šīm iebūvētām funkcijām ir iespējams izveidot to aprakstošo regulāro izteiksmi. Piemēram, regulārā izteiksme funkcijai $head (x)$ var tikt izveidota ka visu to šķautņu kopa, kas iziet no $x$ aprakstošā automāta sākuma stāvokļa. Kad šablona apstrāde tiek sākta, var arī izsecināt iespējamo virknes garumu intervālu un izvadīt brīdinājumus gadījumā, ja ir iespējama funkcijas $head (x)$ izsaukšana no tukšā saraksta.

\fixme{Все остальные действия выражаются при помощи тейла хеда и конса и рекурсии, для каждой операции можно вывести тип. Подробности находятся в стадии рисёрча.}

Šī nodaļa tikai vispārīgi apraksta tipu secināšanas sistēmas ideju. Uz doto brīdi tā atrodas izstrādes stadijā, bet var redzēt ka tās izveide ir pamatota. Sīkāka informācija par tipu sistēmu ir saņemama pie Eq kompilatora izstrādes komandas.