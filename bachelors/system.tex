\section{Par izstrādājamo sistēmu (ievads \#3)}
Ka var redzēt no iepriekšējās nodaļas, pašmodificējošās gramatikas ir diezgan sarežģīts rēks, kas kaut arī ir ļoti lietderīgs, mūsdienās gandrīz netiek lietots. Tas netiek lietots savas sarežģītības dēļ un dēļ tā, ka vispārīgā gadījumā pašmodificējošo gramatiku ir ļoti grūti kontrolēt. Ļaujot neierobežoti modificēt gramatiku mēs varam nonākt pie gadījuma, kad sākotnējā gramatika tiek izmesta ārā, bet tās vietā parādās cita, pilnīgi jauna. Tas netiek lietots savas saredzamības dēj un dēļ tā, ka vispārīgā gadījumā pašmodificējošo gramatiku ir ļoti grūti kontrolēt. Ļaujot neierobežoti modificēt gramatiku mēs varam nonākt pie gadījuma, kad sākotnēja gramatika tiek izmesta ārā, bet tās vietā parādās cita, pilnīgi jauna. Šādā gadījumā šīs jaunās gramatikas adekvātumu un korektību nevar garantēt.

Šīs darbs apraksta iespēju izveidot pašmodificējošo kodu ar funkcionālās makro valodas palīdzību. Šī makro valoda ļaus izveidot jaunas valodas konstrukcijas no jau eksistējošām vienībām. Parsētāja darba laikā makro sastapšanas reizes tiks pārrakstītas uz kodu ar attiecīgu struktūru, kas var tikt atpazīti ar valodas sākotnējo gramatiku. Tātad šī sistēma ļaus modificēt gramatiku nebojājot jau eksistējošo sintaksi. Neko pavisam jauno šī makro sistēma nejaus izveidot, lai paliktu savietojamība ar sākotnējo gramatiku, tomēr tā ļaus atvieglot programmētāja darbu dodot iespēju aizstāt kodā sarežģītas konstrukcijas ar vienkāršākām. 
\subsection{Origins - Eq}
Valoda Eq tiek izstrādāta (kur?). Šīs valodas sintakse bāzējas uz LaTeX teksta procesora sintakses, kas ir standarts priekš zinātniskām publikācijām. Konsekventi programma, kas rakstīta valodā Eq ir korekti interpretējama ar LaTeX procesoru. Tajā pašā laikā Eq programma varēs tikt kompilēta vairākumam mūsdienīgu arhitektūru. 
\fixme{Saite uz Eq projektu}

Šeit būs pavisam nedaudz informācijas par to, kas ir Eq, kāpēc vispār parādies šī ideja.

\subsection{Sintakse un darbība}
fixme{Šeit būs makro sintakses piemērs} 

\subsection{Pieejas universālums}
Kaut arī ideja un pieejas izstrāde sākās ar valodu Eq, tā nav piesaistīta tieši šai valodai. Visspēcīgāka šīs sistēmas īpašība ir tas, ka tā ir universāla un var tikt pielietota jebkādam parserim kas atbilst dažiem nosacījumiem. 
Par parsētāja modeli

\fixme{Kāpēc ir izvēlēts tieši LL(k) parsētāji?}
\subsection{Sistēmas īpašības}
Šī nodaļa aprakstīs, kā mēs gribam realizēt gramatikas pašmodificēšanos, lai izmaiņas būtu kontrolētas.
Mēs gribam norobežot modificēšanas iespējas

\subsubsection{Pārrakstīšanas sistēma}
Match sistēma, atļauto regulāro izteiksmju sintakse. Sintakse ir viegli paplašināma.
Konteksti
\subsubsection{Tipu sistēma}
Kā tiks pārbaudīti tipi.