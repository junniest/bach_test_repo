\section{Ievads}
Kontekstneatkarīgas gramatikas ir vienkāršs un uzskatāms rīks, kas ļauj aprakstīt programmēšanas valodu sintaksi. Diemžēl, mūsdienīgo programmēšanas valodu sintaksi ir grūti aprakstīt ar viennozīmīgām kontekstneatkarīgām gramatikām. Vairākumam modernu valodu parsētāji ir rakstīti ar rokām, programmētājam uzmanīgi risinot gramatikas konfliktus.  Eksistē arī parsētāju ģeneratori, piemēram ANTLR, kas ar ieskatu uz k tokeniem uz priekšu tokenu virknē var izlemt, kā atrisināt parsēšanas neviennozīmības. Tomēr lietojot šādus parsētāju ģeneratorus ir jāprot veidot tiem saprotamas gramatikas un konfliktu risināšanas noteikumus.

Šī darba mērķis ir aprakstīt iespēju dot programmētājam papildināt programmēšanas valodu sintaksi. Lai to īstenot, var piedāvāt divus variantus. Pirmais no tiem ir pārrakstīt vai pārģenerēt parsētāju līdzko parādās nepieciešamība ieviest jaunas konstrukcijas. Ir pašsaprotami, ka radikālu gramatikas izmaiņu gadījumā tas tiešām būs nepieciešams. Tomēr gadījumos, kad izmaiņas tiek veiktas lai atvieglotu programmētāja darbu vai padarītu programmas kodu pārskatāmu, parsētāja pārrakstīšana ir lieks darbs. Otrais variants, pie kura pieturas dotais darbs, ir iznest sintakses modificēšanu uz lietotāja līmeni. Tas nozīmē, ka valodas lietotājam jāpiedāvā kaut kāds mehānisms valodas konstrukciju manipulēšanai.

Dotais darbs apskata sistēmu, kas ļauj dinamiski paplašināt programmēšanas valodu sintaksi. Par pamata principu šīs sistēmas izstrādē ir ņemts pašmodificējamo gramatiku jēdziens. Par dinamiski modificējamo gramatiku iespējamību un realizējamību iet runa dažādos rakstos. Tomēr vispārīgā gadījumā šādām gramatikām trūkst jebkādu kontroles mehānismu. Tādas gramatikas var nekontrolējami mainīties, aizvietojot sākotnējo gramatiku ar pavisam jaunu. Tas var izraisīt dažādas problēmas, bet galvenās no tām ir parsējamības bojāšana (piemēram, kreisās rekursijas ieviešana LL parsētāju gadījumā) un middle-end savietojamības zaudēšana (kompilators vairs var nesaprast parsētāja izveidoto sintakses koku).

Piedāvātā sistēma ir balstīta uz sakarīgu gramatikas modifikāciju iespēju ierobežojumu, kā arī uz tipu sistēmas, kas ļaus pārliecināties, ka ieviestās modifikācijas ir korektas. Tas tiks nodrošināts ar specifisku makro šablonu sintaksi. Sistēmas galvenā īpašība ir tas, ka pēc gramatikas transformācijas izpildes modificētais izejas kods būs atpazīstams ar sākotnējo valodas gramatiku.

Aprakstāmā sistēma nav piesaistīta pie kādas konkrētas programmēšanas valodas. Tā tiek izstrādāta bāzējoties uz LL parsētāju ar noteiktām īpašībām. Perspektīvā tā varēs tikt lietota jebkurai valodai, kuras parsētājam piemīt līdzīgas īpašības un piedāvāt šai valodai pašmodificēšanas iespējas. Lietojot makro šablonus sakrišanas atrašanai kodā un vienkāršu programmēšanas valodu atrasto virkņu modifikācijai, šī sistēma ļaus izveidot jaunas gramatiskas konstrukcijas no jau eksistējošās programmēšanas valodas bāzes funkcionalitātes.

Šī darba mērķis ir izstrādāt prototipu, kas pierādīs iespēju izveidot strādājošu šablonu sakrišanas sistēmu, kas strādātu uz tokeniem. Prototipa izstrādes galvenais uzdevums ir atrast efektīvu veidu, ka apstrādāt makro prioritātes, ieejas un izejas no kontekstiem un sakrišanas konstatēšanu.

\fixme{Šeit būs paša prototipa apraksts, kad prototips būs tomēr gatavs.}

Šī dokumenta organizācija ir sekojoša. Nodaļa 2 ievieš un paskaidro galvenos jēdzienus, kas vajadzīgi, lai aprakstīt sistēmu. Nodaļa 3 apraksta problēmu un stāsta, kāpēc šī problēma ir aktuāla. Tā arī piedāvā citus risinājuma piemērus ar pamatojumiem, kāpēc tomēr ir vajadzīga cita pieeja. Nodaļa 4 vispārīgi apraksta izstrādājamo sistēmu un tās galvenās īpašības. Nodara 5, savukārt, apraksta prototipu, rīkus un algoritmus, kas tika lietoti izstrādē. Tā arī pamato, kāpēc daži jau gatavie risinājumi nav lietojami šajā gadījumā. 6. nodaļā ir aprakstītas prototipa iespējas, darba izstrādes rezultāti un parādītas testēšanas stratēģijas, bet 7. nodaļa apraksta darba secinājumus.