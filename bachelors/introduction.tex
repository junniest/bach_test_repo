\section{Ievads}
Mūsdienīgo programmēšanas valodu sintaksi nevar aprakstīt ar viennozīmīgām bezkonteksta gramatikām. Tāpēc vairākumam valodu parsētāji ir rakstīti ar rokām, uzmanīgi risinot gramatikas konfliktus. Un kaut arī eksistē parsētāju ģeneratori (piem. ANTLR), kas ar neierobežotu ieskatu kodā var atrisināt gramatikas likumu konfliktus, bieži vien to lietošanas sarežģītība ir salīdzināma ar paša parsētāja rakstīšanu.

Tomēr ievērojamas problēmas parādās tad, kad ir nepieciešams pievienot valodai jaunas konstrukcijas. Tas nozīmē, ka parsētāji ir jāparaksta tā, lai iekļautu jaunos likumus un atrisināt jaunus konfliktus. Ir pašsaprotami, ka gadījumos, kad valoda mainās radikāli, tas būs nepieciešams. Tomēr nelielu izmaiņu gadījumā, it īpaši tādu, kas atvieglo programmētāja darbu, varētu iztikt arī bez tā, atļaujot programmētajam pašam pielāgot valodas sintaksi savam vajadzībām.

Dotais darbs apskata sistēmu, kas ļauj dinamiski paplašināt programmēšanas valodu sintaksi un par pamata principu šīs sistēmas izstrādē ir ņemts pašmodificējamo gramatiku jēdziens. Dinamiski modificējamo gramatiku realizējamība ir aprakstīta dažādos rakstos, tomēr vispārīgā gadījumā šāda tipa gramatikas var nekontrolējami mainīties, izveidojot pavisam citu gramatiku sākotnējās gramatikas vietā. Līdz ar to neierobežotas modifikācijas var izraisīt neprognozējamas sekas.
Piedāvātā sistēma ir balstīta uz sakarīgu gramatikas modifikāciju iespēju ierobežojumu, kā arī uz tipu sistēmas, kas ļaus pārliecināties, ka modifikācijas ir korektas. Lai kontrolēt modifikāciju procesu sintaktiskās izmaiņas tiks apskatītas ka dinamisks priekšprocesēšanas variants. Sistēmas galvenā īpašība ir tas, ka pēc gramatikas transformācijas izpildes modificētais izejas kods būs garantēti atpazīstams ar sākotnējo gramatiku.

Aprakstāmās sistēmas ideja ir radusies programmēšanas valodas Eq kompilatora izstrādes laikā bet tā nav piesaistīta pie kādas programmēšanas valodas, bet gan pie konkrēta parsētāju tipa. Perspektīvā tā var tikt lietota jebkurai valodai, kuras parserim piemīt noteiktas īpašības un piedāvāt šai valodai pašmodificēšanas iespējas.

Lietojot nelielu funkcionālu valodu šī sistēma ļaus izveidot jaunas gramatiskas konstrukcijas no jau eksistējošās programmēšanas valodas bāzes funkcionalitātes.
\fixme{Šeit būs paša prototipa apraksts, kad prototips būs tomēr gatavs.}

Šī dokumenta organizācija ir sekojoša. Nodaļa 2 ievieš un paskaidro galvenos jēdzienus, kas vajadzīgi, lai aprakstīt sistēmu. Nodaļa 3 apraksta problēmu un stāsta, kāpēc šī problēma ir aktuāla. Tā arī piedāvā citus risinājuma piemērus ar pamatojumiem, kāpēc tomēr ir vajadzīga cita pieeja. Nodaļa 4 vispārīgi apraksta izstrādājamo sistēmu un tās galvenās īpašības. Nodara 5, savukārt, apraksta prototipu, rīkus un algoritmus, kas tika lietoti izstrādē. Tā arī pamato, kāpēc daži jau gatavie risinājumi nav lietojami šajā gadījumā. 6. nodaļā ir aprakstītas prototipa iespējas un darba izstrādes rezultēti, bet 7. nodaļa apraksta darba secinājumus. Tālāk darbā ir literatūras saraksts, atzinības un pielikumi (prototipa koda gabali un darba piemēri).