\section{\label{s:introduction}Ievads}

Mūsdienīgas programmēšanas valodas ātri attīstās, lai paliktu konkurētspējīgi un lietojami. Tomēr valodas sintakses izmaiņu ieviešanas process bieži aizņem daudz laika un pūļu. Tas ir tāpēc, ka vairākumam valodu eksistē vairāki kompilatori un piemīt mazas pašmodificēšanas spējas. Ir saprotams, ka ievērojami izmaiņu gadījumā būs nepieciešama kompilatoru pārstrāde. Taču nelielu sintaktisku izmaiņu gadījumā vairāku kompilatoru adaptēšana ir pārāk darbietilpīgs uzdevums.

Šīs darbs apskata iespēju iznest valodas sintakses izmaiņas uz valodas lietotāja līmeni, t.i. dot lietotājam iespēju modificēt valodas sintaksi rakstot programmas. Tas dos iespēju lietotājiem pašiem ieviest izmaiņas, kas ir nepieciešamas, lai paplašināt valodas iespējas.

Šādu iespēju dod dinamisku gramatiku pieejas pielietošana. Dinamiskas jeb adaptīvas gramatiku princips ir ļaut pievienot un dzēst valodas gramatikas likumus ar speciālām konstrukcijām. Tomēr šāda pieeja ir ļoti grūti kontrolējama, jo patvaļīgu gramatikas pārveidojumu gadījumā var tikt izveidota gramatika, ko nevarēs apstrādāt. Dinamisko gramatiku pielietošana arī prasa ļoti specifisku parsētāju izveidošanas pieeju, kas nozīmē ļoti nopietnu parsētāju pārstrādi, jo parasti parsētāji neatbalsta gramatikas modificēšanu programmas apstrādes laikā.

Dinamisko gramatiku principa lietošana \emph{as is} ļoti cieši piesaistītu to konkrētas valodas gramatikai.
Savukārt plānotā sistēma tiek projektēta ar iedomu, ka to varēs lietot dažādu valodu paplašināšanai ar minimālām nepieciešamām izmaiņām kompilatoros.

Aprakstāmā sistēma neļaus pa tiešo modificēt valodas gramatikas likumus, bet gan piedāvās iespēju veidot makro izteiksmes, kas ļaus paplašināt valodu konstrukciju kopu ar jau eksistējošo konstrukciju kombinācijām. Sistēma tiks izveidota ka parsētāja virsbūve, tātad nebūs nepieciešama kardināla parsētāja pārstrāde.

Piedāvātā sistēma ir balstīta uz sakarīgu gramatikas modifikāciju iespēju ierobežojumu, kā arī uz tipu sistēmas, kas ļaus pārliecināties, ka ieviestās modifikācijas ir korektas. Tas tiks nodrošināts ar specifisku makro šablonu sintaksi. Sistēmas galvenā īpašība ir tas, ka pēc gramatikas transformācijas izpildes modificētais izejas kods būs atpazīstams ar sākotnējo valodas gramatiku.

Šīs darbs ir fokusēts uz regulāro izteiksmju šablonu apstrādi un sakrišanu meklēšanu programmas tokenu virknēs. Tas tiek izstrādāts transformācijas sistēmas projekta ietvaros. Lai parādītu šādas apakšsistēmas iespējamību tika izstrādāts sakrišanu meklēšanas apakšsistēmas prototips. Prototipa izstrādes galvenais uzdevums ir atrast efektīvu veidu, ka apstrādāt makro prioritātes, ieejas un izejas no programmas tvērumiem un sakrišanas konstatēšanu.

Šī dokumenta organizācija ir sekojoša. Nodaļa~\ref{s:motivation} apraksta šī darba izstrādes pamatojumu un apskata gadījumus, kurus nevar apstrādāt lietojot jau eksistējošos rīkus. Nodaļa~\ref{s:system} apraksta piedāvāto sistēmu, tās īpašības un darbības principus. Nodaļa~\ref{s:prototype} stāsta par transformācijas sistēmas šablonu apstrādes apakšsistēmas prototipu, par algoritmiem, kas tika lietoti tā izstrādē. Nodaļa~\ref{s:results} apraksta prototipa testēšanas stratēģijas un izstrādes rezultātus, bet nodaļa~\ref{s:conclusions} piedāvā darba secinājumus un projekta turpināšanas perspektīvas.