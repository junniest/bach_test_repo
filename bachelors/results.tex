\section{\label{s:results}Rezultāti}

Šī darba ietvaros tika izstrādāts prototips sakrišanu meklēšanas sistēmai, kura koncepcija var kalpot ka bāze transformācijas sistēmas izstrādei. Šablonu apakšsistēmas atrastās virknes kalpos ka ieejas dati pārrakstīšanas apakšsistēmai. Tipu sistēma, savukārt, lietos regulāro izteiksmju minimizētos automātus lai pārbaudītu tipu sakritību makro izteiksmju ietvaros.

Regulārās izteiksmes tika izvēlētas ka saskarne transformāciju meklēšanai...
Darbs parāda, ka ir iespējams veidot regulāro izteiksmju šablonus un ar tiem attiecīgi apstrādāt ieejas daļiņu virkni. Gadījuma, ja valodas parsētājs būs modelēts aprakstītā veidā, būs iespējams veikt sakrišanu meklēšanu.
Prototipa izstrādes laikā tika identificētas iespējamās problēmas un izņēmumi, kuri ierobežo šablonu sistēmas darbu.

Darba gaitā tika sagatavots raksta melnraksts, kas apraksta sistēmas koncepciju un pielietošanu. Tālākā sistēmas izstrādes gaitā tas tiks pilnveidots un publicēts. Raksta melnraksts ir atrodams pielikumā $N$.