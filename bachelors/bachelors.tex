\documentclass[12pt, a4paper]{report}
\usepackage{polyglossia}
\setdefaultlanguage{latvian}
\setotherlanguages{english, russian}
\usepackage{bacdarbs}
\usepackage{mathtools}
\usepackage{fixlatvian}
\usepackage{mathtools}
\usepackage{graphicx}
\usepackage{listings}

\newcommand{\fixme}[1]{\vskip 5mm\noindent{\bf FIXME}: {\it #1}}

\usepackage[hmargin=2cm,vmargin=2cm,lmargin=3cm,rmargin=2cm]{geometry} 

\setlength{\parindent}{1cm}

\usepackage{setspace}
\onehalfspacing

\setmainfont{Times New Roman}

\setcounter{secnumdepth}{4} 

\renewcommand{\thesection}{\arabic{section}}

\newcommand{\kw}[2]{\par\noindent{\small{\em #1\/}: #2}}

\nosaukums{Ar regulārām izteiksmēm paplašinātu gramatiku dinamiska parsēšana.}
\darbatips{Bakalaura}
\autors{Jūlija Pečerska}
\vadvards{Guntis Arnicāns}
\vadinfo{LU docents}
\gads{2012}
\univer{Latvijas Universitāte}
\fakul{Datorikas}
\nod{Profesors}

\begin{document}

\titullapa

\begin{abstract}
Anotācijas teksts latviešu valodā

\kw{Atslēgvārdi}{Dinamiskās gramatikas, priekšprocesēšana, makro, regulārās izteiksmes, galīgi determinēti automāti, Python}
\end{abstract}

\begin{otherlanguage}{english}
\begin{abstract}
Abstract text in English

\kw{Keywords}{Dynamic grammars, preprocessing, macros, regular expressions, determinate finite automata, Python}
\end{abstract}
\end{otherlanguage}

\setcounter{tocdepth}{4}
\tableofcontents

\section {Termini un apzīmējumi}
Šeit būs aprakstīti termini, saīsinājumi un, ja būs nepieciešamība, apzīmējumi.

%\section{Ievads}
Mūsdienīgo programmēšanas valodu sintaksi nevar aprakstīt ar viennozīmīgām kontekstneatkarīgām gramatikām. Tāpēc vairākumam valodu parsētāji ir rakstīti ar rokām, uzmanīgi risinot gramatikas konfliktus. Un kaut arī eksistē parsētāju ģeneratori (piem. ANTLR), kas ar neierobežotu ieskatu kodā var atrisināt gramatikas likumu konfliktus, bieži vien to lietošanas sarežģītība ir salīdzināma ar paša parsētāja rakstīšanu.

Tomēr ievērojamas problēmas parādās tad, kad ir nepieciešams pievienot valodai jaunas konstrukcijas. Tas nozīmē, ka parsētāji ir jāparaksta tā, lai iekļautu jaunos likumus un atrisināt jaunus konfliktus. Ir pašsaprotami, ka gadījumos, kad valoda mainās radikāli, tas būs nepieciešams. Tomēr nelielu izmaiņu gadījumā, it īpaši tādu, kas atvieglo programmētāja darbu, varētu iztikt arī bez tā, atļaujot programmētajam pašam pielāgot valodas sintaksi savam vajadzībām.

Dotais darbs apskata sistēmu, kas ļauj dinamiski paplašināt programmēšanas valodu sintaksi un par pamata principu šīs sistēmas izstrādē ir ņemts pašmodificējamo gramatiku jēdziens. Dinamiski modificējamo gramatiku realizējamība ir aprakstīta dažādos rakstos, tomēr vispārīgā gadījumā šāda tipa gramatikas var nekontrolējami mainīties, izveidojot pavisam citu gramatiku sākotnējās gramatikas vietā. Līdz ar to neierobežotas modifikācijas var izraisīt neprognozējamas sekas.
Piedāvātā sistēma ir balstīta uz sakarīgu gramatikas modifikāciju iespēju ierobežojumu, kā arī uz tipu sistēmas, kas ļaus pārliecināties, ka modifikācijas ir korektas. Lai kontrolēt modifikāciju procesu sintaktiskās izmaiņas tiks apskatītas ka dinamisks priekšprocesēšanas variants. Sistēmas galvenā īpašība ir tas, ka pēc gramatikas transformācijas izpildes modificētais izejas kods būs atpazīstams ar sākotnējo gramatiku.

Aprakstāmās sistēmas ideja ir radusies programmēšanas valodas Eq kompilatora izstrādes laikā. Tomēr tā nav piesaistīta pie kādas programmēšanas valodas, bet gan pie konkrēta parsētāju tipa. Perspektīvā tā var tikt lietota jebkurai valodai, kuras parsētājam piemīt noteiktas īpašības un piedāvāt šai valodai pašmodificēšanas iespējas. Lietojot makro šablonus sakrišanas atrašanai kodā un vienkāršu funkcionālu valodu atrasto virkņu modifikācijai, šī sistēma ļaus izveidot jaunas gramatiskas konstrukcijas no jau eksistējošās programmēšanas valodas bāzes funkcionalitātes.

Šī darba mērķis ir izstrādāt prototipu, kas pierādīs iespēju izveidot strādājošu šablonu sakrišanas sistēmu, kas strādātu uz leksiskā analizatora izveidotiem tokeniem. Prototipa izstrādes galvenais uzdevums ir atrast efektīvu veidu, ka apstrādāt makro prioritātes, ieejas un izejas no kontekstiem un sakrišanas konstatēšanu. 
\fixme{Šeit būs paša prototipa apraksts, kad prototips būs tomēr gatavs.}

Šī dokumenta organizācija ir sekojoša. Nodaļa 2 ievieš un paskaidro galvenos jēdzienus, kas vajadzīgi, lai aprakstīt sistēmu. Nodaļa 3 apraksta problēmu un stāsta, kāpēc šī problēma ir aktuāla. Tā arī piedāvā citus risinājuma piemērus ar pamatojumiem, kāpēc tomēr ir vajadzīga cita pieeja. Nodaļa 4 vispārīgi apraksta izstrādājamo sistēmu un tās galvenās īpašības. Nodara 5, savukārt, apraksta prototipu, rīkus un algoritmus, kas tika lietoti izstrādē. Tā arī pamato, kāpēc daži jau gatavie risinājumi nav lietojami šajā gadījumā. 6. nodaļā ir aprakstītas prototipa iespējas, darba izstrādes rezultāti un parādītas testēšanas stratēģijas, bet 7. nodaļa apraksta darba secinājumus.

\section{Ievads}
Основная цель этой работы - показать, что возможно сделать грамматику динамически парсируемой, контролируя её модификации. Данная работа представляет теоретическую базу для системы, которая позволит на лету расширять синтаксис языка. Представленная система является надстройкой над парсером, следовательно для её использования нет необходимости вносить значительные изменения в его работу.

Система разрабатывается в составе группы студентов университетов разных стран. Автор данной работы участвовал в разработке идеи в целом, однако фокусом была разработка подсистемы, которая позволит находить появления совпадения с шаблонами в коде программы.

В основе предложенной системы лежит принцип динамических - самомодифицирующихся - грамматик. Проблема с динамическими грамматиками заключается в том, что они могут много всего и контролировать их очень сложно. Однако данная работа выдвигает некоторые ограничения для контроля корректности изменений, вносящихся в грамматику. Для этого будет использоваться система вывода типов, при помощи которой можно будет следить за тем, чтобы переписывалось нужное и в нужное.

Данная система не позволит добавлять и удалять правила грамматики на усмотрение программиста. Однако она даёт удобную возможность дополнять существующие конструкции новыми при помощи языка макросов. Это существенно упростит работу с языком, так как чтобы добавить удобные конструкции не  удет необходимости исправлять парсер языка.

Чтобы расширить возможности языка макросов, его система шаблонов будет расширена регулярными выражениями. Одна из особенностей описываемой системы заключается в том, что она работает не на тексте, а на токенах, за счёт чего и возможно устроить систему вывода типов. Далее список заматченных токенов из текста программы будет преобразовываться соответственно правилу трансформации, описанному в макросе.

В рамках данной работы будет описана модель предлагаемой системы. На данный момент система находится в стадии прототипирования, поэтому типовая система и система трансформаций будет только обозначена, а не рассмотрена полностью. Эта работа фокусируется на разработке эффективного прототипа для системы поиска совпадений в коде.

На момент написания работы есть готовый прототип системы поиска совпадений с шаблонами макросов. Основной принцип действия прототипа - создание одного детерминированного автомата для поиска совпадений по всем шаблонам по одному проходу по коду, по возможности за линейное время.

Далее организация работы будет следующей. Глава 2 рассказывает об актуальности проблемы, об общепринятом подходе к написанию парсеров языков и о возможностях динамических грамматик. Она также идентифицирует проблемы и сложности в использовании принципа самомодификации языка. Глава 3 рассказывает об основных принципах предлагаемой системы. Она даёт общий обзор языка модификаций и типовой системы, а также рассказывает о работе системы в рамках парсера языка. Глава 4 представляет прототип системы поиска совпадений с шаблонами, рассказывает об используемых алгоритмах и описывает причины выбора подходов. Там же описан подход к тестированию прототипа. В главе 5 описаны результаты данной работы, глава 6 представляет выводы.

\section{\label{sec:extendedbullshit}Uzdevuma pamatojums}

\subsection{Par programmēšanas valodu reprezentāciju}

Programmēšanas valodas ir jēdzienu sistēma, kas ļauj aprakstīt algoritmus. Šai sistēmai jābūt saprotamai programmētājam, tāpēc tiek meklēti veidi, kā tās sintaksi var nodefinēt formāli. Vienkāršs un intuitīvi saprotams rīks, kas der šīm uzdevumam, ir kontekstneatkarīgas gramatikas.

Kontekstneatkarīgas gramatikas piedāvāja N. Homskis, kas plānoja lietot tos lai ieveidotu reālo cilvēku valodu modeļus. Šinī jomā tās gandrīz netiek lietotas, jo dabiskās valodas ir pārāk sarežģītas un ar daudziem izņēmumiem no gramatikas likumiem. Tomēr šīs gramatikas tiek lietotas lai vispārināti aprakstītu programmēšanas valodu sintaksi. Programmēšanas valodas globālā līmenī nav kontekst-neatkarīgas, bet tomēr tās ir neatkarīgas lokāli. Kaut arī ne visas programmēšanas valodu īpašības var aprakstīt ar kontekstneatkarīgām gramatikām, tās ir ērti lietot lai parādīt valodas konstrukciju struktūru.

Kontekstneatkarīgas gramatikas sastāv no četrām daļām. Pirmā ir simbolu kopa, kas tiek saukta par termināliem simboliem. Otrā ir simbolu kopa, kas tiek saukta par netermināliem simboliem. Trešā ir gramatikas sākuma simbols, kas ir viens no neterminālu simbolu kopas. Un, beidzot, ceturtā gramatikas daļa ir pārrakstīšanas likumu kopa. Terminālie simboli ir gramatikas definētās valodas vārdnīca. No tiem tiks sastādīta valoda, kuru definē dotā gramatika. Neterminālie simboli, savukārt, var tikt apskatīti ka termināļu un termināļu virkņu klases.

Pārrakstīšanas likumi tiek pierakstīti izskatā \verb|A| $\rightarrow$ \verb|b|, kur \verb|A| ir viens no netermināliem simboliem, bet b ir neterminālu un terminālu simbolu virkne. Kad kāda likuma kreisē puse parādās apstādāmo simbolu rindā, rinda var tikt pārrakstīta aizvietojot kreisēs puses netermināli ar labo likuma daļu. \verb|A| $\xRightarrow[G]{*}$ \verb|b| parāda, ka \verb|A| var tikt pārveidots virknē \verb|b| lietojot gramatikas $G$ pārrakstīšanas likumus. Šādas secīgu pārveidojumu rinda tiek saukta par atvasinājumu. \cite{Shutt:AdaptiveGrammars}

Pārveidojumu rindas var tikt attēlotas koku veidā. Šīs koks skaidri parāda kā simboli no termināļu virknes tiek grupēti apakšvirknēs, katra no kurām pieder kādam no netermināliem simboliem. Bet vēl svarīgāk, šīs koks, ko sauc par parsēšanas koku, ir struktūra, kas reprezentē apstrādājamo programmu. Kompilatorā šāda struktūra veicina programmas izejas teksta translāciju uz izpildāmu kodu. Gadījumā, ja gramatikā eksistē divi parsēšanas koki vienam un tam pašam atvasinājumam, gramatika ir neviennozīmīga. Neviennozīmības padara gramatikas nelietojamas programmēšanas valodu aprakstam, jo šadā gadījumā kompilators nevarēs izsekot pareizu programmas struktūru. \cite{Hopcroft:IntroAutomataTheory}

Parsētāji ir programmas, kas izpilda programmas teksta pārstrādi parsēšanas kokā. 

Vairākums parsētāju mūsdienās aktuālākam valodām (piemēram C/C++) ir rakstīti manuāli. 
Parsētāju tipi - LR, LL, to trūkumi

\fixme{paskaidrot, no nozīmē reducēt gramatikas likumus}

Bet parsētāji strādā ar tokeniem, nevis ar leksēmām. 

Pirmā programmas kompilēšanas fāze ir leksiskā analīze jeb skanēšana. Tās laikā leksiskais analizators lasa ieejas simbolu virkni (programmas izejas tekstu) un veido jēdzīgas simbolu grupas, kas ir sauktas par leksēmām. Katrai leksēmai leksiskais analizators izveido speciālu objektu, kas tiek saukts par tokenu. Katram tokenam ir glabāts tokena tips, ko lieto parsētājs lai izveidotu programmas struktūru. Ja ir nepieciešams, tiek glabāta arī tokena vērtība, parasti tā ir norāde uz elementu simbolu tabulā, kurā glabājas informācija par tokenu - tips, nosaukums. Simbolu tabula ir nepieciešama tālākā kompilatora darbā lai paveiktu semantisko analīzi un koda ģenerāciju. Šajā darbā vienkāršības dēļ tiks uzskatīts, ka tokena vērtības ailītē glabāsies   leksēma, ko nolasīja analizators. Tālāk tokeni tiks apzimēti šādā veidā:

\begin{verbatim}
{token-type : token-value}
\end{verbatim}

Nolasīto tokenu virkne tiek padota parsētājam tālākai apstrādei.

Piemēram apskatīsim nelielu programmas izejas koda gabalu - \verb|sum = item + 5|. Šīs izejas kods var tikt sadalīts sekojošos tokenos:
\begin{enumerate}
\item \verb|sum| ir leksēma, kas tiks pārtulkota tokenā \verb|{id:sum}|. \verb|id| ir tokena klase, kas parāda, ka nolasītais tokens ir kaut kāds identifikātors. Tokena vērtībā nonāk identifikatora nosaukums \verb|sum|.
\item Piešķiršanas operators \verb|=| tiks pārveidots tokenā \verb|{=}| Šīm tokenam nav nepieciešams glabāt vērtību, tāpēc otrā tokena apraksta komponente ir izlaista. Lai atvieglotu tokenu virkņu uztveri šī darba ietvaros operatoru tokenu tipi tiks apzīmēti ar operatoru simboliem, kaut arī pareizāk būtu izveidot korektus tokena tipu nosaukumus, piemēram \verb|{assign}|.
\item Leksēma \verb|item| analoģiki \verb|sum| tiks pārtulkota tokenā \verb|{id:item}|.
\item Summas operators \verb|+| tiks pārtulkots tokenā \verb|{+}|.
\item Leksēma 5 tiks pārtulkota tokenā \verb|{int:5}|.
\end{enumerate}

Tātad izejas kods \verb|sum = item1 + 5| pēc leksiskās analīzes tiks pārveidots tokenu plūsmā \verb|{id:sum}, {=}, {id:item1}, {+}, {int:5}|. \cite{DragonBook}

Šī darbā arī tiek lietots jēdziens pseido-tokens. Pseido-tokens ir citu tokenu grupa, kas tiek aizvietota ar vienu objektu. Tas var tikt darīts, lai vienreiz noparsētu izteiksmi nevajadzētu apstrādāt vēlreiz. Tokenu aizvietošana ar pseido-tokeniem notiek gramatikas likumu reducēšanas brīdī. Kad, piemēram, tokenu virkne \verb|{id:a} '+' {id:b}| tiek atpazīta ka derīga izteiksme gramatikas ietvaros, tā var tikt aizvietota ar pseido-tokenu \verb|{expr:a + b}|.

\subsection{\label{subsec:dynamicgrammars}Par dinamiskām gramatikām}

Starp īpašībām, kuras nevar aprakstīt ar bezkonteksta gramatikām ir leksiskais tvērums (lexical scope) un statiskā tipizācija (static typing).

\fixme{Uzrakstīt!}
Dinamiskas vai adaptīvās gramatikas ir gramatiskais formālisms, kas ļauj modificēt gramatikas likumu kopu ar gramatikas rīkiem. \cite{Shutt:AdaptiveGrammars}

Dinamiskas gramatikas, kas tās ir. Fakti par to, ka tās jau ir pētītas un reāli implementējamas un lietojamas. Reālais labums no tām.
\fixme{No otras puses kāpēc tās daudz nepētīja un daudz reāli nelieto.} Tās vispārīgā gadījumā ir nekontrolējamas.

\subsubsection{Kā tos parasti vēlās lietot}
adding grammar rules when adding a variable - makes static semantics easier to control.

BUT: problems with scope, recursion and other stuff \cite{Christiansen:SurveyAdaptableGrammars}

%\section{\label{s:motivation}Problemātika un risinājuma koncepcija}

\subsection{Problemātika}

Viena no metodēm, kā varētu ļaut lietotājam paplašināt valodas sintaksi ir izveidot kross\--kom\-pi\-la\-to\-ru, kas transformētu jauno sintaksi tā, lai standarta kompilators to varētu atpazīt. Bet šīs metodes problēma ir tas, ka lielākas daļas moderno valodu sintaksi ir neiespējams noparsēt lietojot automātiskos rīkus. Zemāk ir piedāvāti daži piemēri gadījumiem no populāras valodas C, kad automātiskā parsēšana ir neiespējama.

\begin{enumerate}
\item
Valodā C lietotājs var nodefinēt patvaļīgu tipu lietojot konstrukciju \verb|typedef|. Šāda veida iespēja padara neiespējamu šādas izteiksmes apstrādi \verb|(x) + 5|, ja vien mēs neesam pārliecināti, kas ir \verb|x| - tips vai mainīgais. Ja \verb|x| ir tips, tad šī izteiksme pārveido izteiksmes \verb|+ 5| vērtību uz tipu \verb|x|. Ja \verb|x| ir mainīgais, tad šī izteiksme nozīmē vienkāršu mainīgā \verb|x| un vērtības \verb|5| saskaitīšanu. 
\item
C valodā eksistē operators postfikss operators \verb|++|, kas palielina argumentu par vienu vienību. Pieņemsim, ka ir iespēja paplašināt C valodas sintaksi ar infiksu operatoru \verb|++|, kas savieno divus masīvus un pierakstīt konstanšu masīvus \verb|[1, 2, 3]| veidā. Tad izteiksme \verb|a ++ [1]| būtu nepārsējama, jo eksistē vismaz divi to interpretācijas veidi. Tas varētu tikt saprasts ka postfiksā operatora \verb|++| pielietošana mainīgam \verb|a| un tad \verb|a| indeksēšana ar \verb|[1]|. Vai arī tas varētu būt divu masīvu \verb|a| un \verb|[1]| konkatenācija.
\end{enumerate}

%Dažreiz arī programmatūras koda dalīšana pa tokeniem ir atkarīga no šī koda konteksta, kas padara ne tikai parsēšanas procesu, bet arī leksēšanas procesu neautomatizējamu.
%
%\begin{enumerate}
%\item
%Valoda C++ ļauj lietotājam izveidot ligzdveida veidnes, piemēram, šādas \\ \verb|template <typename foo, list <int>>|. Šajā gadījumā simboli \verb|>>| aizver divas atvērtās grupas pēc kārtas. Lai tas tiešām būtu atpazīts, ka grupu aizvēršana, lekserim jāzin simbolu konteksts, jo parasti simbolu kombinācija \verb|>>| nozīmē operāciju pārbīdei pa labi.
%\item
%Gadījumā, ja lietotājam ir dota iespēja definēt savus operatorus, ieviešot operatoru, kas pārklāj eksistējošos, ir jāmaina leksēšanas likumus. Piemēram, ja lietotājs definē unāru operatoru \verb|+-|, tad izteiksmei \verb|+-5| ir jābūt saprastai ka \verb|(+-, 5)|, nevis ka \verb|(+, -, 5)|.
%\end{enumerate}

Apskatītie piemēri dod iespēju secināt, ka automātisku parsētāju ģeneratoru lietošana var būt tik pat sarežģīta, ka parsētāja rakstīšana ar rokām. Izrādās, ka daudzām eksistējošām valodām parsētāji arī tiek rakstīti manuāli (piemēram C/C++/ObjectiveC kompilators GNU GCC \cite{GCC} - http://gcc.gnu.org/wiki/New\_C\_Parser). Tas nozīmē, ka kross-kompilatoru visticamāk būs jāraksta manuāli, risinot eksistējošās gramatikas konfliktus, un oriģinālvalodas ievērojamu izmaiņu gadījumā būs jāpastrādā abi kompilatori, kas nozīmē divreiz vairāk darba. 

Šīs darbs apskata pieeju, kas lietos eksistējošo valodas parsētāju ka pamatu savam darbam un piedāvās likumu kopu, kas ļaus  modificēt valodas gramatiku parsēšanas laikā. Taču patvaļīgas gramatikas izmaiņas var novest pie nekontrolējamas valodas evolūcijas. Tāpēc aprakstāmā pieejā tiek piedāvātas ierobežotas izmaņu iespējas, kas tiks kontrolētas ar speciāli izveidotas tipu sistēmas palīdzību.

Sistēma ļaus ieviest jaunas konstrukcijas, konstruējot tās no jau eksistējošām valodas vienībām. Tā transformēs programmas gabalus attiecīgi pierakstītiem likumiem tā, lai valodas sākotnējā gramatika būtu tiem pielietojama. Šīs transformācijas korektumu nodrošinās tipu pārbaudes sistēma.

Ļoti līdzīgu uzdevumu, izņemot tikai transformāciju korektuma pārbaudi, pilda arī vispārējie priekšprocesori. Varbūt ir iespējams izveidot minēto sistēmu lietojot kādu no eksistējošām priekšprocesēšanas sistēmām, pievienojot tai kādas korektuma pārbaudes?

Jebkura priekšprocesora viens no galveniem mērķiem ir vienas elementu virknes aizvietošana ar citu. Virknes vienība var būt atšķirīga atkarībā no priekšprocesora, bet parasti šī vienība ir kādu vienas klases rakstzīmju kopa. Klašu daudzums parasti ir fiksēts (skaitlis, burts, tukšums, u.t.t.). Dažreiz zīmes piederība pie klases ir statiska, ka C priekšprocesorā, dažreiz ir dinamiska, ka, piemēram, \TeX{}, kur par atdalītāju var nodefinēt jebkādu specifisku simbolu. Tad apstrāde ir šo virkņu aizvietošana ar citām izveidotām virknēm.

Svarīgākā problēma šādai teksta apstrādes pieejai ir tas, ka tai neiespējams pārbaudīt korektumu. Apskatīsim sekojošu C makro piemēru:
\begin{verbatim}
#define foo(x, y) x y
\end{verbatim}

Pirmkārt, šādam makro nav iespējams statiski izsecināt rezultātu, jo kaut arī \verb|foo(5, 6)| tiks pārveidots par \verb|5 6|, gan \verb|foo(, 5)|, gan \verb|foo(5, )| tiks pārveidots par \verb|5|. Otrkārt, tā kā komats ir makro daļa, nav iespējams kā pirmo makro argumentu padot virkni \verb|5, 6|. To var izdarīt tikai ievietojot argumentu iekavās, tad \verb|foo((5, 6), 7)|, kas tiks pārveidots par \verb|(5, 6) 7|.

Gadījumā, ja ir nepieciešams saplacināt sarakstu, ir nepieciešams izveidot 2 makro, piemēram:
\begin{verbatim}
#define first(x, y) x
#define bar(x, y) first x y
\end{verbatim}

Tomēr aprakstītais makro strādā tikai gadījumos, kad argumentiem ir pareizs tips. Piemēram, izteiksme \verb|bar((5, 6), x)| tiks pārveidota par \verb|5, x|. Bet izteiksme \verb|bar(5, 6)| tiks pārveidota par \verb|first 5 6|, kaut arī tai vajadzētu izraisīt kļūdu.

Var redzēt, ka vispārīgā gadījumā nav iespējams statiski izveidot nekādus secinājumus, tā kā makro rezultāts ir atkarīgs no argumentiem, kuriem tas ir pielietots. Bet patiesībā arī nekādus dinamiskus secinājumus nav iespējams izveidot, jo apstrādājot tekstu neeksistē korektuma kritēriji.

Tomēr pat neņemot vērā korektuma pierādījumu neiespējamību, makro sistēmai trūkst iespēju, lai izveidot jaunas valodas konstrukcijas. Piemēram, būtu dabiski reprezentēt skaitļa moduli ar pierakstu \verb/|a|/. C priekšprocesors, savukārt, ļauj veidot tikai prefiksa formas funkciju makro un konstanšu makro. Jā arī kāda makro sistēma ļautu izveidot minēto pierakstu, parādītos problēmas gadījumos, kad vienam un tam pašam simbolam eksistē dažas nozīmes, piemēram, ar pierakstu \verb/| (a | b) |/, kam jābūt pārveidotam uz \verb/abs(a | b)/.

Apskatītie piemēri parāda, ka ne kross-kompilēšana, ne programmas teksta priekšprocesēšana vispārpieņemtā veidā neder vēlamā rezultāta sasniegšanai. Par šīs problēmas risinājumu varētu kļūt transformācijas sistēma, kas apstrādā nevis programmas tekstu, bet gan programmas tokenu un pseido-tokenu\footnote{Pseido-tokens ir citu tokenu grupa, kas tiek aizvietota ar vienu objektu. Tas var tikt darīts, lai vienreiz noparsētu izteiksmi nevajadzētu apstrādāt vēlreiz. Tokenu aizvietošana ar pseido-tokeniem notiek gramatikas likumu reducēšanas brīdī. Kad, piemēram, tokenu virkne \texttt{{id:a} '+' {id:b}} tiek atpazīta ka derīga izteiksme gramatikas ietvaros, tā var tikt aizvietota ar pseido-tokenu \texttt{{expr:a + b}}.} virknes. 

Lai padarītu transformācijas sistēmu vairāk spēcīgu un ļautu atpazīt vispārīgākas virknes, tās makro šabloni tiks paplašināti ar regulāro izteiksmju elementiem,

 joprojām saglabājot iespēju kontrolēt transformāciju korektumu.

Šādā sistēma dos iespēju paplašināt valodu lietotāja līmenī, nevis kompilatora līmenī. Tas arī dos lielāku brīvību izmaiņu izstrādē, jo lietotājiem būs iespēja veidot makro bibliotēkas ar jaunām iespējām un izplatīt tās. Tā varēs būt pielāgota dažādām valodām, jo tā strādās ārpus paplašināmās valodas gramatikas.

\subsection{Idejas rašanās - valoda Eq}
Šīs makro transformācijas sistēmas ideja ir radusies valodas Eq\footnote{Pirmkods atrodams tiešsaistē - https://github.com/zayac/eq} kompilatora izstrādes gaitā, ar ko nodarbojas zinātniskā grupa, kuras sastāvā ir cilvēki no Compiler Technology \& Computer Architecture Group, University of Hertfordshire (Hertfordshire, England), Heriot-Watt University (Edinburgh, Scotland) un Moscow Institute of Physics and Technology (Dolgoprudny, Russia). Šīs valodas sintakse bāzējas uz \LaTeX{} teksta procesora sintakses, kas ir standarts priekš zinātniskām publikācijām. Korekti uzrakstīta Eq valodas programma var tikt interpretēta ar \LaTeX{} procesoru. Perspektīvā Eq programma varēs tikt kompilēta un izpildīta uz vairākuma mūsdienīgo arhitektūru. 

Lai atvieglotu izstrādi valodā Eq tika nolemts izveidot makro sistēmu, kas ļaus pielāgot sintaksi programmētāja vajadzībām. Tomēr bez kaut kādas šablonu sistēmas makro iespējas ir ļoti ierobežotas. Tāpēc tika izlemts lietot šablonus ar minimālu regulāro izteiksmju sintaksi, kas dod brīvību sakritību aprakstīšanai. 

Lai izveidot jaunas konstrukcijas no tokeniem, kas tika atpazīti ir nepieciešami kaut kādi rīki, lai apstrādāt tokenus, kas tika atpazīti, ka sakrītoši ar vienu no šabloniem. 

 tāpēc tika izlemts lietot regulāro izteiksmju šablonus.  kas dod brīvību sakrišanas meklēšanas mehānismam. Tālāk, lai apstrādāt regulārās izteiksmes sakrātos tokenus, tika nolemts izveidot vienkāršu funkcionālu valodu, kas ļaus pārstrādāt pseido-tokenu virknes atkarībā no programmētāja izveidotiem šabloniem.

Bet kaut arī ideja un pieejas izstrāde sākās ar valodu Eq, tā nav piesaistīta tieši šai valodai. Visspēcīgāka šīs sistēmas īpašība ir tas, ka tā ir universāla un var tikt pielietota jebkādam parsētājam kas atbilst dažiem nosacījumiem. Par parsētājiem nepieciešamām īpašībām tiks runāts apakšnodaļā~\ref{sbs:sys_parserqualities}.

\subsection{\label{sbs:sys_approach}Sistēmas koncepcija}

Aprakstāmā transformāciju sistēma tiek projektēta ņemot vērā divu eksistējošo pieeju pieredzi. Pirmā no pieejam ir programmas koda priekšprocesēšana - koda makro ierakstu apstrāde pirms parsētāja darba sākšanos. Priekšprocesors parasti aizvieto kādas konstrukcijas ar citām noteikti definētam konstrukcijām. Otrā pieeja ir adaptīvās gramatikas - gramatikas, kas ļauj programmas kodam modificēt savu apstrādes gramatiku. Abas pieejas dod ļoti spēcīgus rīkus programmēšanas valodu izstrādē. Tomēr abām šīm pieejām ir savas problēmas un trūkumi, kurus šīs sistēmas projektēšanā mēģināja atrisināt. 

Pirmā problēma, no kuras šī sistēmas izstrādē mēģināja izvairīties ir nekorekta simbolu ar divām nozīmēm apstrāde. Pieņemsim, ka mēs gribam funkciju \verb|abs (x)| apzīmēt ka \verb/| x |/. Apskatīsim izteiksmi \verb/| (a | b) + c|/, kurai vajadzētu tikt pārveidotai par \verb/abs ((a | b) + c)/. Gadījumā, ja transformācijas sistēma apstrādātu tekstu, tā nebūs spējīga pārveidot šādu konstrukciju. Tiks apstrādāti pirmie divi \verb/|/simboli, no \verb/| (a |/, izveidojot nekorektu konstrukciju \verb/abs( (a) b ) + c) |/. Šīs problēmas izvēlētais risinājums ir aprakstīts apakšnodaļā~\ref{sbsbs:sys_texttransform}.

Otrā problēma ir dinamisku gramatiku nekontrolējamība. Dinamiskas gramatikas ir ļoti spēcīgs rīks, kas mūsdienās gandrīz netiek lietots. Tas tā ir tādēļ, ka dodot iespēju lietotājam patvaļīgi pievienot un dzēst gramatikas likumus, tiek zaudēta iespēja kontrolēt izmaiņu korektumu. Robežgadījums varētu būt tad, kad sākotnējā gramatika tiek pilnībā aizvietota ar citu gramatiku. Tad, kaut arī sākotnējā gramatika bija derīga parsēšanai ar eksistējošo algoritmu, jaunai gramatikai var piemīt īpašības, kas neļaus to apstrādāt (piemēram, kreisā rekursija LL parsētāju gadījumā). Šīs problēmas izvēlētais risinājums ir apskatīts apakšnodaļā~\ref{sbsbs:sys_dynamicgrammars}.

Trešā problēma ir tas, ka nav iespējams vienkārši ieviest gramatikas modifikācijas jau eksistējošā valodas parsētāja, ja vien tā arhitektūra no sākuma atbalsta gramatikas izmaiņas. Bet tā kā parasti šāda iespēja netiek iekļauta, visticamāk būs nepieciešamas nopietnas parsētāja adaptācijas. Aprakstāmā sistēma, savukārt, mēģina dot iespēju paplašināt valodas gramatiku bāzējoties uz vienu no plaši lietojamam parsētāju arhitektūrām. Kā tas tiks darīts ir aprakstīts apakšnodaļā~\ref{sbsbs:sys_parsermodifications}.

%dinamiskumu gramatikas līmenī nav iespējams ieviest jau eksistējošā valodas parsētājā bez ievērojamām izmaiņām. Parsētāja adaptācija, kas ļaus lietotājam modificēt gramatiku, ir sarežģīts darbs, kas dažreiz prasīs parsētāja arhitektūras izmaiņas. Iespēja pievienot valodai dinamiskumu nepārstrādājot parsētāju ir aprakstīta apakšnodaļā~\ref{sbsbs:sys_parsermodifications}.

\subsubsection{\label{sbsbs:sys_texttransform}Priekšprocesori}

Ir dažādas pieejas programmu pirmkoda priekšprocesēšanai. Visvairāk izplatītas no tām ir divas pieejas. Viena no pieejām ir sintaktiskā pieeja - sintaktiskie priekšprocesori tiek palaisti pēc parsētāja darbības un apstrādā sintaktiskos kokus, ko uzbūvē parsētājs. Dēļ aprakstāmās sistēmas īpašībām šajā darbā netiks apskatīti sintaktiskie priekšprocesori, jo līdz sistēmas darba izpildei, gadījumā, ja tika ieviestas kaut kādas transformācijas, parsētājs nevar uzbūvēt pareizu sintaktisko koku. Otra no pieejām ir leksiskā, leksiskie priekšprocesori tiek palaisti pirms pirmkoda parsēšanas un tiek nav zināšanu par apstrādājamas valodas sintaksi (piem. C/C++ priekšprocesors). 

Leksiskie priekšprocesori pēc savām īpašībām ir tuvi aprakstāmai sistēmai. Ar makro valodu palīdzību tiem tiek uzdoti koda pārrakstīšanas likumi, un kods tiek pārveidots attiecīgi tām. Bet leksisko priekšprocesoru vislielākais trūkums ir tas, ka tie apstrādā tekstu pa simboliem neievērojot izteiksmju un konstrukciju struktūru. Apskatīsim jau minētu piemēru \verb/abs ((a | b) + c)/. Ar tādu makro sistēmu, kas neievēro koda struktūru, tātad neievēro to, ka patiesībā \verb/(a | b) + c/ ir atomāra konstrukcija izteiksmē, šādu koda gabalu pareizi apstrādāt nevarēs\footnote{C/C++ priekšprocesors vispār neatļaus tādu konstrukciju izveidot, kaut arī šāds pieraksts ir diezgan loģisks no matemātiķu skatu punkta. C/C++ makro sistēma ļauj veidot tikai makro konstantes un prefiksa formas funkcijas.}.

Priekšprocesoru var iemācīt apstrādāt šāda veida konstrukcijas un atpazīt tos, ka atomārās izteiksmes. Bet tas nozīmēs, ka priekšprocesoram būs jāzina apstrādājamas valodas gramatika, kas neatbilst priekšprocesora lomai kompilēšanas procesā un nozīmē ka būs divreiz jāimplementē sintakses atpazīšana.

Lai izvairīties no šīs problēmas tika izvēlēts apstrādāt nevis programmas tekstu, bet gan programmas daļiņu un pseido-daļiņu virkni, ko daļēji jau apstrādāja parsētājs. Tas nozīmē, ka makro šablonu sintakse būs bāzēta uz daļiņu aprakstiem, nevis uz tekstuālām izteiksmēm. Piemēram, ja ir nepieciešams izveidot šablonu, kas pārveidos funkcijas ar nosaukumu \verb|bar| par funkcijām ar nosaukumu \verb|foo|, makro šablonā būs jāieraksta daļiņa ar tipu \verb|id| un vērtību \verb|bar| - \verb|{id:bar}|. Tad, kad programmas daļiņu virknē tiks atrasta daļiņa ar šādu tipu un vērtību, tā tiks aizvietota ar citu daļiņu \verb|{id:foo}|. Šāda pieeja dod iespēju programmas tekstā meklēt specifiskus daļiņu tipus, nevis specifiskas teksta daļas. Tas dod iespēju meklēt, piemēram, jebkādus identifikatorus, šablonā norādot daļiņu \verb|{id}| bez vērtības. 

Makro šablonu sistēma arī ļauj lietot pseido-daļiņas savu šablonu aprakstos, t.i. ļauj lietot daļiņu \verb|{expr}|. Pseido-daļiņa \verb|{expr}| dotajā sintaksē apzīmē kādu izteiksmi. Tā tiek saukta par pseido-daļiņu tādēļ, ka programmas tekstā tā ir reprezentēta ar citu daļiņu virkni, kaut arī patiesībā tā ir atomāra vienība.

Tā kā aprakstāmā sistēma tiek izstrādāta tā, lai tā varētu darboties nezinot neko par valodas gramatiku, tā nezina, kādas varētu būt izteiksmes dotajā valodā. Tad, kad transformācijas sistēmu makro ir atrodama šāda pseido-daļiņa, sistēma nemēģina pati izsecināt, vai šāda daļiņa ir nolasāma no parsētās virknes. Lietojot speciālu saskarni tā "pajautā" parsētājam, vai sagaidītā pseido-daļiņa ir atrodama sākot no dotās daļiņas virknē. Gadījumā, ja parsētājs var izveidot izteiksmi, tas paziņo par to, un transformācijas sistēma turpina virknes apstrādi. Piemēram, konstrukciju \verb/(a | b) + c/ parsētājs atpazīs ka pseido-daļiņu \verb|{expr}|.

Gadījumā, ja kāds no šabloniem tiks atpazīts ieejas virknē, atpazītā apakšvirkne tiks pārveidota citā apakšvirknē, kas aizvietos iepriekšējo. Tālāk aizvietotā virkne tiks apstrādāta ar sākotnējo valodas gramatiku.

Otrā leksiskā tipa priekšprocesoru problēma ir tas, ka tie strādā ārpus programmas tvērumiem. Tas nozīmē, ka tvēruma sākuma daļiņa (piemēram, figūriekava, C/C++, Java un citu valodu gadījumā) tiek uzskatīts par parastu tekstu un var tikt pārrakstīts. Loģiskāk būtu, ja konkrētā tvērumā definēti makro tiktu mantoti līdzīgi ka mainīgie, kas nozīmē, ka šabloni, kas ir specifiski tvērumam, būtu ar lielāku prioritāti ka tie, kas definēti vispārīgākā tvērumā. 

Sistēmas sakrišanas meklēšanas mehānisms tiks izstrādāts ņemot vērā programmas tvēruma maiņu. Tātad šabloni, kuri tiek ieviesti konkrētā tvērumā, strādās tikai tā ietvaros.

\subsubsection{\label{sbsbs:sys_dynamicgrammars}Dinamiskas gramatikas}

Sistēma adaptē dinamisko gramatiku principu, ieviešot izmaiņu kontroli. Dinamiskas gramatikas vispārīgā gadījumā nekontrolē ieviestās izmaiņas, kas var sabojāt valodas parsējamību.

Transformācijas sistēmas makro nepievieno sistēmai jaunus gramatikas likumus tajā nozīmē, ka tie modificē gramatiku. Tie pievieno jaunu daļiņu virknes pārrakstīšanas likumu, kas tiek izpildīts, kad tiek atrasta sakrišana ar makro specificētu šablonu. Tātad jaunās konstrukcijas daļiņu virknē ir atpazītas un pārrakstītas uz konstrukcijām, kuras jau ir zināmas parsētājam. Makro sistēma nedos iespēju dzēst eksistējošos likumus no valodas gramatikas.

Lai nodrošinātu gramatikas likumu pievienošanas kontroli, tiek ieviestas dažas tipu specifikācijas. Katrs no makro attiecas uz kādu konkrētu gramatikas produkciju, un nevar tikt pārbaudīts citā parsēšanas brīdī. Katram makro pārrakstīšanas likumam arī ir specificēts tips, kas tiek lietots lai statiski pārliecināties par to, ka transformācija ir korekta dotā tipa ietvaros. Tipu sistēma detalizētāk ir aprakstīta apakšnodaļā~\ref{sbsbs:sys_typesystem}.

\subsubsection{\label{sbsbs:sys_parsermodifications}Parsētāja modifikācijas}

Vēl viena svarīga dinamisku gramatiku īpašība ir tas, ka to lietošanai ir nepieciešams speciāli izstrādāts parsētājs, kas atļauj savu parsēšanas tabulu modifikācijas. Ir jāeksistē iespējai pievienot un dzēst attiecīgus gramatikas likumus, kā arī parsētājam jāprot pārbūvēt parsēšanas tabulas atbilstoši ieviestām izmaiņām.

Šī dinamisku gramatiku īpašība ļoti ierobežo iespēju lietot tos jau eksistējošo valodu paplašināšanai. Šāda veida papildināšana vairākumā gadījumu nozīmēs jauna parsētāja izveidošanu. Bet zinot, ka mūsdienīgām valodām parasti eksistē vairāki kompilatori, kurus izstrādā dažādi cilvēki, šādu izmaiņu ieviešana globālajā līmenī var kļūt neiespējama.

Aprakstāmā sistēma, savukārt, ir domāta ka palīgrīks parsētājam. Parsētāju izmaiņas, kas būs nepieciešamas integrēšanai ar sistēmu ir minimālas. Tas dos iespēju lietot to jebkuram parsētājam, kas atbilst dažiem apstrādes nosacījumiem, kas ir apskatīti apakšnodaļā~\ref{sbs:sys_parserqualities}.

Sistēma dos iespēju papildināt valodas sintaksi bez nepieciešamības pilnībā pārstrādāt valodas parsētājus. Tā strādās ārpus valodas gramatikas. Pirms katras produkcijas apstrādes ar standartu valodas gramatiku likumiem tiek izsaukta transformācijas sistēma. 

\section{Transformāciju sistēma}
Ka var redzēt no nodaļas~\ref{subsec:dynamicgrammars}, pašmodificējošās gramatikas ir diezgan sarežģīts rīks, kas kaut arī ir ļoti lietderīgs, mūsdienās gandrīz netiek lietots. Tas netiek lietots savas sarežģītības dēļ un dēļ tā, ka vispārīgā gadījumā pašmodificējošo gramatiku ir ļoti grūti kontrolēt. Ļaujot neierobežoti modificēt gramatiku mēs varam nonākt pie gadījuma, kad sākotnējā gramatika tiek izmesta ārā, bet tās vietā parādās cita, pilnīgi jauna. Neierobežotas modifikācijas iespējas var arī ieviest tādas gramatikas īpašības, kas neļaus parsētājam pareizi darboties (piemēram kreisā rekursija LL parsētāju gadījumā). Tātad vispārīgā gadījumā jaunās gramatikas pareizību nevar garantēt.

Šīs darbs piedāvā uzbūves principus sistēmai, kura dod iespēju programmētājam dinamiski paplašināt valodas iespējas ar makro valodas palīdzību. Šī makro valoda ļaus izveidot jaunas valodas konstrukcijas no jau eksistējošām vienībām ar regulāro izteiksmju un nelielas funkcionālās valodas palīdzību. Parsētāja darba laikā makro sastapšanas reizes tiks pārrakstītas uz kodu ar attiecīgu struktūru, kas var tikt atpazīti ar valodas sākotnējo gramatiku. Tātad šī sistēma ļaus modificēt gramatiku nebojājot jau eksistējošo sintaksi. Nekādas pavisam jaunas konstrukcijas šī makro sistēma nejaus izveidot, lai paliktu savietojamība ar sākotnējo gramatiku, tomēr tā ļaus atvieglot programmētāja darbu dodot iespēju aizstāt kodā sarežģītas konstrukcijas ar vienkāršākām. 

Tālāk aprakstāmā sistēma tiks saukta par transformāciju sistēmu. Šī nodaļa dos vispārīgu ieskatu transformācijas sistēmas uzbūvē, darba gaitā, aprakstīs transformācijas sistēmas likumu sintaksi un parādīs iespēju pierādīt transformācijas pareizību.

\subsection{Idejas rašanās - valoda Eq}
Šīs makro transformācijas sistēmas ideja ir radusies valodas Eq (atrodams tiešsaistē - https://github.com/zayac/eq) izstrādes gaitā, kurā piedalās cilvēku grupa no Heriot-Watt University (Riccarton, Edinburgh) un Moscow Institute of Physics and Technology (Dolgoprudny, Russia). Šīs valodas sintakse bāzējas uz \LaTeX{} teksta procesora sintakses, kas ir standarts priekš zinātniskām publikācijām. Konsekventi programma, kas rakstīta valodā Eq ir korekti interpretējama ar \LaTeX{} procesoru. Tajā pašā laikā Eq programma varēs tikt kompilēta vairākumam mūsdienīgu arhitektūru. 

Tā kā \LaTeX{} sintakse ir viegi paplašināma, arī Eq valodas sintaksei izlēma piedāvāt paplašināšanas iespējas. Makro sintakse bez kaut kādas šablonu sistēmas ir bezjēdzīga, jo ir ļoti ierobežota. Tāpēc tika izvēlēta regulāro izteiksmju šablonu sistēma, kas dod brīvību sakrišanas meklēšanas mehānismam. Tālāk lai apstrādāt regulārās izteiksmes apstrādātos tokenus tika nolemts izveidot vienkāršu funkcionālu valodu, kas ļaus pārstrādāt tokenu virknes atkarībā no programmētāja izveidotiem šabloniem.

Bet kaut arī ideja un pieejas izstrāde sākās ar valodu Eq, tā nav piesaistīta tieši šai valodai. Visspēcīgāka šīs sistēmas īpašība ir tas, ka tā ir universāla un var tikt pielietota jebkādam parsētājam kas atbilst dažiem nosacījumiem. Par parsētājiem nepieciešamām īpašībām tiks runāts apakšnodaļā~\ref{subsec:parserqualities}.

\subsection{\label{subsec:parserqualities}Parsētāji}
Lai parsētājs varētu kļūt par bāzi izstrādājamai transformāciju sistēmai, tam jābūt izstrādātam ar rekursīvas nokāpšanas algoritmiem LL(k) vai LL(*). LL ir viena no intuitīvi saprotamākām parsētāju rakstīšanas pieejam, kas ar lejupejošo procesu apstrādā programmatūras tekstu. LL parsētājiem nav nepieciešams atsevišķs darbs parsēšanas tabulas izveidošanā, tātad parsēšanas process ir vairāk saprotams cilvēkam un vienkāršāk realizējams, samazinot kļūdu varbūtību. Gadījumā, ka gramatika ir labi rakstīta (k, simbolu skaits ieskatam uz priekšu, ir mazs) LL parsētāja darba ātruma atkarībā no tokenu daudzuma var tuvoties lineārai. \cite{Lewis:LLParsers}

Tā kā transformāciju sistēma tiek veidota ka paplašinājums parsētājam, tā prasa lai parsētājs uzvedās zināmā veidā. Zemāk tiks aprakstītas īpašības, kurām jāatbilst parsētājam, lai uz tā veiksmīgi varētu uzbūvēt aprakstāmo sistēmu.
\begin{description}
\item[Tokenu virkne]
Parsētājam jāprot aplūkot tokenu virkni ka abpusēji saistītu sarakstu, lai eksistētu iespēja to apstaigāt abos virzienos. Tam arī jādod iespēju aizvietot kaut kādu tokenu virkni ar jaunu un ļaut uzsākt apstrādi no jaunās virknes sākuma.
\item[Pseido-tokeni]
Parsētāji parasti pielieto (reducē) gramatikas likumus ielasot tokenus no ieejas virknes. Pseido-tokens, savukārt, konceptuāli ir atomārs ieejas plūsmas elements, bet īstenībā attēlo jau reducētu kaut kādu valodas gramatikas likumu. Viens no pseido-tokeniem, piemēram, ir tokens izteiksme - \verb|<expr>|, kas var sastāvēt no daudziem dažādiem tokeniem (piem. \verb|(a+b*c)+d|). Tas nav viens tokens, bet tā ir tokenu grupa, ko atpazīst parsētājs un kas var tikt attēlots ka atomāra vienība.
\item[Vadīšanas funkcijas]
Pirmkārt, mēs prasam, lai katra gramatikas produkcija tiktu reprezentēta ar vadīšanas funkciju (\emph{handle-function}). Ir svarīgi atzīmēt, ka šim funkcijām būs blakus efekti, tāpēc to izsaukšanas kārtība ir svarīga. Šo funkciju signatūrai jāizskatās šādi: \verb|Parser| $\to$ \verb/(AST|Error)/, tas ir, funkcija ieejā iegūst parsētāja objektu un izejā atgriež abstraktā sintakses koka (Abstract Syntax Tree) mezglu vai arī kļūdu. Šīs funkcijas atkārto gramatikas struktūru, tas ir ja gramatikas produkcija A ir atkarīga no produkcijas B, A-vadīšanas funkcija izsauks B-vadīšanas funkciju. 

Katra no šādām funkcijām pēc nepieciešamības implementē arī kļūdu apstrādi un risina konfliktus starp produkcijām ar valodas apraksta palīdzību.
\item[Piederības funkcijas]
Katrai vadīšanas funkcijai pārī ir piekārtota funkcija-predikāts. Šīs predikāts pārbauda, vai tā vietā tokenu virknē, uz kuru dotajā brīdī norāda parsētājs, atbilst parsētam gramatikas likumam. Šādas piederības funkcijas (\emph{is-function}) izpilde nemaina parsētāja stāvokli. 

\item[Sakrišanas funkcijas]
Katras vadīšanas funkcijas darbības sākumā tiek izsaukta tā sauktā sakrišanas funkcija (\emph{match-function}). Sakrišanas funkcija ir transformācijas sistēmas saskarne ar signatūru \verb|(Parser, Production)| $\to$ \verb|Parser|. Tā pārbauda, vai tā vieta tokenu virknē, uz kuru rāda parsētājs, ir derīga kaut kādai transformācijai dotās produkcijas ietvaros. Ja pārbaude ir veiksmīga, funkcija izpilda sakrītošās virknes substitūciju ar jaunu virkni un parsētāja stāvoklī uzliek norādi uz aizvietotās virknes sākumu. Gadījumā,  ja pārbaude nav veiksmīga, funkcija nemaina parsētāja stāvokli, un parsētājs var turpināt darbu nemodificētas gramatikas ietvaros.
\end{description}

Ja izstrādājamās valodas parsētāja modelis atbilst aprakstītām īpašībām, tad uz tās var veiksmīgi uzbūvēt aprakstāmo transformāciju sistēmu un ļaut programmētājam ieviest modifikācijas oriģinālās valodas sintaksē.

\subsection{Makro sistēmas sintakse}
Makro izteiksmes strādā stingri kaut kādas produkcijas ietvaros, tāpēc makro sintaksē tiek lietoti tipi, kas tiek apzīmēti ar produkciju nosaukumiem. Tipi tiks lietoti lai nodrošinātu tokenu virknes korektību sākotnējās gramatikas ietvaros pēc sintakses izmaiņu ieviešanas. Transformāciju sistēma sastāv no \emph{match} makro likumiem un transformāciju funkcijām. Makro kreisā puse satur regulāro izteiksmi no tokeniem un pseido-tokeniem, kas tālāk tiek izmantota lai atrast tokenu virkni, kurai šī transformācija ir pielietojama. Makro labā pusē ir atrodamas funkcijas, kas ir rakstītas vienkāršā funkcionāla valodā $T$. Valoda $T$ tiek lietota lai izpildītu transformācijas ar tokenu virknēm, kas tiek akceptēti ar makro funkcijas kreiso pusi.

Apakšnodaļā~\ref{subsec:system_qualities} tiks vispārīgi aprakstīta tipu sistēma un funkcionālā valoda $T$, bet tā kā tās neietilpst šī darba ietvaros, sīkāk tās aprakstītas nebūs.

Vispirms apskatīsim \textit{match} likumus, kas modificē apstrādājamās gramatikas produkcijas uzvedību. \emph{Match} makro sintakses visparīgu piemēru var redzēt figūrā~\ref{fig:matchsyntax}.
\begin{figure}[h!]
\begin{verbatim}
match [\prod1] v = regexp → [\prod2] f(v)
\end{verbatim}
\caption{\label{fid:matchsyntax}\emph{Match} makro sintakses vispārīgs piemērs}
\end{figure}

Šīs piemērs ir uztverams sekojoši. Ja produkcijas \verb|prod1| sākumā ir atrodama pseido-tokenu virkne, kas atbilst regulārai izteiksmei \verb|regexp|, tad tai tiek piekārtots mainīgais ar vārdu \verb|v|. Mainīgais \verb|v| var tikt lietots makro labajā pusē kaut kādas funkcijas izpildē. Tātad ja tāda virkne \verb|v| eksistē, tā tika aizstāta ar pseido-tokenu virkni, ko atgriezīs \verb|f(v)|  un tālāk reducēta pēc gramatikas produkcijas \verb|prod2| likumiem. 

Regulārā izteiksme \verb|regexp| ir vienkārša standarta regulārā izteiksme, kas gramatika ir definēta figūrā~\ref{fig:regexpsyntax}.

\begin{figure}[h!]
\begin{verbatim}
regexp          → concat-regexp | regexp
concat-regexp   → asterisk-regexp  concat-regexp
asterisk-regexp → unary-regexp * | unary-regexp
unary-regexp    → pseudo-token | ( regexp )
\end{verbatim}
\caption{\label{fig:regexpsyntax}Regulāro izteiksmju gramatika uz pseido-tokeniem}
\end{figure}

Pagaidām sistēmas prototipa izstrādē tiek lietota šāda minimāla sintakse, bet tālākā darba gaitā tā viegli var tikt paplašināta.

Vēl viens korektā makro piemērs: Pieņemsim, ka ērtības dēļ programmētājs grib ieviest sekojošu notāciju absolūtās vērtības izrēķināšanai - \verb/|<expr>|/. Sākotnējā valodas gramatikā eksistē absolūtās vērtības funkcija izskatā \verb|abs(<expr>)|. Tad makro, kas parādīts figūrā~\ref{fig:matchsample1} izdarītu šo substitūciju, ļaujot programmētājam lietot ērtāku funkcijas pierakstu.

\begin{figure}[h!]
\begin{verbatim}
match [{expr}] v = {|} {expr} {|}
    → [{expr}] {id:abs} {(} {expr} {)}
\end{verbatim}
\caption{\label{fig:matchsample1}Makro piemērs}
\end{figure}

Tagad mēs varam izveidot definētās makro sintakses korektu piemēru. Pieņemsim, ka funkcija \verb|replace| ir definēta valodā $T$ ar trim argumentiem, un darba gaitā tā jebkurā pseido-tokenu virknē aizvieto elementus, kas sakrīt ar otro argumentu, ar trešo funkcijas argumentu. Pieņemsim arī, ka mums ir nepieciešams izsaukt funkciju \verb|bar| ar vienu argumentu, kas ir summa no funkcijas \verb|foo| argumentiem. Šādā gadījumā makro, kas parādīts figūrā~\ref{fig:matchsample2}, izpildīs nepieciešamu darbību.

\begin{figure}[h!]
\begin{verbatim}
match [{expr}] v = {id:foo} {(} {expr} ( {,} {expr} ) * {)}
    → [{expr}] {id:bar} (replace (tail v) {,} {+})
\end{verbatim}
\caption{\label{fig:matchsample1}Makro piemērs}
\end{figure}

\subsection{\label{subsec:system_qualities}Sistēmas īpašības}
Šī nodaļa aprakstīs, kā mēs gribam realizēt gramatikas pašmodificēšanos, lai izmaiņas būtu kontrolētas.
Mēs gribam norobežot modificēšanas iespējas

\subsubsection{Pārrakstīšanas sistēma}
Match sistēma, atļauto regulāro izteiksmju sintakse. Sintakse ir viegli paplašināma.
Konteksti
\subsubsection{Tipu sistēma}
Kā tiks pārbaudīti tipi.

\subsection{Sistēmas sakars ar priekšprocesoriem}
Ir divu veidu priekšprocesori - leksiskie un sintaktiskie. Leksiskie priekšprocesori tiek palaisti pirms pirmkoda parsēšanas un nezin neko par apstrādājamas valodas sintaksi (piem. C/C++ priekšprocesors). No otras puses sintaktiskie priekšprocesori tiek palaisti pēc parsera darbības un apstrādā sintaktiskos kokus, ko uzbūvē parsētājs. Dēļ aprakstāmās sistēmas īpašībām šajā darbā netiks apskatīti sintaktiskie priekšprocesori, jo sistēmas īpašība ir tāda, ka līdz tas darba izpildei parsētājs nevar uzbūvēt sintaktisko koku.

Bet leksiskie priekšprocesori pēc savām īpašībām ir tuvi aprakstāmai sistēmai. Ar makro valodu palīdzību tiem tiek uzdoti koda pārrakstīšanas likumi, un kods tiek pārveidots attiecīgi tēs. Bet leksisko priekšprocesoru vislielākais trūkums ir tas, ka tie apstrādā tekstu pa tokeniem neievērojot izteiksmju un konstrukciju struktūru. Piemēram, apskatīsim šādu izteiksmi - \verb/|(a|b)+c|/, kurai vajadzētu tikt pārveidotai uz \verb/abs((a|b)+c)/. Ar tādu makro sistēmu, kas neievēro koda struktūru, tātad neievēro to, ka patiesībā \verb/(a|b)+c/ ir atomāra konstrukcija izteiksmē, šādu koda gabalu pareizi apstrādāt nevarēs. Vidējā \verb/|/ zīme sabojās konstrukciju un priekšprocesors nevarēs apstrādāt šādu gadījumu.

Priekšprocesoru var iemācīt apstrādāt šāda veida konstrukcijas un atpazīt tos, ka atomārās izteiksmes. Bet tas nozīmēs, ka priekšprocesoram būs jāzina apstrādājamas valodas sintakse, kas neatbilst priekšprocesora lomai kompilēšanas procesā un nozīmē ka būs divreiz jāimplementē sintakses atpazīšana.

Otrā problēma ar šāda tika priekšprocesoriem ir tas, ka tie strādā ārpus programmas kontekstiem. Tas nozīmē, ka konteksta sākuma tokens (\verb|{| C/C++, Java un citu valodu gadījumā) tiek uzskatīts par parastu tekstu un var tikt pārrakstīts. Loģiskāk būtu, ja kontekstu makro tiktu mantoti līdzīgi ka mainīgie  makro, kas ir specifiski kontekstam būtu ar lielāku prioritāti ka tie, kas definēti vispārīgākā kontekstā. 

\section{\label{s:prototype}Prototipa realizācija}

Šī darba ietvaros tika izstrādāts prototips sakrišanu meklēšanas sistēmai. Šī nodaļa apraksta prototipa īpašības, kā arī pieejas un algoritmus, kas tika lietoti tā realizācijā. Prototips vienkāršības un izstrādes ātruma dēļ tika rakstīts Python valodā, un ir viegli palaižams un atkļūdojams uz jebkura datora ar pieejamu 2.7. Python versiju.

Prototips tika izstrādāts ar iedomu pēc iespējas samazināt sakrišanu meklēšanas laiku, jo transformācijas sistēmas izsaukumi notiks katras produkcijas apstrādē.

Šīs nodaļas organizācija ir sekojoša. Apakšnodaļa~\ref{sbs:prot_syntax} īsi apraksta prototipam pieejamu sintaksi. Apakšnodaļa~\ref{sbs:prot_approach} apskata darbību virkni, ko prototips izpilda darba gaitā. Tālāk apakšnodaļa~\ref{sbs:prot_conflictresolving} apskata izvēlētās metodes šablonu konfliktu risināšanai. Apakšnodaļā~\ref{sbs:prot_motivation} ir aprakstīts, kāpēc tika izvēlēta šāda realizācijas pieeja un apakšnodaļa~\ref{sbs:prot_realizations} apskatīts, kāpēc dotajā gadījumā nav derīgi kādi gatavie risinājumi. Apakšnodaļa~\ref{sbs:prot_algorithms} stāsta par realizācijā lietotiem algoritmiem, bet apakšnodaļa~\ref{sbs:prot_problems}, savukārt, apraksta problēmas ar kurām saskārās darba autors un izņēmumus, kas pagaidām netiek implementēti prototipā.

\subsection{\label{sbs:prot_syntax}Atļautā makro sintakse}

Kā jau bija pateikts, makro pieejamā regulāro izteiksmju sintakse ir minimāla. Tā atļauj lietot \verb|*| lai identificēt daļiņu virknes un \verb/|/ lai izvēlēties starp dažiem daļiņu tipiem.

Prototips arī ļauj veidot regulārās izteiksmes ar specificētu daļiņu vai pseido-daļiņu vērtībām. Piemēram, regulārā izteiksme \verb|{id:foo}| sagaidīs tieši identifikatoru \verb|foo|, bet izteiksme \verb|{id}| sagaidīs jebkuru identifikatoru. Tas ievieš dažādas problēmas, kas tiks aprakstītas zemāk, bet dod lielāku brīvību šablonu sistēmas lietotājam.

\subsection{\label{sbs:prot_approach}Vispārīgā pieeja}

Prototips imitē darbu reālajā vidē, saņemot daļiņas no ieejas plūsmas pa vienam no atsevišķas saskarnes. Daļiņu plūsmas saskarne imitē leksera darbu. Kamēr prototips nav saņēmis nevienu regulāro izteiksmi, tas ignorē tokenu plūsmas apstrādes izsaukumus. Tiklīdz prototipam atnāk izsaukums apstrādāt tokenu regulāro izteiksmi, tas uzsāk regulārās izteiksmes parsēšanu. Parsēšanas procesā tiek izveidots galīgs automāts, kas akceptē regulārās izteiksmes uzdotās virknes. Katra jauna regulāra izteiksme izveido jaunu automātu.

Tad, kad atnāk daļiņu apstrādes pieprasījums, sistēma izpilda pārejas starp automātu stāvokļiem, meklējot sakrišanas, un atceras tokenus, kurus jau ir nolasījusi. Sistēma atrod garāko virkni, kas atbilst kādam no šabloniem un tad atgriež tās identifikatoru un nolasīto daļiņu virkni, lai turpmāk transformēšanas mehānisms varētu pārstrādāt to jaunajā virknē.

Pieņemot, ka transformēšanas sistēma ir izstrādāta, tālākā darba gaita būs sekojoša. Transformēšanas sistēma aizstāv ielasīto virkni ar citu, kas ir konstruēta pēc akceptētā makro šablona noteikumiem. Tad sakrišanas meklēšanas sistēmas darbs tiek uzsākts no jauna no aizvietotās virknes sākuma.

Sistēma turpina darbu aprakstītā gaitā līdz ko neviens no šabloniem vairs netiek akceptēts. Pēc sistēmas apstāšanās tiek iegūta jauna daļiņu virkne, kas tika apstrādāta attiecīgi kodā ierakstītiem makro, ja tika atrastas sakrišanas. Kad sistēma tiks integrēta ar reālu kompilatoru, tā strādās paralēli ar parsētāju un tālāk sistēmas izejas daļiņu virkne tiks apstrādāta ar standartiem valodas likumiem.

\subsection{\label{sbs:prot_conflictresolving}Makro konfliktu risināšana}
Makro šablonu konflikti var rasties tad, kad daži makro var tikt akceptēti vienlaikus. Tas var notikt gadījumos, ja divas regulārs izteiksmes akceptē līdzīgas virknes. Zemāk tiks aprakstīts, kā tika izvēlēts risināt dažādas konfliktu situācijas. 

Reālajā situācijā var rasties 3 konfliktu veidi. Pirmais var rasties gadījumā, kad divas izteiksmes atpazīst vienu un to pašu virkni vienā tvērumā. Otrais var rasties gadījumā, kad jaunajā tvērumā parādās šablons, kas ir līdzīgs jau eksistējošam šablonam no vispārīgāka tvēruma. Trešais var rasties tad, kad viena no izteiksmēm akceptē kādu virkni, bet cita akceptē garāku virkni.

\subsubsection{Divu makro konflikts vienā tvērumā.}

Gadījumā, ja viena tvēruma ietvaros eksistē divi šabloni, kas dod sakritību ar vienādu garumu, tad tiek ņemtas vērā prioritātes. Tā izteiksme, kas tika ielasīta agrāk būs ar lielāku prioritāti nekā tā, kas ir ielasīta vēlāk. Tātad ja secīgi tiks ielasītas divas izteiksmes \verb|{id} {(} {)}| un \verb|{id} {(} ( {real} * ) {)}|, tad ielasot virkni \verb|{id:foo} {(} {)}| tiks akceptēta pirmā izteiksme. Gadījumā, ja izteiksmes tiks ielasītas pretējā secībā, pirmā izteiksme nekad netiks atpazīta, jo otrā izteiksme pārklāj visas pirmās izteiksmes korektās ieejas.

\subsubsection{Divu makro konflikts dažādos tvērumos.}

Tvēruma iekšienē strādā tādi paši likumi par izteiksmju prioritātēm - izteiksme, kas bija agrāk ir ar lielāku prioritāti. Bet makro, kas ir specifiski tvērumam ir ar lielāku prioritāti nekā vispārīgāki makro. Tātad, ja pirmajā tvērumā tiks ieviestas makro ar identifikatoriem \verb|1| un \verb|2|, bet otrajā tvērumā tiks ieviestas makro ar identifikatoriem \verb|3| un \verb|4|, to prioritāšu rinda izskatīsies sekojoši: \verb|3, 4, 1, 2|. Pirmie makro ir ar lielāku prioritāti, nekā tie, kas atnāca vēlāk, bet vēlāka tvēruma makro ir ar lielāku prioritāti neka tie, kas atrodami agrākā tvērumā.

\subsubsection{Dažādu virkņu garumu konflikts}
Prototips strādā pēc mantkārīga (\emph{greedy}) principa - tas akceptē visgarāko iespējamo šablona sakritību. Tātad, ja eksistē divi šabloni \verb|{int} {,}| un \verb|({int} {,}) *|, tad tokenu virkne \verb|{int:4} {,} {int:6} {,}| tiks akceptēta ar otru šablonu, neskatoties uz to, ka augstākās prioritātes šablona sakritība tika konstatēta agrāk.

\subsection{\label{sbs:prot_motivation}Realizācijas pamatojums}

Galvenais princips uz kura bāzējas prototipa izstrāde ir samazināt apstrādes laiku. Tātad prototipa risinājums tika izveidots tā, lai jebkurā laika momentā šablonu sakrišanu meklēšanai būtu nepieciešams lineārs laiks un tikai viena tokenu virknes apstaigāšana. Šāda pieeja ir izvēlēta ar iedomu, ka makro pievienošana tiks izpildīta tikai vienreiz, un to daudzums būs samērā neliels, bet sakrišanu meklēšana tiks pildīta katrā produkcijā, un, sliktākajā gadījumā, katrai daļiņai no ieejas plūsmas.

Ielasītā regulārā izteiksme tiek pārsēta un pārveidota nedeterminēta galīgā automātā. Tad šablona nedeterminēts galīgs automāts tiek determinēts un minimizēts. Tātad katrai regulārai izteiksmei tiek izveidots minimāls determinēts automāts, kurš ir optimizēts gan pēc apstaigāšanas laika, gan pēc aizņemtās vietas.

Tālāk, lai nodrošinātu visu šablonu pārbaudi vienlaikus un minimizēt meklēšanas laiku, ir nepieciešams apvienot izveidotos automātus. To var izdarīt dažos veidos. Vienkāršākais no tiek būtu glabāt visus galīgos automātus sarakstā. Pieņemsim, ka ir $n$ šabloni, kurus vajag pārbaudīt. Tad automātu saraksts reprezentē nedeterminētu galīgu automātu ar $n$ $\varepsilon$-zariem no sākuma stāvokļa, katrs no kuriem ved pie sākuma stāvokļa vienam no jau izveidotiem determinētiem automātiem.

Cits veids, kā to varētu izpildīt, ir apvienot visus izveidotos šablonu automātus vienā determinētā galīgā automātā. Tieši šīs veids tika izvēlēts šī darba ietvaros lai pēc iespējas samazinātu laiku sakrišanu meklēšanai. Kaut arī automātu apvienošana šādā veidā ir laikietilpīga, tā samazina laika kārtu sakritību meklēšanai.

\subsection{\label{sbs:prot_algorithms}Lietotie algoritmi}

Šī apakšnodaļa apraksta algoritmus, kas tika lietoti prototipa realizācijā. Kā jau bija teikts, meklēšanas laika optimizācijai tika izvēlēta pieeja, kur visi regulāro izteiksmju automāti tiek sapludināti kopā.

Visu automātu pārejas pa zariem notiek nevis pa kādu simbolu, bet gan pa attiecīgu daļiņu. Regulāro izteiksmju apstrādes gaitā daļiņas \verb|{id:foo}| un \verb|{id}| tiek uzskatīti par dažādiem, kaut arī \verb|{id:foo}| ir apakšgadījums daļiņai \verb|{id}|. Šīs fakts tiek iegaumēts tikai sakrišanu meklēšanas gaitā.

Algoritmi, kas tika lietoti prototipa implementācijā tika ņemti no dažādiem literatūras avotiem, kas ir norādīti apakšnodaļās, un adaptēti lietošanai aprakstāmos nolūkos.

\subsubsection{Regulāro izteiksmju pārveidošana nedeterminētā galīgā automātā}

\fixme{Proof that they have the same computational power}

Regulāro izteiksmju translēšana uz nedeterminētu galīgu automātu ir diezgan vienkārša. Lai to paveikt ir nepieciešams  pārveidot galvenos regulārās izteiksmes kontroles elementus un automāta gabaliem. Tā kā uz doto brīdi prototips atbalsta tikai ierobežotu regulāro izteiksmju sintaksi, to ir vienkārši izdarīt.

Nedeterminēts galīgs automāts (NGA) veselai regulārai izteiksmei ir izveidots to daļējiem automātiem katrai regulārās izteiksmes daļai. Katram operatoram tiek piekārtota attiecīga konstrukcija. Daļējiem automātiem nav akceptējošu stāvokļu, tiem ir pārejas uz nekurieni, kuras vēlāk tiks lietotas lai savienotu automāta daļas. Pilnīga automāta būvēšanas process beigsies ar akceptējošā stāvokļa pievienošanu palikušajām pārejām. Zemāk tiek parādīti automāti katrai no regulārās izteiksmes iespējamām sastāvdaļām.

Attēlā~\ref{fig:auto_token} ir parādīts NGA vienai daļiņai ${token}$.
\begin{figure}[H]
  \centering
    \includegraphics[scale=1.5]{pictures/auto_token}
  \caption{\label{fig:auto_token}Automāts vienai daļiņai}
\end{figure}

Attēlā~\ref{fig:auto_sequence} ir parādīts NGA divu automātu konkatenācijai $e_1 e_2$.
\begin{figure}[H]
  \centering
    \includegraphics[scale=1.5]{pictures/auto_sequence}
  \caption{\label{fig:auto_sequence}Automāts divu automātu konkatenācijai}
\end{figure}

Attēlā~\ref{fig:auto_or} ir parādīts NGA izvēlei starp diviem automātiem $e_1 | e_2$.
\begin{figure}[H]
  \centering
    \includegraphics[scale=1.5]{pictures/auto_or}
  \caption{\label{fig:auto_or}Automāts izvēlei starp diviem automātiem}
\end{figure}

Attēlā~\ref{fig:auto_asterisk} ir parādīts NGA priekš konstrukcijas $e_1 *$.
\begin{figure}[H]
  \centering
    \includegraphics[scale=1.5]{pictures/auto_asterisk}
  \caption{\label{fig:auto_asterisk}Automāts automātu virknei}
\end{figure}

Tālāk šie automāti tiek apvienoti vienā, un beigās tiek pievienots akceptējošais stāvoklis.

Apskatīsim piemēru - izteiksmi \verb/{id} ({real} | {int}) */. Sākumā tiek izveidoti NGA izteiksmes daļām. Attēls~\ref{fig:auto_token_id} parāda NGA priekš \verb|{id}|. Attēls~\ref{fig:auto_or_ex} parāda NGA priekš daļas \verb/{real} | {int}/. Attēls~\ref{fig:auto_asterisk_ex} parāda NGA priekš \verb/({real} | {int}) */.

\begin{figure}[H]
  \centering
    \includegraphics[scale=1.5]{pictures/auto_token_id}
  \caption{\label{fig:auto_token_id}Automāts daļiņai \texttt{\{id\}}}
\end{figure}

\begin{figure}[H]
  \centering
    \includegraphics[scale=1.5]{pictures/auto_or_ex}
  \caption{\label{fig:auto_or_ex}Automāts izteiksmei \texttt{\{real\} | \{int\}}}
\end{figure}

\begin{figure}[H]
  \centering
    \includegraphics[scale=1.5]{pictures/auto_asterisk_ex}
  \caption{\label{fig:auto_asterisk_ex}Automāts izteiksmei \texttt{(\{real\} | \{int\}) *}}
\end{figure}

Tad izveidotie automāti var tikt savienoti un beigās tiem tiek pievienots akceptējošais stāvoklis (attēls~\ref{fig:auto_full_ex}).

\begin{figure}[H]
  \centering
    \includegraphics[scale=1.25]{pictures/auto_full_ex}
  \caption{\label{fig:auto_full_ex}Automāts izteiksmei \texttt{\{id\} (\{real\} | \{int\}) * \{;\}}}
\end{figure}

Tā tiek izveidots nedeterminēts automāts katrai regulārai izteiksmei. \cite{Cox:RegexpMatchingFast}, \cite{DragonBook}

\subsubsection{Determinizācija}

Kaut arī daudzām valodām ir vienkāršāk uzbūvēt nedeterminētu galīgu automātu (piemēram, pašām regulārām izteiksmēm tas ir loģiskāk), ir patiess tas, ka katra valoda var tikt aprakstīta gan ar nedeterminētu, gan ar determinētu galīgu automātu. Turklāt, dzīvē sastopamās situācijās DGA parasti satur tik pat daudz stāvokļu, cik ir NGA. Sliktākajā gadījumā, tomēr var gadīties, ka mazākais iespējamais DGA saturēs $m^n$ stāvokļu ($m$ - ieejas alfabēta elementu skaits), kamēr mazākais NGA saturēs $n$ stāvokļus\footnote{Pieņemsim, ka automāta valoda sastāv no diviem simboliem - {0, 1}. Sliktākais gadījums, kad DGA tiešām saturēs $2^n$ stāvokļus attiecībā pret $n$ NGA stāvokļiem, var rasties tad, kad, piemēram, automāta valodā $n$-tais simbols no virknes beigām ir 1. Tad DGA būs jāprot atcerēties pēdējos $n$ simbolus. Tā kā ir divi ieejas alfabēta simboli, automātam ir jāatceras visas to dažādas $2^n$ kombinācijas.}.

Izrādās, ka patiesībā katram NGA eksistē ekvivalents DGA, ko var uzbūvēt ar apakškopu sastādīšanas algoritmu\footnote{Pierādījumu tam, ka uzbūvētais DGA tik tiešām akceptē to pašu valodu, ko NGA, sk.  \cite{Hopcroft:IntroAutomataTheory}, teorēma 2.11.}. Vadošā doma šī algoritmā ir tas, ka katrs determinētā galīgā automāta (DGA) stāvoklis ir kādu NGA stāvokļu kopa. Pēc ieejas virknes $a_1, a_2, ..., a_n$ ielasīšanas DGA atrodas stāvoklī, kas atbilst NGA stāvokļu kopai, kuru var sasniegt apstaigājot virkni $a_1, a_2, ..., a_n$.

Algoritms ir paņemts no~\cite{DragonBook}.

\paragraph*{Algoritms 1: NGA transformēšana uz DGA}
\subparagraph{Ieeja:}NGA $N$.
\subparagraph{Izeja:}DGA $D$, kas ir ekvivalents $N$.
\subparagraph{Algoritms:} Sākumā algoritms konstruē pāreju tabulu priekš $D$. Katrs $D$ stāvoklis ir $N$ stāvokļu kopa, tātad tabula tiek konstruēta tā, lai $D$ simulētu vienlaikus visas pārejas, ko var izpildīt $N$, saņemot kādu ieejas virkni. Lai automāts kļūtu determinēts, ir nepieciešams atbrīvoties no iespējas atrasties dažos stāvokļos vienlaikus. Tātad vajag atbrīvoties no $\varepsilon$-pārejām, un no daudzkārtīgām pārejām no viena stāvokļa pa vienu ieejas simbolu.

Tabulā~\ref{fig:NGAoperations} var redzēt divas funkcijas, kas ir nepieciešamas NGA apstrādes izpildei. Šīs funkcijas no NGA stāvokļiem un pārejām veido jaunas stāvokļu kopas, kuras veidos DGA stāvokļus.

\begin{table}[H]
  \caption{NGA apstaigāšanas funkcijas}
  \centering
  \begin{tabular}{|c|p{350pt}|}
    \hline
    Funkcija & Apraksts \\ \hline
    $\varepsilon-closure (T)$ & 
    NGA stāvokļu kopa, kas ir sasniedzama lietojot tikai $\varepsilon$-pārejas no visiem stāvokļiem no kopas $T$. \\ \hline
    $move (T, a)$ & 
    NGA stāvokļu kopa, kas ir sasniedzama lietojot pārejas pa simbolu $a$ no visiem stāvokļiem no kopas $T$. \\
    \hline
  \end{tabular}
\label{fig:NGAoperations}
\end{table}

Ir nepieciešams apstrādāt visas tādas $N$ stāvokļu kopas, kuras ir sasniedzamas, $N$ saņemot kaut kādu ieejas virkni. Indukcijas bāzes pieņēmums ir tas, ka pirms darbības uzsākšanas $N$ var atrasties jebkurā no stāvokļiem, kurus var sasniegt pārejot pa $\varepsilon$ bultiņām no $N$ sākuma stāvokļa. Ja $s_0$ ir $N$ sākuma stāvoklis, $D$ sākuma stāvoklis būs $\varepsilon-closure (set (s_0))$. Indukcijai pieņemam, ka $N$ var atrasties $T$ stāvokļu kopā pēc virknes $x$ ielasīšanas. Tad, ja $N$ ielasīs nākamo simbolu $a$, tad $N$ var pārvietoties jebkura no stāvokļiem $move (T, a)$. Taču pēc $a$ ielasīšanas var notikt vēl dažas $\varepsilon$-pārejas, tāpēc pēc virknes $xa$ ielasīšanas $N$ var atrasties jebkurā no stāvokļiem $\varepsilon-closure (move (T, a))$. Attēls~\ref{fig:det_algorithm} parāda pseido-kodu algoritmam, kā šādā veidā var tikt uzkonstruēti visi DGA stāvokļi un tā pāreju tabula.

Automāta $D$ sākuma stāvoklis ir $\varepsilon-closure (set (s_0))$, bet $D$ akceptējošie stāvokļi ir visas tās NGA stāvokļu kopas, kas satur vismaz vienu akceptējošu stāvokli. $Dstates$ ir jauna automāta $D$ stāvokļu saraksts un $Dtran$ ir stāvokļu pāreju tabula.

\begin{figure}[H]
  \begin{algorithmic}
  \State initially, $\varepsilon-closure (set (s_0))$  is the only state in $Dstates$, and is unmarked
  \While{there is an unmarked state $S$ in $Dstates$}
      \State mark $S$
      \For{each available oath $t$ from $S$} 
          \State $U = \varepsilon-closure (move (S, a))$
          \If {$U$ is not in $Dstates$}
              \State add $U$ as an unmarked state to $Dstates$
          \EndIf
          \State $Dtran [S, a] = U$;
      \EndFor
  \EndWhile
  \end{algorithmic}
  \caption{\label{fig:det_algorithm}Automāta determinēšanas algoritms}
\end{figure}

\subparagraph{Sarežģītība:} Sarežģītības novērtējums šīm algoritmam ir diezgan nepatīkams. Sliktākajā gadījumā tas būs $O(m^(n+1))$, kur $n$ ir NGA stāvokļu daudzums un $m$ ir ieejas alfabēta simbolu skaits. Algoritms var uzģenerēt līdz $m^n$ DGA stāvokļiem, katram no kuram ir $m$ pārejas. Taču parasti tas tā nenotiek un DGA stāvokļu skaits ir līdzīgs NGA stāvokļu skaitam, un algoritma sarežģītība ir $O(n*m)$. \cite{DragonBook, Hopcroft:IntroAutomataTheory}

\subsubsection{\label{sbsbs:prot_minimization}Minimizēšana}

Izveidotais determinēts galīgs automāts var būt neoptimāls pēc stāvokļu skaita. Bet no šī skaitļa ir atkarīgs tālāko soļu izpildes ātrums. Tātad ir nepieciešams izveidot automātu, kas atpazīs to pašu valodu un saturēs minimālu iespējamu stāvokļi skaitu.

Var pierādīt, ka katram automātam eksistē ekvivalents minimāls automāts\footnote{Šī fakta pierādījumu sk.~\cite{Bassino:ComplexityMinimization}}. Vēl vairāk, ja eksistē 2 dažādi automāti ar vienādu stāvokļu daudzumu, kas atpazīst vienu un to pašu valodu, tad tie ir vienādi līdz stāvokļu nosaukumiem\footnote{Tā kā stāvokļu nosaukumi neietekmē automāta darbību, divi automāti tiek saukti par vienādiem līdz pat stāvokļu nosaukumiem, ja viens no tiem var tikt pārveidots otrajā vienkārši pārsaucot to stāvokļus.}.

Tālāk teiksim, ka virkne $x$ atšķir stāvokļu $s$ no stāvokļa $t$ tad, kad tikai viens stāvoklis, ko var sasniegt no $t$ un $s$ pa $x$ ir akceptējošs. Tātad divi stāvokļi ir atšķirami tad, kad eksistē tāda virkne, kas viņus atšķir. Jebkurš akceptējošs stāvoklis ir atšķirams no jebkura neakceptējoša stāvokļa ar tukšu virkni (stāvoklis nevar būt akceptējošs un neakceptējošs vienlaikus). Divi neatšķirami stāvokļi ir ekvivalenti\footnote{\cite{Hopcroft:IntroAutomataTheory} nodaļa 4.4}.

Algoritms ir paņemts no~\cite{DragonBook}.

\paragraph*{Algoritms 2: DGA minimizēšana}
\subparagraph{Ieeja:}DGA $D$.
\subparagraph{Izeja:}Jauns DGA $D'$, kas ir minimāls un ekvivalents $D$.
\subparagraph{Algoritms:} 

Minimizēšanas algoritma vadošā doma ir sadalīt automātu neatšķiramos stāvokļu grupās. Tas izveido ekvivalentu stāvokļu grupas, kas tālāk var tikt apvienotas vienā, izveidojot minimāla automāta stāvokļus.

Minimizēšanas gaitā automāta stāvokļi tiek sadalīti grupās, ko uz doto brīdi algoritms nevar atšķirt. Jebkuri divi stāvokļi no dažādām grupām ir atšķirami. Katrā nākamajā algoritma iterācijā eksistējošās grupas tiek sadalītas mazākajās grupās, gadījumā, ja kādā grupā parādās atšķirami stāvokļi. Algoritms apstājas līdz ko neviena grupa nevar tikt sadalīta sīkāk.

Pirms algoritms uzsāk darbu, stāvokļi tiek sadalīti divās grupās - akceptējošie stāvokļi un neakceptējošie stāvokļi. Šo grupu stāvokļi ir atšķirami ar tukšu virkni. Tālāk tiek pa vienai apstrādātas grupas no pašreizējā sadalījums. Katrai grupai tiek pārbaudīts, vai tās stāvokļi var tikt atšķirti ar kādu ieejas simbolu - vai kāds no ieejas simboliem noved uz divām vai vairākām dažādam stāvokļu grupām. Ja tādi simboli eksistē, tiek izveidotas jaunas grupas, tādas, ka divi stāvokļi atrodas vienā grupā tad un tikai tad, ja tie aiziet uz vienādām grupām pa vienādiem simboliem. Process ir atkārtots visām pašreizējā sadalījuma grupām, tad atkal jaunam sadalījumam, kamēr neviena no grupām vairs nevar tikt sadalīta.

Attēls~\ref{fig:min_algorithm} parāda minimizēšanas algoritmu pseido-kodā.

\begin{figure}[H]
  \begin{algorithmic}
  \State initially, partitioning $\Pi$ contains two groups, $F$ and $S-F$, the accepting and nonaccepting states of $D$, $\Pi_{new}$ is empty
  \While{$\Pi_{new}$ is not equal to $\Pi$}
      \State $\Pi_{new} = \Pi$
      \For{each group $G$ of $\Pi$}
          \State partition $G$ into subgroups such that two states $s$ and $t$ are in the same subgroup if and only if for all input symbols $a$, states $s$ and $t$ have transitions on $a$ to states in the same group of $\Pi$
          \State replace $G$ in $\Pi_{new}$ by the set of all subgroups formed
      \EndFor
  \EndWhile
  \State $\Pi_{final} = \Pi$
  \end{algorithmic}
  \caption{\label{fig:min_algorithm}Automāta minimizēšanas algoritms}
\end{figure}

Tālāk paliek apstrādāt jaunizveidotās stāvokļu grupas izveidojot jaunu determinētu automātu. Lai to izdarītu, no katras sadalījuma $\Pi_{final}$ grupas tiek izvēlēts grupas pārstāvis. Grupu pārstāvji izveidos jaunus stāvokļus automātam $D'$. Pārējās komponentes minimālam automātam $D'$ tiks izveidotas sekojoši:
\begin{enumerate}
\item Automāta $D'$ sākuma stāvoklis ir pārstāvis tai grupai, kura satur automāta $D$ sākuma stāvokli.
\item Automāta $D'$ akceptējošie stāvokļi ir pārstāvji tām grupām, kuras satur automāta $D$ akceptējošos stāvokļus. Katra no grupām satur vai nu tikai akceptējošus, vai nu tikai neakceptējošus stāvokļus, jo algoritma darba gaitā jaunās grupas tika izveidotas tikai sadalot jau eksistējošas grupas, bet sākuma sadalījums atdalīja šīs stāvokļu klases.
\item Pieņemsim, ka $s$ ir kādas $\Pi_{final}$ grupas $G$ pārstāvis, un automāts $D$ no stāvokļa $s$ pa ieejas simbolu $a$ pāriet uz stāvokli $t$. Pieņemsim, ka $r$ ir grupas $H$ pārstāvis, $H$ satur $t$. Tad automātā $D'$ ir pāreja no stāvokļa $s$ uz stāvokli $r$ pa ieejas simbolu $a$.
\end{enumerate}

\subparagraph{Sarežģītība:}
Šī algoritma sarežģītība sliktākajā gadījumā ir $O(n^2)$, kur $n$ ir sākotnējā automāta stāvokļu daudzums. Tomēr vidēji algoritma darbības sarežģītība ir $O(n$ log $n)$. \cite{Bassino:ComplexityMinimization}

\subsubsection{Apvienošana}

Kā jau bija teikts agrāk, lai samazinātu sakrišanu meklēšanas laiku, tika izvēlēts apvienot visus meklēšanas automātus vienā. Tā kā makro atnāk pa vienam dažādās vietās programmas kodā un var sākt uzreiz tikt lietotas, nav iespējams gaidīt kamēr sakrāsies vairāki automāti apvienošanai. Tikko parādās divi automāti tie tūlīt pat tiek apvienoti vienā sistēmā. Tālāk, kad parādās citi makro, to automāti tiek pievienoti jau eksistējošam.

Algoritms pēc savas būtības ir determinēšanas algoritma adaptācija. Vienīgais uzņēmums ir tas, ka nevienā no automātiem neeksistē $\varepsilon$-pārejas. Tātad tas, no kā vajag atbrīvoties, ir pārejas pa vienu un to pašu simbolu uz diviem dažādiem stāvokļiem. Tas tiek darīts apvienojot divus stāvokļus no dažādiem automātiem. 

\paragraph*{Algoritms 3: Divu DGA apvienošana}
\subparagraph{Ieeja:}DGA $D_1$ un $D_2$.
\subparagraph{Izeja:}Jauns DGA $D'$, kas apvieno $D_1$ un $D_2$.
\subparagraph{Algoritms:} 

Algoritms sāk darbu apvienojot $D_1$ un $D_2$ sākuma stāvokļus. Šo stāvokļu kombinācija veido automāta $D'$ sākuma stāvokli. Tālāk algoritms apskata visas iespējamas pārejas no katra no kombinētiem stāvokļiem un apvieno to rezultātus jaunajos stāvokļos. 

Apskatīsim piemēru automātu apvienošanai. Attēls~\ref{fig:auto_m_1} parāda determinētu minimālu automātu priekš regulārās izteiksmes \verb|{id}* {int}*|. Tas satur divus stāvokļus, \verb|a| un \verb|b|, kuri abi ir akceptējoši. Attēls~\ref{fig:auto_m_2}, savukārt, parāda DGA priekš izteiksmes \verb|{id} {real}* {int}|.

\begin{figure}[H]
  \centering
    \includegraphics[scale=1.5]{pictures/auto_m_1}
  \caption{\label{fig:auto_m_1}Automāts izteiksmei \texttt{\{id\}* \{int\}*}}
\end{figure}

\begin{figure}[H]
  \centering
    \includegraphics[scale=1.5]{pictures/auto_m_2}
  \caption{\label{fig:auto_m_2}Automāts izteiksmei \texttt{\{id\} \{real\}* \{int\}}}
\end{figure}

To apvienotais automāts ir parādīts attēlā~\ref{fig:auto_merge}.

\begin{figure}[H]
  \centering
    \includegraphics[scale=1.5]{pictures/auto_merge}
  \caption{\label{fig:auto_merge}Automāts izteiksmju \texttt{\{id\}* \{int\}*} un \texttt{\{id\} \{real\}* \{int\}} apvienojumam}
\end{figure}

Attēls~\ref{fig:uni_algorithm} parāda apvienošanas algoritmu pseido-kodā. $s_0$ ir jaunā automāta $D'$ sākuma stāvoklis, $r_0$ ir $D_1$ sākuma stāvoklis un $t_0$ ir $D_2$ sākuma stāvoklis. $Dstates$ ir automāta $D'$ stāvokļu saraksts, $Dtran$ ir $D'$ pāreju tabula. Funkcija $move (S, t)$ atgriež tos stāvokļus, uz kuriem var nokļūt no $S$ pa tokenu $t$. Tā kā parasti $S$ sastāv no diviem stāvokļiem $r_i$ un $t_i$, tā atgriež pārejas rezultātu no katra no tiem. Rezultāts arī var būt tikai viens stāvoklis, gadījumā, ja no kāda $r_i$ un $t_i$ neeksistē pāreja pa doto daļiņu.

\begin{figure}[H]
  \begin{algorithmic}
  \State initially, $s_0 = (r_0, t_0)$ is the only state in $Dstates$, and is unmarked
  \While{there is an unmarked state $S$ in $Dstates$}
      \State mark $S$
      \For{each available move $t$ from $S$} \Comment{$S$ is a combination of some states $r_i$ and $t_i$ of $D_1$ and $D_2$, although it might be just a single state from one of the automata.}
          \State $U = move (S, t)$
          \If {$U$ is not in $Dstates$}
              \State add $U$ as an unmarked state to $Dstates$
          \EndIf
          \State $Dtran [S, t] = U$;
      \EndFor
  \EndWhile
  \end{algorithmic}
  \caption{\label{fig:uni_algorithm}Divu automātu apvienošanas algoritms}
\end{figure}

\subparagraph{Sarežģītība:}
Šī algoritma sarežģītība ir $O(n*m*l)$, kur  $n$ ir pirmā automāta stāvokļu daudzums, $m$ ir otrā automāta stāvokļu daudzums, un $l$ maksimāls pāreju daudzums no katra no stāvokļiem.

\subsubsection{Sakrišanu meklēšana}
Sakrišanu meklēšana NGA sliktākajā gadījumā būs ar sarežģītību $O(n^2*l)$, kur $n$ ir NGA stāvokļu skaits un $l$ - pārbaudāmās virknes elementu skaits. Tas var notikt, kad visi NGA stāvokļi ir aktīvi vienā laika brīdī, un no katra no tiem eksistē $n$ pārejas pa ieejas simbolu uz visiem automāta stāvokļiem.

Tīrā DGA gadījumā sakrišanu pārbaude katram ieejas elementam ir $O(l)$, kur $l$ ir pārbaudāmās virknes garums. 

Diemžēl pilnībā lineāra laika sakrišanu meklēšana nav iespējama tādēļ, ka prototips dod iespēju lietot makro ar tokenu vērtībām. Piemēram, eksistē 2 makro, viens no kuriem gaida tokenu \verb|{id}|, un otrs \verb|{id:foo}|. Gadījumā, kad sakrišanu meklēšanas procesā parādās daļiņa \verb|{id:foo}|, automātam nav iespējas izsecināt, kurš no ceļiem novedīs pie garākas sakritības. Tādēļ tas iet pa abiem ceļiem vienlaikus, saglabājot abus stāvokļus.

Kaut arī tas ievieš nenoteiktību, tā var parādīties tikai augstāk minētā gadījumā un izveidot ne vairāk ka 2 ceļus vienlaikus. Tātad kaut arī nenoteiktība pastāv, tai ir ļoti maza iespējamība un maza ietekme uz sakrišanu meklēšanas laiku.

\subsubsection{Tvērumi}

Viena no galvenām šīs sistēmas īpašībām ir iespēja atšķirt programmatūras tvērumus. Ir dažādi veidi, kā var izveidot automātus, kas atšķirtu atsevišķu tvērumu makro. Viens no veidiem varētu būt tāds - katram tvērumam izveidot automātu rindu, kur tvērumam specifiskākie automāti tiks pārbaudīti pirmie. Bet gadījumā, ja ir $n$ iekļautie tvērumi un nepieciešamais makro ir atrodams pirmajā automātā, būs jāizpilda vismaz $n$ meklēšanas, līdz ko pareizais šablons tiks atrasts. Atstājot tikai vienu aktīvu automātu vienā laika brīdī, arī ir dažas iespējas. Varētu visus šablonus likt vienā automātā kopā, un tad pēc izejas no tvēruma attiecīgos šablonus dzēst ārā. Tas nozīmētu, ka katram tvērumam jāatceras makro, kas tika pievienoti, un jāprot dzēst daļu no stāvokļiem ārā no automāta. Bet stāvokļu dzēšana ir laikietilpīga operācija, jo tās izpildīšanai būs nepieciešams apstaigāt visu lielo automātu, dzēšot no tā nevajadzīgos stāvokļus.

Tāpēc tika izvēlēta sekojoša pieeja. Ja prototips darba gaitā sastapās ar tvēruma sākuma simbolu, tas izveido eksistējošā automāta kopiju un ieliek to kaudzē. Tad automātam tiek pievienotas tvēruma makro. Izejot no tvēruma tā specifiskais automāts tiek izmests ārā un darbs tiek turpināts ar pēdējo automātu no kaudzes, kas atbilst iepriekšējam tvērumam. Šādā veidā jebkurā laika brīdī aktīvs ir tikai viens sakrišanu meklēšanas automāts.

\subsection{\label{sbs:prot_problems}Izņēmumi}

\subsubsection{Transformācijas}

Izstrādātais prototips nenodarbojas ar tokenu virkņu transformācijām, jo tā nav sakrišanu meklēšanas mehānisma uzdevums. Tālākajā sistēmas izstrādes gaitā notiks integrācija ar transformāciju sistēmu vai arī abu mehānismu sapludināšana vienā.

\subsubsection{Produkcijas}

Prototipā pagaidām nav implementēta apstrādes dalīšana pa gramatikas produkcijam, visas regulārās izteiksmes ir sapludinātas vienā automāta. Regulāro izteiksmju dalīšana pa tipiem tiks ieviesta vēlāk, kad tiks uzsākta integrācija un sadarbība ar reālu parsētāju. Tā varētu tikt implementēta līdzīgi tam, kā tiek realizēti konteksti - pa vienam sapludinātam automātam priekš katra produkcijas tipa.

\subsubsection{Daļiņu klašu mantošana}

Sistēmai nav nekādas informācijas par valodas gramatiku. Tieši tāpēc daļiņa \verb|{real}| netiks uztverta ka \verb|{expr}|, kaut arī racionāls skaitlis ir izteiksme. Tā kā sistēmai jābūt neatkarīgai no valodas gramatikas, šī hierarhija nav iekodējama transformāciju sistēmā. To ir jānodrošina parsētājam, attiecīgi apstrādājot tokenus un apkopojot to nozīmi. Par to arī būs jārūpējas sistēmas lietotājam rakstot savas makro izteiksmes.

\subsubsection{Regulārās izteiksmes daļu grupēšana}

Tādēļ, ka aprakstītā pieeja mēģina pēc iespējas minimizēt meklēšanas laiku, tā neatļauj veidot atrasto tokenu grupēšanu, kā tas ir parasti pieņemts regulārās izteiksmēs. Daļiņu grupēšana un atpakaļnorādes (\emph{backreferences}) uz tokenu grupām nav atļautas.

Tomēr ja parādīsies nepieciešamība, ir apskatīta arī pieeja, kas dos šādu iespēju. Tas varētu tikt izpildīts, determinējot tikai automāta stāvokļus grupu iekšienē, pēc tam ar $\varepsilon$-pārejām secīgi savienojot grupu automātu akceptējošus un sākuma stāvokļus. Automāts būs determinēts tikai grupu ietvaros, bet tad būs pieejamas atrasto tokenu grupas. Tomēr šāda pieeja neatļaus sapludināt dažus automātus vienā, jo sapludināšanas procedūra sabojās grupēšanu.

\subsubsection{Sapludinātā automāta minimizēšana}

Automātu minimizēšana ir diezgan darbietilpīga operācija, tāpēc tā tiek izpildīta tikai uz atsevišķiem automātiem. Apvienotais visu šablonu automāts var nebūt minimāls, jo dažādu minimālu automātu apvienošana negarantē šo faktu. Bet tā kā apvienota automāta stāvokļu daudzums var būt ļoti liels, minimizēšanas izpilde var būt neefektīva. Tā kā minimizēšana samazina tikai automāta aizņemto vietu, nevis apstaigāšanas laiku, to šajā gadījumā var izlaist. Sapludinātā automāta minimizēšanu apgrūtina arī tas fakts, ka to stāvokļus vajadzēs atšķirt arī pēc tā fakta, kāds no šabloniem ir akceptēts, nevis tikai pēc tā, vai stāvoklis ir akceptējošs.

\subsection{\label{subsec:solution_optimization}Optimizācijas iespējas}

Regulāro izteiksmju optimizēšana uz doto brīdi netiek izpildīta, jo to ir vērts izpildīt uz regulārām izteiksmēm, kas tiks lietoti daudzas reizes. Darba apskatītā situācijā regulārās izteiksmes tiks lietotas tikai vienas programmas ietvaros un to optimizēšanai nav īpašas jēgas. Tomēr bez reāliem piemēriem nevar izšķirt, vai tas izveidos būtisku paātrinājumu šādā konkrētā gadījumā vai nē. Dažas regulāro izteiksmju pārrakstīšanas pieejas, kas varētu būt lietotas tālākā izstrādē ir aprakstītas \cite{Yu:FMR}.

Dažreiz divu automātu sapludināšana var izraisīt pārāk lielu stāvokļu daudzumu rašanos. Ja reālajā situācijā tas ietekmēs apstrādes laiku, vai parādīsies nepieciešamība ierobežot automāta aizņemto laiku, var apskatīt iespēju glabāt dažus automātus ar stāvokļu skaitu robežu, nevis tikai vienu. Šāda pieejā arī ir aprakstīta avotā \cite{Yu:FMR}.

Minimizēšanas algoritmu var aizvietot ar citu, ātrāku algoritmu, piemēram, Hopkrofta minimizēšanas algoritmu, kura izpildes laiks ir $O(n$ log $n)$, kur $n$ ir automāta stāvokļu daudzums. \cite{Berstel:MA}

\subsection{\label{sbs:res_testing}Prototipa testēšana}

Prototips tika testēts visā izstrādes laikā. Zemāk tiks aprakstīti automātiski palaižamie testi. 

\subsubsection{Stresa testēšana}

Prototips izstrādes laikā tika testēts ar lieliem automātiski ģenerēto datu apjomiem. Tika ģenerētas patvaļīgas regulāras izteiksmes ar iekavu un \verb|*| un \verb/|/ simbolu palīdzību. Katrai regulārai izteiksmei tika izveidota arī simbolu virkne, ko šai izteiksmei jāprot atpazīt. Tad uz vienas un tās pašas regulārās izteiksmes un simbolu virknes tika palaists gan prototips, kas tolaik apstrādāja simbolus, gan Python iebūvētais regulāro izteiksmju apstrādes mehānisms. Vienā piegājienā tika ģenerēti 500 šādi testi. Prototipa beigu izstrādes posmā visi šādi testi tika veiksmīgi izpildīti.

Diemžēl pagaidām šī pieeja netiek implementēta ar daļiņu regulārām izteiksmēm, jo nav iespējams pārbaudīt sistēmas darba ekvivalenci ar kādu citu sistēmu. Tāpēc tika veikta intensīva sistēmas testēšana, lai pārbaudītu pēc iespējas vairāk reālajā darbā iespējamo situāciju. Tomēr šādas stresa pārbaudes parādīja ka pats regulāro izteiksmju apstrādes mehānisms strādā korekti.

\subsubsection{Sistēmas testēšana}

Prototipam tika izveidoti apmēram 20 testi, kas pārbauda to darbību iespējamās situācijās. Tabula~\ref{fig:tests} parāda konceptuālu testu sadalījumu pa grupām. Dažas grupas pārklājas, jo, piemēram, tvērumu pārbaudošie testi pārbauda arī korektas regulāro izteiksmju prioritātes.

\begin{longtable}{|p{90pt}|p{210pt}|p{120pt}|}
\caption{\label{fig:tests}Prototipa testu sadalījums pa grupām} \\ \hline
\textbf{Testa nosaukums} & \textbf{Testa apraksts} & \textbf{Kas tiek pārbaudīts} \\ \hline
\endhead
\multicolumn{ 3}{|c|}{Testi prioritāšu pārbaudei} \\ \hline
Testi ar dažiem šabloniem vienā tvērumā & Testu gaitā tiek izveidotas dažādas regulāras izteiksmes, kuru aprakstītās valodas pārklājas. Tās tiek ievietotas vienā tvērumā pēc kārtas. Tad tiek pārbaudīts, ka daļiņu saraksts tiek akceptēts ar pareizu šablonu attiecībā pret to prioritātēm. & Vai tiek korekti apstrādātas šablonu prioritātes viena tvēruma ietvaros. \\ \hline
Testi ar dažiem šabloniem vienā tvērumā, kur kāds no šabloniem akceptē garāku virkni & Testu gaitā tiek izveidotas dažādas regulāras izteiksmes, kuru aprakstītās valodas pārklājas. Tās tiek ievietotas vienā tvērumā pēc kārtas. Tad tiek pārbaudīts, ka tiek akceptēta garākā iespējamā daļiņu virkne. & Vai tiek korekti apstrādātas šablonu prioritātes viena tvēruma ietvaros. \\ \hline
\multicolumn{ 3}{|c|}{Testi daļiņu vērtībām} \\ \hline
Testi ar daļiņu vērtībām & Testu gaitā tiek izveidotas dažādas regulārās izteiksmes ar daļiņu vērtībām un bez tām. Tiem tiek padotas dažādas daļiņu virknes. & Vai tiek korekti apstrādātas šablonu prioritātes un vērtību sakrišanas. \\ \hline
Testi ar vairākiem pieejamiem stāvokļiem vienlaikus & Testu gaitā tiek izveidotas dažādas regulārās izteiksmes ar daļiņu vērtībām. Tiem tiek padotas daļiņu virknes ar šādām pašām vērtībām, lai izveidotu situācijas, kad ir pieejami daži stāvokļi vienlaikus. & Vai tiek korekti apstrādātas situācijas, kad parādās nenoteiktība. \\ \hline
\multicolumn{ 3}{|c|}{Testi tvērumu pārbaudei} \\ \hline
Testi ar tvērumu iekļaušanas dziļumu 1 & Testu gaitā tiek izveidotas dažas regulāras izteiksmes, kuru aprakstītās valodas pārklājas. Tās tiek ievietotas nultajā un pirmajā tvērumā. Tad tiek pārbaudīti daļiņu saraksti. & Vai tiek korekti apstrādāta tvēruma parādīšanās. Vai tiek korekti apstrādātas šablonu prioritātes starp tvērumiem. \\ \hline
Testi ar tvērumu iekļaušanas dziļumu >1 & Testu gaitā tiek izveidotas dažas regulāras izteiksmes, kuru aprakstītās valodas pārklājas. Tās tiek ievietotas dažādos tvērumos ar dziļumu kas ir lielāks par vienu. Tad tiek pārbaudīti daļiņu saraksti. & Vai tiek korekti apstrādāti dažādi tvērumu dziļumi. Vai tiek korekti apstrādātas prioritātes starp tvērumiem. \\ \hline
Testi ar izeju no tvēruma & Testu gaitā tiek izveidotas dažas regulāras izteiksmes, kas tiek ieliktas nultajā un pirmajā tvērumā. Tiek pārbaudīts, ka pirmā tvēruma šablons atpazīst daļiņu virkni. Tālāk pirmais tvērums tiek pamests un tiek pārbaudīts, ka tā šablons vairs nav aktīvs. & Vai pēc izejas no tvēruma attiecīgie šabloni ir noteikti izdzēsti. \\ \hline
Testi ar izeju no tvēruma un nākamā tvēruma izveidi & Testu gaitā tiek izveidots 1. līmeņa tvērums ar regulārām izteiksmēm. Tiek pārbaudīts, ka pirmā tvēruma šabloni atpazīst daļiņu virknes. Tad šīs tvērums tiek pamests un tiek izveidots jauns pirmā līmeņa tvērums. Tiek pārbaudīts, ka vecā tvēruma šabloni ir izmesti, un ka jaunā tvēruma šabloni tiek atpazīti. & Vai pēc izejas no tvēruma attiecīgie šabloni ir izdzēsti un pēc ieejas jaunajā tvērumā tiek akceptēti pareizi šabloni. \\ \hline
\multicolumn{ 3}{|c|}{Testi bez šablonu sakritībām} \\ \hline
Testi bez neviena šablona & Testu gaitā tiek izveidota sistēma bez neviena šablona. Tiek pārbaudīts, ka neviena sakritība netiek atrasta. & Vai sistēma korekti apstrādā situāciju, kad nav neviena šablona. \\ \hline
Testi ar šabloniem un datiem kas nesakrīt & Testu gaitā tiek izveidota sistēma ar dažiem šabloniem. Tad tiek padotas daļiņu virknes kuras neder nevienam no eksistējošiem šabloniem. & Vai sistēma korekti apstrādā situāciju, kad neviena sakrišana nav atrasta. \\ \hline
\end{longtable}

\subsection{\label{sbs:res_tintegration}Prototipa integrēšana Eq}

Prototips pagaidām netiek integrēts Eq valodas kompilatorā, bet tas tiek plānots tuvākajā nākotnē. Tā kā prototips tika izstrādāts bāzējoties uz Eq parsētāja īpašībām, to būs viegli integrēt eksistējošā kodā. Tā darbs ir gandrīz neatkarīgs no parsētāja darba un neietekmēs jau eksistējošo programmu darbību.

Prototips piedāvā saskarni lai uzsākt jauna tvēruma apstrādi (funkcija \verb|enter_context()|), lai pamestu tvērumu (funkcija \verb|leave_context()|), lai pievienotu makro (\verb|add_match(regexp)|) un lai apstaigātu daļiņu virkni meklējot sakrišanas (\verb|match_stream(stream)|). Integrēšanai prototipu būs jāpapildina ar iespēju padot produkcijas tipu daļiņu virkņu apstrādes funkcijām. Prototipa daļiņu saņemšanas funkcijas būs jāpārslēdz uz parsētāja piedāvāto saskarni daļiņu dabūšanai.

Prototipam ir nepieciešama vienkārša saskarne no eksistējošā parsētāja. Parsētājam jādot pieeju pie daļiņu virknes lasīšanas, kā arī jāprot aizvietot atrastās daļiņu virknes ar citām virknēm, iespējams, ar citu garumu. Parsētājam arī jāprot pārstartēt daļiņu virknes lasīšanu no aizvietotas virknes sākuma. Prototipam nepieciešamā saskarne ir implementēta Eq parsētājā.

Apvienojot parsētāja un sakritību meklēšanas prototipu būs nepieciešams ievietot prototipa funkciju izsaukumus katras produkcijas apstrādes sākumā. Gadījumos, kas parsētājs sastapās ar daļiņu, kas identificē makro sākšanos, būs nepieciešams izsaukt regulārās izteiksmes parsēšanas funkciju. Savukārt, kad tiek apstrādātas citas produkcijas, būs nepieciešams izsaukt sakrišanu meklēšanu. Abu funkciju izsaukumos būs nepieciešams padot arī produkcijas tipu, lai prototips varētu atšķirt, kādu no automātiem papildināt vai lietot sakrišanu atrašanai. Prototipa funkcijas būs jāizsauc ari tvērumu pārslēgšanu brīdī, lai tas varētu implementēt tvērumu makro prioritāšu sadalīšanu.

\section{Rezultāti}
\subsection{Prototipa īpašības}
 Prototips pagaidām netiek integrēts Eq valodas kompilātorā, bet tas tiek plānots tuvākajā nākotnē.

\fixme{Šeit droši vien jāapraksta vairāk par beigu prototipa versiju, par to, ko viņa varēs darīt.} Cik tā ir efektīva?

\section{Secinājumi}

Tālāk darbs tik turpināts (šeit var pārfrāzēt Conclusions no raksta melnraksta).

%This paper describes a theoretical background for constructing an extensible dynamic macroprocessor and provides a view on solving shortcomings that ex- isting preprocessors have. We present to the judjment of the reader a parser model that includes facilities necessary for the dynamic extension constructing and the syntax we propose for describing macros using regular expression ele- ments. We also give several examples to demonstrate the described extension’s capabilities, which can be applied to perform metaprogramming techniques as well as language syntax extension.

%We have prototyped the described system and plan to develop it further on to create an actual usable example of the described approach. There are, for sure, still plenty of issues to tackle and resolve, however, we expect the project to be helpful in wide areas of language design, compiler optimisations and macro languages.

%\section{Random thoughts}
\subsection{Our goals}
Apskatāmās sistēmas 2 galvenie mērķi ir dot iespēju ieviest jaunas konstrukcijas un tajā pašā laikā saglabāt korektu jau iepriekšeksistējošās sintakses apstrādi.

\subsection{Why adaptable grammars are cool}
Adaptīvās gramatikas ir ļoti spēcīgs rīks kompilatoru un parsētāju būvēšanā. 
Static semantics - gandrīz sintakse, bet ne gluži

\subsection{Why adaptable grammars suck}
В общем к ним сложно написать формализм и они не особо применимы реально, потому что очень уж ограничены.

Адаптирующиеся грамматики в общем случае мало того, что требуют пересчитывания всей таблицы парсинга - омг, так ещё и могут накосячить с тем, что потеряется парсабельность самой грамматики - LL(1) или LALR() или ещё какая-нибудь, которая необходима для того, чтобы работал парсер. Следовательно фиг ты их прикрутишь в обычном виде к уже существующему парсеру \cite{Christiansen:SurveyAdaptableGrammars}

\bibliography{bachelors}{}
\bibliographystyle{plain}
\end{document}